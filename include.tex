\documentclass{article}
\usepackage{amsmath, amsthm, amssymb, amsfonts}
\usepackage{thmtools}
\usepackage{graphicx}
\usepackage{setspace}
\usepackage{geometry}
\usepackage{float}
\usepackage{amsthm}
\usepackage{hyperref}
\usepackage[utf8]{inputenc}
\usepackage[english]{babel}
\usepackage{framed}
\usepackage[dvipsnames]{xcolor}
\usepackage[skins,breakable]{tcolorbox}
\usepackage{awesomebox}
\usepackage{mathrsfs}  
\usepackage{xcolor}
\usepackage{wrapfig}
\usepackage{algorithm2e}
\RestyleAlgo{ruled}

\usepackage{pstricks-add}
\usepackage{epsfig}
\usepackage{pst-grad} % For gradients
\usepackage{pst-plot} % For axes
\usepackage[space]{grffile} % For spaces in paths
\usepackage{etoolbox} % For spaces in paths
\makeatletter % For spaces in paths
\patchcmd\Gread@eps{\@inputcheck#1 }{\@inputcheck"#1"\relax}{}{}
\makeatother

\colorlet{shadecolor}{blue!10}

\colorlet{LightGray}{White!90!Periwinkle}
\colorlet{LightOrange}{Orange!15}
\colorlet{LightGreen}{Green!15}

\newcommand{\HRule}[1]{\rule{\linewidth}{#1}}

\declaretheoremstyle[name=定理]{thmsty}
\declaretheorem[style=thmsty,numberwithin=section]{theorem}
\tcolorboxenvironment{theorem}{colback=LightGray,breakable}

\declaretheoremstyle[name=命题,]{prosty}
\declaretheorem[style=prosty,numberlike=theorem]{proposition}
\tcolorboxenvironment{proposition}{colback=LightOrange,breakable}

\declaretheoremstyle[name=原则,]{prcpsty}
\declaretheorem[style=prcpsty,numberlike=theorem]{principle}
\tcolorboxenvironment{principle}{colback=LightGreen,breakable}

\declaretheoremstyle[name=定义,]{prcpsty}
\declaretheorem[style=prcpsty,numberlike=theorem]{definition}
\tcolorboxenvironment{definition}{colback=white,breakable}

\declaretheoremstyle[name=推论,]{prcpsty}
\declaretheorem[style=prcpsty,numberlike=theorem]{corollary}
\tcolorboxenvironment{corollary}{colback=white,breakable}

\newtheorem{example}{例子}[section]

\usepackage{xsim}
\DeclareExerciseEnvironmentTemplate{custom}{%
  \begin{tcolorbox}[boxrule = 0pt]
  \tcbox[on line,colback=teal,colframe=teal,coltext=white,size=small]{%
    \faToggleOn\sffamily\bfseries\
    \XSIMmixedcase{\GetExerciseName}
    \GetExerciseProperty{counter}%
  }\quad
}{\hfill\textbf{$\#$}\end{tcolorbox}}

\xsimsetup{
  exercise/within = part,
  exercise/template = custom ,
  exercise/name = 练习
}


\newenvironment{remark}{%
  \par\medskip
  \noindent
  \textbf{注:}
}{%
  \par\medskip
}

\renewenvironment{solution}{%
  \par\medskip
  \noindent
  \textbf{解答:}
}{%
  \par\medskip
}

\newenvironment{solution*}{%
  \par\medskip
  \noindent 
  \small\textbf{提示或解答:}
}{%
  \par\medskip
}

\newenvironment{definition*}{%
  \par\medskip
  \noindent
  \textbf{定义:}
}{%
  \par\medskip
}

\newenvironment{lemma}{%
  \par\medskip
  \noindent
  \textbf{引理:}
}{%
  \par\medskip
}

\newenvironment{proposition*}{%
  \par\medskip
  \noindent
  \textbf{性质: }
}{%
  \par\medskip
}

\newtcolorbox{asidebox}{
  colback=gray!10,
  colframe=gray!60,
  fonttitle=\bfseries,
  title={Aside},
  breakable=true
}

\newtcolorbox{webaside}{
  colback=cyan!10,
  colframe=cyan!60,
  fonttitle=\bfseries,
  title={Web Demonstrate Aside},
  breakable=true
}

\usepackage{enumitem}

\setlist{nosep}

\setstretch{1.2}
\geometry{
    textheight=9in,
    textwidth=5.5in,
    top=1in,
    headheight=12pt,
    headsep=25pt,
    footskip=30pt
}

\usepackage{environ}
\usepackage[tikz]{bclogo}
\usepackage{tikz}
\usetikzlibrary{calc}
\NewEnviron{takeaway}
  {\par\medskip\noindent
  \begin{tikzpicture}
    \node[inner sep=0pt] (box) {\parbox[t]{.99\textwidth}{%
      \begin{minipage}{.3\textwidth}
      \centering\tikz[scale=5]\node[scale=3,rotate=30]{\bclampe};
      \end{minipage}%
      \begin{minipage}{.65\textwidth}
      \textbf{Takeaway Message}\par\smallskip
      \BODY
      \end{minipage}\hfill}%
    };
    \draw[red!75!black,line width=3pt] 
      ( $ (box.north east) + (-5pt,3pt) $ ) -- ( $ (box.north east) + (0,3pt) $ ) -- ( $ (box.south east) + (0,-3pt) $ ) -- + (-5pt,0);
    \draw[red!75!black,line width=3pt] 
      ( $ (box.north west) + (5pt,3pt) $ ) -- ( $ (box.north west) + (0,3pt) $ ) -- ( $ (box.south west) + (0,-3pt) $ ) -- + (5pt,0);
  \end{tikzpicture}\par\medskip%
}

\usepackage{marginnote}
\renewcommand*{\marginfont}{\color{gray}\ttfamily\small}
\usepackage{setspace}
\newcounter{paranum}[section]
\newcommand{\Par}[1]{\vspace{10pt}\noindent\textbf{\refstepcounter{paranum}\theparanum. }\textbf{#1}~~}
\newcommand{\lec}[1]{\reversemarginpar\marginnote{{\textbf{#1}}}}
\newcommand{\mn}[1]{\marginnote{{#1}}}
\renewcommand{\algorithmcfname}{算法}

\newcommand{\stirling}[2]{\left\{{#1 \atop #2}\right\}}
\newcommand{\binomt}[2]{\left(\left({#1 \atop #2}\right)\right)}
\newcommand{\pf}[4]{#1_{#2}^{#3_{#4}}}
\newcommand{\pl}[4]{#1_{#2}{#3^{#4}}}
\newcommand{\ty}[3]{{#1} \equiv {#2} ~(\bmod {#3})}
\newcommand{\Z}{{\mathbb Z}}
\newcommand{\one}{\mathbf{1}}
\newcommand{\varsub}[2]{\stackrel{#1}{\stackrel{\rule{#2}{0.4pt}}{\rule{#2}{0.4pt}}}}
\newcommand{\dd}{\mathrm{d}}
\newcommand{\Ep}[1]{\mathbb E[#1]}

\renewcommand{\red}[1]{{{\color{red}#1}}}
\newcommand{\teal}[1]{{{\color{teal}#1}}}
\renewcommand{\blue}[1]{{{\color{blue}#1}}}
\newcommand{\purple}[1]{{{\color{purple}#1}}}