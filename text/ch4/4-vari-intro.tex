\section{方差}
随机变量 $X$ 的方差, 表征 $X$ 取值的散布程度. 最简单的想法是: 可以使用$\Ep{|X-\Ep{X}|}$来表征. 但是, 由于带着绝对值在求导的时候会带来不少麻烦, 我们考虑使用平方来操作, 也就是$\Ep{(X-\Ep{X})^2}$. 

\begin{definition}
    定义 设 $X$ 是一个随机变量, 若 $E\left\{[X-E(X)]^2\right\}$ 存在, 则称 $\Ep{(X-\Ep{x})^2}$ 为 $X$ 的方差, 记为 $D(X)$ 或 $\operatorname{Var}(X)$, 即
$$
D(X)=\operatorname{Var}(X):=\Ep{(X-\Ep{X})^2}.
$$
\end{definition}

在应用上还引入量 $\sqrt{D(X)}$, 记为 $\sigma(X)$, 称为标准差或均方差.

其实际意义是明显的: 我们查看方差的大小
\begin{itemize}
    \item $D(X)$ 较小:  $X$ 的取值比较集中在 $\Ep X $ 的附近
    \item $D(X)$ 较大: 表示 $X$ 的取值较分散.
\end{itemize}

其实, 方差说特别也不特别, 它实际上就是随机变量 $X$ 的函数 $g(X)=(X-E(X))^2$的期望. 

对这个式子做一些简单的化简, 我们发现, 要计算一个数的方差, 我们只需要计算$\Ep{X}$以及$\Ep{X^2}$即可. 

   \begin{proposition}
       随机变量 $X$ 的方差可按下列公式计算.
   $$
   D(X)=E\left(X^2\right)-[E(X)]^2 .
   $$
   \end{proposition}
   
   其证明如下: 
   \begin{proof}
       由数学期望的性质 $1^{\circ}, 2^{\circ}, 3^{\circ}$ 得
   $$
   \begin{aligned}
   D(X) & =E\left\{[X-E(X)]^2\right\}=E\left\{X^2-2 X E(X)+[E(X)]^2\right\} \\
   & =E\left(X^2\right)-2 E(X) E(X)+[E(X)]^2 \\
   & =E\left(X^2\right)-[E(X)]^2 .
   \end{aligned}
   $$
      \end{proof}




   \subsection{方差的性质}

   我们首先从抽象的层面说明方差具有哪些性质. 实际上, 这只是个关于期望的性质的练习 -- 方差只是求解一堆数学期望而已. 我们先假设下面所遇到的随机变量其方差存在: 

   \paragraph{1. 常数的方差是0} 设 $C$ 是常数, 则 $D(C)=0$.
   \begin{proof}
    代入定义,有 $D(C)=E\left\{[C-E(C)]^2\right\}=0$.
   \end{proof}


   \paragraph{2. 一个变量的线性变换后的方差} 设 $X$ 是随机变量, $C$ 是常数, 则有
   $$
   D(C X)=C^2 D(X), \quad D(X+C)=D(X) .
   $$

   \begin{proof}
    
    $$
\begin{aligned}
D(C X) & =E\left\{[C X-E(C X)]^2\right\}=C^2 E\left\{[X-E(X)]^2\right\}=C^2 D(X) \\
D(X+C) & =E\left\{[X+C-E(X+C)]^2\right\}=E\left\{[X-E(X)]^2\right\}=D(X)
\end{aligned}
$$
   \end{proof}

   \paragraph{3. 两个随机变量和的方差} 设 $X, Y$ 是两个随机变量, 则有\mn{可不要忘记了这个2. !}
   $$
   D(X+Y)=D(X)+D(Y)+2 E\{(X-E(X))(Y-E(Y))\} .
   $$
   
   特别, 若 $X, Y$ 相互独立, 则有
   $$
   D(X+Y)=D(X)+D(Y) .
   $$
   
   这一性质可以推广到任意有限多个相互独立的随机变量之和的情况.

   \begin{proof}
    $$
\begin{aligned}
D(X+Y)= & E\left\{[(X+Y)-E(X+Y)]^2\right\} \\
= & E\left\{[(X-E(X))+(Y-E(Y))]^2\right\} \\
= & E\left\{(X-E(X))^2\right\}+E\left\{(Y-E(Y))^2\right\} \\
& +2 E\{[(X-E(X)][Y-E(Y)]\} \\
= & D(X)+D(Y)+2 E\{[X-E(X)][Y-E(Y)]\} .
\end{aligned}
$$

对于其中的$ E\{[(X-E(X)][Y-E(Y)]\}$一项: $$
\begin{aligned}
& 2 E\{[(X-E(X)][Y-E(Y)]\} \\
& =2 E\{X Y-X E(Y)-Y E(X)+E(X) E(Y)\} \\
& =2\{E(X Y)-E(X) E(Y)-E(Y) E(X)+E(X) E(Y)\} \\
& =2\{E(X Y)-E(X) E(Y)\} .
\end{aligned}
$$
   \end{proof}

   

    \subsection{常见分布的方差}
    \paragraph{1. 均匀分布} 由于期望$\Ep{X}$我们已经求过了, 我们来看平方的期望$\Ep{X^2}$. 经过简单的计算, 我们发现$$
    E\left(X^2\right)=\int_{-\infty}^{\infty} x^2 f(x) d x=\frac{b^2+a b+a^2}{3}
    $$
    
    因此
    $$
    D(X)=E\left(X^2\right)-E(X)^2=\frac{(b-a)^2}{12}.
    $$

    \paragraph{2. Poisson 分布} 我们回忆参数为$\lambda$的Poisson分布的概率分布是$f(k )=\frac{e^{-\lambda} \lambda^k}{k !}$. 同样的, 
    $$
    \begin{aligned}
    E\left(X^2\right) & =\sum_{k=0}^{\infty} k^2 \frac{\lambda^k e^{-\lambda}}{k !}=\sum_{k=1}^{\infty} k e^{-\lambda} \lambda \frac{\lambda^{k-1}}{(k-1) !} \\
    & =\sum_{k=1}^{\infty}(k-1+1) \lambda e^{-\lambda} \frac{\lambda^{k-1}}{(k-1) !} \\
    & =\sum_{k=1}^{\infty}(k-1) \lambda e^{-\lambda} \frac{\lambda^{k-1}}{(k-1) !}+\lambda e^{-\lambda} \sum_{k=1}^{\infty} \frac{\lambda^{k-1}}{(k-1) !} \\
    & =\lambda^2 e^{-\lambda} \sum_{k=2}^{\infty} \frac{\lambda^{k-2}}{(k-2) !}+\lambda \\
    & =\lambda^2+\lambda \\
    D(X) & =\lambda^2+\lambda-\lambda^2=\lambda
    \end{aligned}
    $$

    \paragraph{3. 指数分布} 回忆它的概率分布函数为$$
    f(x)=\lambda e^{-\lambda x}, x \geq 0
    $$并且$\Ep{X}=1/\lambda$. 我们还是关注$\Ep{X^2}$. 
    $$
\begin{aligned}
E X^2 & =\int_0^{+\infty} x^2 \lambda e^{-\lambda x} d x \\
& =-\int_0^{+\infty} x^2 d e^{-\lambda x}=-\left.x^2 e^{-\lambda x}\right|_0 ^{+\infty}+\int_0^{+\infty} 2 x e^{-\lambda x} d x 
\end{aligned}
$$

因此
$$
\operatorname{Var}(\mathrm{X})=E X^2-(E X)^2=\frac{1}{\lambda^2}
$$

\paragraph{4. 正态分布} 我们发现正态分布的通用表达式比较麻烦. 我们首先计算标准正态分布的方差, 然后试图通过线性变换的方法得到原来的正态分布表达式. 回忆$N(0,1)=\varphi(x)=\frac{1}{\sqrt{2 \pi}} e^{-\frac{x^2}{2}}$. 

计算的大概过程大概如下: 
\begin{itemize}
    \item 写出表达式: $E\left(X^2\right)=\int_{-\infty}^{+\infty} \frac{1}{\sqrt{2 \pi}} x^2 e^{-\frac{x^2}{2}} d x$
    \item 由于是偶函数, $E\left(X^2\right)=2 \int_0^{+\infty} \frac{1}{\sqrt{2 \pi}} x^2 e^{-\frac{x^2}{2}} d x$
    \item 希望计算出$\int_0^{+\infty} \frac{1}{\sqrt{2 \pi}} x^2 e^{-\frac{x^2}{2}} d x$.
    \item 极坐标换元, 得到值为$1/2$.
    \item 因此$\Ep{X^2}=1$.
    \item $D(X)=E\left(X^2\right)-(E(X))^2=1-0^2=1$
    \item 通过标准化得到方差
\end{itemize}

接着我们做线性变换代换回去. 
$$
\begin{aligned}
& X \sim N\left(\mu, \sigma^2\right), Z \sim N(0,1) \\
& Z=\frac{X-\mu}{\sigma} \sim N(0,1) \\
& D(Z)=D\left(\frac{X-\mu}{\sigma}\right)=D\left(\frac{X}{\sigma}-\frac{\mu}{\sigma}\right)=D\left(\frac{X}{\sigma}\right)=\frac{1}{\sigma^2} D(X)=1
\end{aligned}
$$
得到$D(x)=\sigma^2$.

\paragraph{5. 几何分布 } 我们回忆其分布函数为$P(X=k)=(1-p)^{k-1} p$ $\frac{1}{p}$, 以及$\Ep X=1/p$. 方便起见, 令$q:=1-p$. 

那么
$$\begin{aligned} D(X) & =E\left(X^2\right)-(E(X))^2 \\ & =\sum_{k=1}^{\infty} k^2 q^{k-1} p-\frac{1}{p^2} \\ & =p\left[\sum_{k=1}^{\infty}(k+1) k q^{k-1}-\sum_{k=1}^{\infty} k q^{k-1}\right]-\frac{1}{p^2} \\ & =p\left(\frac{\mathrm{d}^2}{\mathrm{~d} q^2} \sum_{k=1}^{\infty} q^{k+1}-\frac{\mathrm{d}}{\mathrm{d} q} \sum_{k=1}^{\infty} q^k\right)-\frac{1}{p^2} \\ & =p\left[\frac{2}{(1-q)^3}-\frac{1}{(1-q)^2}\right]-\frac{1}{p^2}=\frac{q}{p^2} .\end{aligned}$$

\paragraph{6. 二项分布} 
我们在上一节里面用非常痛苦的方法求解了它的期望. 这里, 我们还是先不用任何的性质, 再来痛苦地求解一下它的方差. 

$$\begin{aligned} E\left(k^2\right) & =\sum_{k=0}^n k^2 p(k) \\ & =\sum_{k=1}^n k^2\left(\begin{array}{l}n \\ k\end{array}\right) p^k q^{n-k} \\ & =\sum_{k=1}^n[{k}({k}-1)+{k}]\left(\begin{array}{l}n \\ k\end{array}\right) p^k q^{n-k} \\ & =\sum_{k=1}^n k(k-1)\left(\begin{array}{l}n \\ k\end{array}\right) p^k q^{n-k}+\sum_{k=1}^n k\left(\begin{array}{l}n \\ k\end{array}\right) p^k q^{n-k} \\ & =\sum_{k=1}^n k(k-1) \frac{n !}{k !(n-k) !} p^k q^{n-k}+n p \\ & =n(n-1) p^2 \sum_{k=1}^n \frac{(n-2) !}{(k-2) !(n-k) !} p^{k-2} q^{(n-2)-(k-2)}+n p \\ & =n(n-1) p^2+n p\end{aligned}$$

因此$D (X)=\Ep {X^2}-(\Ep X)^2=n(n-1) p^2+n p-(n p)^2=n p(1-p)$. 

这和我们期望的性质也是吻合的. 毕竟, 这就是$n$次独立的0-1分布. 我们的结果也是支持这一点的. 


\paragraph{总结}

\begin{table}
    \begin{tabular}{|c|c|c|c|}
        \hline 分布名称 & PDF & 均值 & 方差 \\
        \hline 二项分布 & $P(X=k)=C_n^kp^k(1-p)^{n-k}$ & $n p$ & $n p(1-p)$ \\
        \hline 几何分布 & $P(X=k)=(1-p)^{k-1} p$ & $\frac{1}{p}$ & $\frac{(1-p)}{p^2}$ \\
        \hline 正态分布 & $f\left(x \mid \mu, \sigma^2\right)=\frac{1}{\sqrt{2 \pi \sigma^2}} e^{-\frac{(x-\mu)^2}{2 \sigma^2}}$ & $\mu$ & $\sigma^2$ \\
        \hline 均匀分布 & $f(x \mid a, b)= \begin{cases}\frac{1}{b-a} & \text { for } a \leq x \leq b \\
            0 & \text { for } x<a \text { or } x>b\end{cases}$ & $\frac{a+b}{2}$ & $\frac{(b-a)^2}{12}$ \\
            \hline 指数分布 & $f(x )=\lambda e^{-\lambda x}$ & $\frac{1}{\lambda}$ & $\frac{1}{\lambda^2}$ \\
            \hline Poisson 分布 & $f(k )=\frac{e^{-\lambda} \lambda^k}{k !}$ & $\lambda$ & $\lambda$ \\
            \hline
        \end{tabular}
        \caption{常见分布的均值和方差}
        \label{table:mean-var}
\end{table}
总的来说, 我们有\cref{table:mean-var}的表格总结我们的结果. 


\subsection{Chebyshev不等式}

利用随机变量的期望和方差, 我们可以推出Chebushev不等式, 来得到更好的界限. 

\begin{theorem}
    对于任意的$a>0$, 有
    $$
    P(|X-\Ep{X}|\geq a)\leq \frac{D(X)}{a^2}
    $$
\end{theorem}

\begin{proof}
    由于有绝对值, 我们首先将绝对值去掉, 得到
    $$
P(|X-\mathbb{E}[X]| \geq a)=P\left((X-\mathbb{E}[X])^2 \geq a^2\right)
$$
由于$(X-\Ep{X})^2$是一个非负的变量, 使用Markov不等式得到: 
$$P\left((X-\mathbb{E}[X])^2 \geq a^2\right) \leq \frac{\mathbb{E}\left[(X-\mathbb{E}[X])^2\right]}{a^2}=\frac{\operatorname{Var}[X]}{a^2}$$


\end{proof}

这个定理在说什么? 这意味着, 它与期望相差的程度的概率可以使用方差的相关性质来描述. 

有了这个定理, 我们可以得到关于方差的另一个重要的性质: 

\paragraph{4. 方差为0的充要条件} $\quad D(X)=0$ 的充要条件是 $X$ 以概率 1 取常数 $E(X)$, 即
$$
P(X=E(X))=1
$$

\begin{proof}
    充分性. 设 $P(X=E(X)\}=1$, 则有 $P\left\{X^2=[E(X)]^2\right)=1$, 于是
    $$
    D(X)=E\left(X^2\right)-[E(X)]^2=0 .
    $$
    
    必要性: 用反证法假设 $P(X=E(X)\}<1$, 则对于某一个数 $\varepsilon>0$, 有 $P(|X-E(X)| \geq \varepsilon)>0$, 但由切比雪夫不等式, 对于任意 $\varepsilon>0$, 由 刚才的不等关系$P\{|X-\mu| \geq \varepsilon) \leq \frac{\sigma^2}{\varepsilon^2}$, $\sigma^2=$ 0 , 由于有
    $$
    P(|X-E(X)| \geq \varepsilon)=0,
    $$
    矛盾, 于是 $P(X=E(X))=1$.
\end{proof}