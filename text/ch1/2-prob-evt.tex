% !TEX root = main.tex
\section{事件的概率}

事件的\textbf{概率}:刻画试验中随机事件发生的\textbf{可能性大小}.

\subsection{概率的统计定义}
\begin{definition}
    设在$n$次试验中,事件$A$发生了$m$次,则称
    \begin{align*}
        f_n(A):=\frac{m}{n}
    \end{align*}
    为事件$A$发生的\textbf{频率}(frequency).
\end{definition}

\begin{definition*}%[概率的统计定义]
    在相同条件下重复进行的试验中,若随着试验次数$n$的增加,
    事件$A$发生的频率稳定在某一常数$p$附近,
    则称$p$为事件$A$的\textbf{概率},记作$P(A)=p$.
\end{definition*}
也就是概率是频率的稳定值. 实际应用中常将大量重复试验中事件的频率作为概率的近似估计.

\begin{proposition*}
    频率的性质:
    \begin{itemize}
        \item $0\le f_n(A)\le 1$;
        \item $f_n(\Omega)=1,\ f_n(\emptyset)=0$;
        \item 若事件$A_1, A_2, \cdots, A_k$两两互斥,则
              $$f_n \left( \bigcup_{i=1}^k A_i \right)=\sum_{i=1}^k f_n(A_i)$$
    \end{itemize}
\end{proposition*}

由于上述的性质, 我们给出概率的数学公理化定义:
\begin{definition}[概率的公理化定义]
    设$\Omega$是样本空间,定义概率空间$(\Omega,\mathcal{F},P)$。对每个事件$A\in \mathcal{F}$定义一个实数$P(A)$与之对应。
    集合函数$P$满足以下条件:
    \begin{itemize}
        \item 非负性:对任意事件$A$,均有$P(A)\ge 0$;
        \item 规范性:$P(\Omega)=1$;
        \item 可加性:若事件序列$\{A_n\}_{n\ge 1}$两两互斥,则
              $$P \left( \bigcup_{n=1}^{\infty} A_n \right)=\sum_{n=1}^{\infty} P(A_n)$$
    \end{itemize}
    则称$P(A)$为事件$A$的\textbf{概率}(probability).
\end{definition}

\begin{takeaway}
    概率是在样本空间上面定义的一个函数, 满足: (1)非负性: 每个事件的概率必须大于等于0; (2)规范性
    所有的事件概率``总和''等于1; (3) 可加性: 互斥事件的概率可以直接相加.
\end{takeaway}
    


\subsection{概率的加法公式}
我们已经在互斥的时候, 规定了其概率的过程. 下面我们来看一看不互斥的情形.
\begin{proposition}[加法公式]
    若两个事件$A,B$互斥,则
    $$P(A\cup B)=P(A)+P(B).$$
\end{proposition}

\begin{remark}
    由加法公式可得到如下性质:
    \begin{itemize}
        \item 对任意事件$A$,有
              $P(A)=1-P\left(\overline{A}\right).$
        \item 对任意两个事件$A,B$,有
              $$P(A\cup B)=P(A)+P(B)-P(AB).$$
    \end{itemize}
\end{remark}

\begin{asidebox}
    我们实际上可以借此瞥见容斥原理.
    \begin{remark}
        若三个事件$A_1, A_2, A_3$两两互斥,则
        $$\pmb P(A_1 \cup A_2 \cup A_3) = P(A_1)+P(A_2)+P(A_3).$$
        对任意三个事件$A_1, A_2, A_3$,有
        \begin{align*}
            \pmb P(A_1 \cup A_2 \cup A_3) & \pmb=\color{blue}P(A_1)+P(A_2)+P(A_3)              \\
                                          & \phantom=\color{red}-P(A_1A_2)-P(A_1A_3)-P(A_2A_3) \\
                                          & \phantom=\color{blue}+P(A_1A_2A_3).
        \end{align*}%
    \end{remark}

    \begin{remark}
        更一般地, 可以使用容斥原理计算:
        若$n$个事件$A_1, A_2, \cdots, A_n$两两互斥,则
        $$\pmb P\left( \bigcup_{i=1}^n A_i \right)=\sum_{i=1}^n P(A_i).$$
        \vspace{0.2in}
        对任意$n$个事件$A_1, A_2, \cdots, A_n$,有
        $$\pmb P\left( \bigcup_{i=1}^n A_i \right)=\sum_{k=1}^n \left[ (-1)^{k+1} \sum_{1\le i_1\le \cdots\le i_k \le n} P(A_{i_1}\cdots A_{i_k}) \right].$$
    \end{remark}


\end{asidebox}

\subsection{古典概型模型}
\begin{definition}
    如果一个随机试验具有以下特点:
    \begin{itemize}%\setcounter{enumi}{2}
        \item 样本空间只含有限多个样本点;
        \item 各样本点出现的可能性相等,
    \end{itemize}
    则称此随机试验是古典型的.此时对每个事件$A\subset \Omega$,
    \begin{align*}
        P(A)=\frac{\mbox{事件$A$包含的样本点数}}{\mbox{样本点的总数}}=\frac{n(A)}{n(\Omega)}
    \end{align*}
    称为事件$A$的\textbf{古典概率}.
\end{definition}

根据上述的定义, 我们可以立即得出$P(\emptyset)=0$,$P(\Omega)=1$.



\input{aside/2star-counting-techique.tex}

\subsection{几何概型}

\begin{definition}
    设样本空间为有限区域$\Omega$,若样本点落入$\Omega$内的任何区域$G$中的概率与区域$G$的测度成正比,则样本点落入$G$内的概率为:
    $$
        p=\frac{\vert G\vert}{\vert \Omega \vert}
    $$
\end{definition}

并且我们发现, $P(A)=0$, 则$A$不一定是不可能事件.




