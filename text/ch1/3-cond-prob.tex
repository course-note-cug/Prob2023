\section{条件概率}

\begin{definition}
    设$P(A)>0$, 称
    \begin{align*}
        P(B|A):=\frac{P(AB)}{P(A)}
    \end{align*}
    为在\textbf{事件$A$发生条件下, 事件$B$的}\newword{条件概率}{conditional probability}. (记作 $P(B \mid A)$ )
\end{definition}

比如, 在古典概率模型中, 
\begin{align*}
    P(B|A)=\frac{\mbox{事件$AB$包含的样本点数}}{\mbox{事件$A$包含的样本点数}}=\frac{n(AB)}{n(A)}.
\end{align*}

条件概率也是概率, 因此它也具有概率的性质:

$$
\begin{aligned}
& P(A \mid A)=1 \\
& P(\varnothing \mid A)=0 \\
& P(B \mid A)=1, \quad B \supseteq A \\
& P\left(B_1+B_2 \mid A\right)=P\left(B_1 \mid A\right)+P\left(B_2 \mid A\right), B_1,B_2\text{互斥}
\end{aligned}
$$

更抽象的, 我们有

\begin{proposition}
    设$P(A)>0$, 则
    \begin{itemize}
        \item 对任意事件$B$, 均有$P(B|A)\ge 0$; 
        \item $P(\Omega|A)=1$; 
        \item 若事件序列$\{B_n\}_{n\ge 1}$两两互斥, 则
              $$P\left( \left. \bigcup_{n=1}^{\infty} B_n \right| A\right)=\sum_{n=1}^{\infty} P(B_n|A).$$
    \end{itemize}
\end{proposition}



\subsection*{全概率公式}

由于条件概率, 我们有概率的乘法公式:

\begin{theorem}[乘法公式]
    由条件概率的定义, 得到
    \begin{itemize}
        \item 如果$P(A)>0$, 则有$P(AB)=P(A)P(B|A)$. 
        \item 如果$P(B)>0$, 则有$P(AB)=P(B)P(A|B)$. 
    \end{itemize}
\end{theorem}

我们同样可以把这个性质推广到$n$个物品的时候.

\begin{corollary}
    如果$P(A_1 A_2\cdots A_{n-1})>0$, 则有\textbf{乘法公式}
    \begin{align*}
          & P(A_1A_2\cdots A_n)                                                \\
        = & P(A_1)P(A_2|A_1)P(A_3|A_1 A_2)\cdots P(A_n|A_1 A_2\cdots A_{n-1}).
    \end{align*}
\end{corollary}



另一个简单而重要的公式称做全概率公式, 是利用条件概率计算复合事件的基本工具. 我们首先希望对样本空间进行划分. 然后再进行求解:

\begin{definition}
    设$\Omega$为某试验的样本空间, $A_1, A_2, \cdots, A_n$为一组事件. 如果以下条件成立:
    \begin{itemize}
        \item $A_1, A_2, \cdots, A_n$两两互斥, 
        \item $A_1 \cup A_2 \cup \cdots \cup A_n=\Omega$, 
    \end{itemize}
    则称$A_1, A_2, \cdots , A_n$为样本空间$\Omega$的一个\textbf{划分}. 
    %或称$A_1, A_2, \cdots, A_n$为一个完备事件组. 
\end{definition}


有了这样的分类之后, 我们就可以给出全概率了.

\begin{theorem}[全概率公式]
    如果$A_1, A_2, \cdots, A_n$ 是样本空间的划分, 且都有正概率, 则对任意事件$B$有
    \begin{align*}
        P(B)=\sum_{i=1}^n P(A_i) P(B|A_i).
    \end{align*}
\end{theorem}

\begin{proof}
    考虑基本事件空间 $\Omega$ 的某个分割 $\mathscr{D}=\left\{A_1, \cdots, A_n\right\}$, 且 $P\left(A_i\right)>0(i=1$, $2, \cdots, n)$. (这样的分割又称做不相容事件的完全事件组.) 显然,
$$
B=B A_1+\cdots+B A_n,
$$

因此
$$
P(B)=\sum_{i=1}^n P\left(B A_i\right),
$$

其中
$$
P\left(B A_i\right)=P\left(B \mid A_i\right) P\left(A_i\right)
$$
\end{proof}


\begin{example}
    假设要对研究生论文抄袭现象进行社会调查, 我们设计两个具有\textbf{相同答案}的问题:
    \begin{itemize}
        \item 你的生日是否在7月1日以前?
        \item 你做论文时是否有过抄袭行为?
    \end{itemize}
    同时提供给受访者一个放有等量红球和白球的袋子, 
    受访者在不被观察的情况下从袋子中随机取一个球观察颜色后放回. 
    如果是红球回答第一个问题, 白球回答第二个问题. 

    假定受访者有150人, 统计出共有60个回答``是''. 问:有抄袭行为的比率是多少?
\end{example}

\begin{solution}
    事件$A$表示抽到白球, 事件$B$表示回答是, 则有
    $$P(B)=P(A)P(B|A)+P(\overline{A})P(B|\overline{A}).$$
    代入已知的概率, 得到
    $$\frac{60}{150}=\frac12\cdot P(B|A)+\frac12\cdot\frac12 $$
    求得
    $$P(B|A)=\frac{3}{10}$$
\end{solution}

\subsection{Bayes公式}
设事件 $A$ 和 $B$ 的概率大于 $0: P(A)>0, P(B)>0$, 利用乘法公式, 我们可以先看$B$而非$A$, 得到: 
$$P(A B)=P(A \mid B) P(B).$$

和刚刚得到的$P(A B)=P(B \mid A) P(A)$对比, 得到了$$P(A \mid B)=\frac{P(A) P(B \mid A)}{P(B)}.$$

\begin{theorem}[Bayes定理]
    设$0<P(A)<1$, $P(B)>0$, 则有
    $$P(A|B)=\frac{ P(AB)}{ P(B)}
        =\frac{P(A) P(B|A)}{P(A) P(B|A)+P(\overline{A})P(B|\overline{A})}$$
\end{theorem}

\begin{webaside}
    著名的科普视频频道主3Blue1Brown曾经对Bayes定律进行了可视化. 可以参考\href{https://www.bilibili.com/video/BV1R7411a76r}{Bilibili: BV1R7411a76r}.
\end{webaside}
假如事件组 $A_1, \cdots, A_n$ 是 $\Omega$ 的一个分割,那么有

\begin{corollary}
    如果$A_1, A_2, \cdots A_n$ 是样本空间的一个划分, 且都有正概率, 则对任意正概率的事件$B$有
    \[
        P(A_i|B)=\frac{P(A_i)P(B|A_i)}{P(A_1)P(B|A_1)+\cdots+P(A_n)P(B|A_n)}.%,\ i=1,\cdots,n
    \]
\end{corollary}

    实际上, 在统计应用中, 事件 $A_1, \cdots, A_n$ 组成事件组 $\left(A_1+\cdots+A_n=\Omega\right)$, 常称做 ``假设'' 或 ``假说'', 而 $P\left(A_i\right)$ 称做假设 $A_i$ 的\emph{先验}概率 $\left.\right)$. 条件概率 $P\left(A_i \mid B\right)$ 称做假设 $A_i$ 在事件 $B$ 出现后的\emph{后验}概率. 


\begin{exercise}
    假设匣中有两枚硬币: $A_1$ 是一对称的硬币, ``正面'' $\mathrm{Z}$ 出现的概率等于 $1 / 2$, 而 $A_2$ 是一枚不对称的硬币, ``正面'' $\mathrm{Z}$ 出现的概率等于 $1 / 3$. 随意选出一枚硬币并将其投掷, 结果掷出正面. 问抽到硬币为对称硬币的概率如何?
\end{exercise}

\begin{solution}
    建立相应的概率模型. 这里自然取集合 $\Omega=\left\{A_1 \mathrm{Z}, A_1 \mathrm{~F}, A_2 \mathrm{Z}, A_2 \mathrm{~F}\right\}$, 可以描绘选取和投掷的结局, 其中 $A_1 \mathrm{Z}$ 表示 ``选中硬币'' $A_1$, 结果掷出正面 $\mathrm{Z}$, 等等, 而 $\mathrm{F}$ 表示硬币掷出反面. 根据条件, 所考虑结局的概率应该是:
$$
P\left(A_1\right)=P\left(A_2\right)=\frac{1}{2}
$$

和
$$
P\left(\mathrm{Z} \mid A_1\right)=\frac{1}{2}, \quad P\left(\mathrm{Z} \mid A_2\right)=\frac{1}{3} .
$$

这些条件唯一决定各结局的概率:
$$
P\left(A_1 \mathrm{Z}\right)=\frac{1}{4}, P\left(A_1 \mathrm{~F}\right)=\frac{1}{4}, P\left(A_2 \mathrm{Z}\right)=\frac{1}{6}, P\left(A_2 \mathrm{~F}\right)=\frac{1}{3}
$$

那么, 根据贝叶斯公式, 所求的概率为
$$
P\left(A_1 \mid \mathrm{Z}\right)=\frac{P\left(A_1\right) P\left(\mathrm{Z} \mid A_1\right)}{P\left(A_1\right) P\left(\mathrm{Z} \mid A_1\right)+P\left(A_2\right) P\left(\mathrm{Z} \mid A_2\right)}=\frac{3}{5}
$$
\end{solution}

\begin{exercise}
    袋子中有10个白球, 5个黑球. 现掷一枚均匀的骰子. 掷出几点就从袋中取几个球. 若已知取出的球全为白球, 求掷出3点的概率. 
\end{exercise}

\begin{solution}
    原问题的意思是在取出的球全为白球的条件下, 掷出三点的概率. 设$B=\{\text{取出的球全是白球}\}$, $A=\{\text{掷出}i\text{点}\}(i=1,2,\cdots, 6)$.

    \begin{align*}
        P(A_3 | B) &= \frac{P(A_3)P(B|A_3)}{P(A_1)P(B|A_1)+P(A_2)P(B|A_2)+\cdots+P(A_6)P(B|A_6)} \\
        &= \frac{\frac16 \times \frac{\binom 53}{\binom {15}3}}{\sum_{i=1}^5 \frac16\times \frac{\binom 5i}{\binom{15}i}+\frac16\times 0}=0.4835
    \end{align*}
\end{solution}


