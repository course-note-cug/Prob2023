\section{条件概率}

\begin{definition}
    设$P(A)>0$,称
    \begin{align*}
        P(B|A):=\frac{P(AB)}{P(A)}
    \end{align*}
    为在\textbf{事件$A$发生条件下,事件$B$的条件概率}.(记作 $P(B \mid A)$ )
\end{definition}

比如, 在古典概率模型中,
\begin{align*}
    P(B|A)=\frac{\mbox{事件$AB$包含的样本点数}}{\mbox{事件$A$包含的样本点数}}=\frac{n(AB)}{n(A)}.
\end{align*}

条件概率也是概率, 因此它也具有概率的性质:

$$
\begin{aligned}
& \mathbf{P}(A \mid A)=1 \\
& \mathbf{P}(\varnothing \mid A)=0 \\
& \mathbf{P}(B \mid A)=1, \quad B \supseteq A \\
& \mathbf{P}\left(B_1+B_2 \mid A\right)=\mathbf{P}\left(B_1 \mid A\right)+\mathbf{P}\left(B_2 \mid A\right), B_1,B_2\text{互斥}
\end{aligned}
$$

更抽象的, 我们有

\begin{proposition}
    设$P(A)>0$,则
    \begin{itemize}
        \item 对任意事件$B$,均有$P(B|A)\ge 0$;
        \item $P(\Omega|A)=1$;
        \item 若事件序列$\{B_n\}_{n\ge 1}$两两互斥,则
              $$P\left( \left. \bigcup_{n=1}^{\infty} B_n \right| A\right)=\sum_{n=1}^{\infty} P(B_n|A)$$
              .
    \end{itemize}
\end{proposition}

由这些性质可见, 对于固定的事件 $A$, 在概率空间 $(\Omega \cap A, \mathscr{B} \cap A)$ 上的条件概率 $\mathbf{P}(\cdot \mid A)$, 以及在空间 $(\Omega, \mathscr{C})$ 上的概率 $\mathbf{P}(\cdot)$ 具有同样的性质, 其中
$$
\mathscr{A} \cap A=\{B \cap A: B \in \mathscr{B}\}
$$

\subsection*{全概率公式}

由于条件概率, 我们有概率的乘法公式:

\begin{theorem}[乘法公式]
    由条件概率的定义,得到
    \begin{itemize}
        \item 如果$P(A)>0$,则有$P(AB)=P(A)P(B|A)$.
        \item 如果$P(B)>0$,则有$P(AB)=P(B)P(A|B)$.
    \end{itemize}
\end{theorem}

我们同样可以把这个性质推广到$n$个物品的时候.

\begin{corollary}
    如果$P(A_1 A_2\cdots A_{n-1})>0$,则有\textbf{乘法公式}
    \begin{align*}
          & P(A_1A_2\cdots A_n)                                                \\
        = & P(A_1)P(A_2|A_1)P(A_3|A_1 A_2)\cdots P(A_n|A_1 A_2\cdots A_{n-1}).
    \end{align*}
\end{corollary}



另一个简单而重要的公式称做全概率公式, 是利用条件概率计算复合事件的基本工具. 我们首先希望对样本空间进行划分. 然后再进行求解:

\begin{definition}
    设$\Omega$为某试验的样本空间,$A_1, A_2, \cdots, A_n$为一组事件.如果以下条件成立:
    \begin{itemize}
        \item $A_1, A_2, \cdots, A_n$两两互斥,
        \item $A_1 \cup A_2 \cup \cdots \cup A_n=\Omega$,
    \end{itemize}
    则称$A_1, A_2, \cdots , A_n$为样本空间$\Omega$的一个\textbf{划分}.
    %或称$A_1, A_2, \cdots, A_n$为一个完备事件组.
\end{definition}


有了这样的分类之后, 我们就可以给出全概率了.

\begin{theorem}[全概率公式]
    如果$A_1, A_2, \cdots, A_n$ 是样本空间的划分,且都有正概率,则对任意事件$B$有
    \begin{align*}
        P(B)=\sum_{i=1}^n P(A_i) P(B|A_i).
    \end{align*}
\end{theorem}

\begin{proof}
    考虑基本事件空间 $\Omega$ 的某个分割 $\mathscr{D}=\left\{A_1, \cdots, A_n\right\}$, 且 $\mathbf{P}\left(A_i\right)>0(i=1$, $2, \cdots, n)$. (这样的分割又称做不相容事件的完全事件组.) 显然,
$$
B=B A_1+\cdots+B A_n,
$$

因此
$$
\mathbf{P}(B)=\sum_{i=1}^n \mathbf{P}\left(B A_i\right),
$$

其中
$$
\mathbf{P}\left(B A_i\right)=\mathbf{P}\left(B \mid A_i\right) \mathbf{P}\left(A_i\right)
$$
\end{proof}


\begin{example}
    假设要对研究生论文抄袭现象进行社会调查,我们设计两个具有\textbf{相同答案}的问题:
    \begin{itemize}
        \item 你的生日是否在7月1日以前?
        \item 你做论文时是否有过抄袭行为?
    \end{itemize}
    同时提供给受访者一个放有等量红球和白球的袋子,
    受访者在不被观察的情况下从袋子中随机取一个球观察颜色后放回.
    如果是红球回答第一个问题,白球回答第二个问题.

    假定受访者有150人,统计出共有60个回答“是”.问:有抄袭行为的比率是多少?
\end{example}

\begin{solution}
    事件$A$表示抽到白球,事件$B$表示回答是,则有
    $$P(B)=P(A)P(B|A)+P(\overline{A})P(B|\overline{A}).$$
    代入已知的概率,得到
    $$\frac{60}{150}=\frac12\cdot P(B|A)+\frac12\cdot\frac12 $$
    求得
    $$P(B|A)=\frac{3}{10}$$
\end{solution}

\subsection{Bayes公式}
设事件 $A$ 和 $B$ 的概率大于 $0: \mathbf{P}(A)>0, \mathbf{P}(B)>0$, 利用乘法公式, 我们可以先看$B$而非$A$, 得到: 
$$\mathbf{P}(A B)=\mathbf{P}(A \mid B) \mathbf{P}(B).$$

和刚刚得到的$\mathbf{P}(A B)=\mathbf{P}(B \mid A) \mathbf{P}(A)$对比, 得到了$$\mathbf{P}(A \mid B)=\frac{\mathbf{P}(A) \mathbf{P}(B \mid A)}{\mathbf{P}(B)}.$$

\begin{theorem}[Bayes定理]
    设$0<P(A)<1$,$P(B)>0$,则有
    $$P(A|B)=\frac{P(AB)}{P(B)}
        =\frac{P(A)P(B|A)}{P(A)P(B|A)+P(\overline{A})P(B|\overline{A})}$$
\end{theorem}

假如事件组 $A_1, \cdots, A_n$ 是 $\Omega$ 的一个分割,那么有

\begin{corollary}
    如果$A_1, A_2, \cdots A_n$ 是样本空间的一个划分,且都有正概率,则对任意正概率的事件$B$有
    \[
        P(A_i|B)=\frac{P(A_i)P(B|A_i)}{P(A_1)P(B|A_1)+\cdots+P(A_n)P(B|A_n)}.%,\ i=1,\cdots,n
    \]
\end{corollary}

\begin{asidebox}
    实际上, 在统计应用中, 事件 $A_1, \cdots, A_n$ 组成事件组 $\left(A_1+\cdots+A_n=\Omega\right)$, 常称做 “假设” 或 “假说”, 而 $\mathbf{P}\left(A_i\right)$ 称做假设 $A_i$ 的\emph{先验}概率 $\left.\right)$. 条件概率 $\mathbf{P}\left(A_i \mid B\right)$ 称做假设 $A_i$ 在事件 $B$ 出现后的\emph{后验}概率. 
\end{asidebox}

\begin{example}
    假设匣中有两枚硬币: $A_1$ 是一对称的硬币, “正面” $\mathrm{Z}$ 出现的概率等于 $1 / 2$, 而 $A_2$ 是一枚不对称的硬币, “正面” $\mathrm{Z}$ 出现的概率等于 $1 / 3$. 随意选出一枚硬币并将其投掷, 结果掷出正面. 问抽到硬币为对称硬币的概率如何?
\end{example}

\begin{solution}
    建立相应的概率模型. 这里自然取集合 $\Omega=\left\{A_1 \mathrm{Z}, A_1 \mathrm{~F}, A_2 \mathrm{Z}, A_2 \mathrm{~F}\right\}$, 可以描绘选取和投掷的结局, 其中 $A_1 \mathrm{Z}$ 表示 “选中硬币” $A_1$, 结果掷出正面 $\mathrm{Z}$, 等等, 而 $\mathrm{F}$ 表示硬币掷出反面. 根据条件, 所考虑结局的概率应该是:
$$
\mathbf{P}\left(A_1\right)=\mathbf{P}\left(A_2\right)=\frac{1}{2}
$$

和
$$
\mathbf{P}\left(\mathrm{Z} \mid A_1\right)=\frac{1}{2}, \quad \mathbf{P}\left(\mathrm{Z} \mid A_2\right)=\frac{1}{3} .
$$

这些条件唯一决定各结局的概率:
$$
\mathbf{P}\left(A_1 \mathrm{Z}\right)=\frac{1}{4}, \mathbf{P}\left(A_1 \mathrm{~F}\right)=\frac{1}{4}, \mathbf{P}\left(A_2 \mathrm{Z}\right)=\frac{1}{6}, \mathbf{P}\left(A_2 \mathrm{~F}\right)=\frac{1}{3}
$$

那么, 根据贝叶斯公式, 所求的概率为
$$
\mathbf{P}\left(A_1 \mid \mathrm{Z}\right)=\frac{\mathbf{P}\left(A_1\right) \mathbf{P}\left(\mathrm{Z} \mid A_1\right)}{\mathbf{P}\left(A_1\right) \mathbf{P}\left(\mathrm{Z} \mid A_1\right)+\mathbf{P}\left(A_2\right) \mathbf{P}\left(\mathrm{Z} \mid A_2\right)}=\frac{3}{5}
$$
\end{solution}

\begin{example}
    袋子中有10个白球, 5个黑球. 现掷一枚均匀的骰子. 掷出几点就从袋中取几个球. 若已知取出的球全为白球, 求掷出3点的概率. 
\end{example}