\section{概率中的基本概念}
\subsection{随机实验}

\begin{definition}
    随机试验是指:
    \begin{itemize}
        \item 试验结果不止一个, 但能明确所有的结果;
        \item 试验前不能预知出现哪种结果. 
    \end{itemize}
\end{definition}

\begin{definition}
    随机事件: 对随机试验的结果的陈述称为\textbf{随机事件}, 简称\textbf{事件}.
\end{definition}

\begin{definition}
    样本空间: 随机试验的每个可能结果$\omega$称为一个\textbf{样本点}. 全体样本点组成的集合$\Omega$称为\textbf{样本空间}.
    $$
        \Omega:=\{\omega\}.
    $$
    每个\textbf{随机事件}都可用样本空间的某个子集表示, 通常记为$A, B, C$等.
    在随机试验中, 当事件中的一个样本点出现时, 称该事件\textbf{发生}.
\end{definition}

\subsection{事件的关系与运算}

\begin{definition}[事件的关系]
    若事件$A$发生时, 事件$B$一定发生. 则称事件$A$包含于事件$B$(或事件$B$ \textbf{包含} $A$), 记作
    $$A\subset B \ (\text{或}B\supset A)$$

    对任意事件$A$, 有$\emptyset \subset A\subset \Omega$.

    若$A\subset B$, 且$B\subset A$, 则称事件$A$与$B$ \textbf{相等}, 记作$A=B$.

\end{definition}

下面来考察事件的运算:

\begin{definition}[事件的并]

    “事件$A$、$B$至少有一个发生”
    称为事件$A$与$B$的\textbf{和}或\textbf{并}(union), 记作
    $$A\cup B \ (\text{或}A+B)$$
    也就是
    $$A\cup B=\{\omega | \omega\in A \ \text{or}\ \omega\in B\}$$
\end{definition}

\begin{remark}
    使用数学归纳法, 事件的并可以推广到多个的情形:如$n$个事件的并
    $$\bigcup_{i=1}^{n} A_i =\text{“事件$A_1, \cdots, A_n$至少有一个发生”}$$
    可数个事件的并
    $$\bigcup_{i=1}^{\infty} A_i =\text{“事件$A_1, A_2, \cdots$至少有一个发生”}$$
\end{remark}

\begin{definition}
    “事件$A$、$B$同时发生”
    称为事件$A$与$B$的\textbf{积}或\textbf{交}(intersection), 记作
    $$AB \ (\text{或}A\cap B)$$
    也就是$$A\cap B=\{\omega | \omega\in A \ \text{and}\ \omega\in B\}$$
\end{definition}

\begin{remark}
    事件的交可以推广到多个的情形:如$n$个事件的交
    $$\bigcap_{i=1}^{n} A_i =\text{“事件$A_1, \cdots, A_n$全都发生”}$$
    可数个事件的交
    $$\bigcap_{i=1}^{\infty} A_i =\text{“事件$A_1, A_2, \cdots$全都发生”}$$
\end{remark}

\begin{definition}
    “事件$A$发生, 但$B$不发生”
    称为事件$A$与$B$的\textbf{差}, 记作
    $$A-B$$
    也就是
    $$A- B=\{\omega | \omega\in A \ \text{and}\ \omega\notin B\}$$
\end{definition}

像集合那样, 我们同样可以引入事件的关系:

\begin{definition}
    若$AB=\emptyset$, 则称$A$与$B$ \textbf{互斥}(或称$A$与$B$ \textbf{不相容}), %(mutually exclusive)
    即$A$与$B$不可能同时发生.
\end{definition}

\begin{definition}
    称$\Omega-A$为事件$A$的\textbf{对立}事件(或称$A$的\textbf{补}), 记为$\overline{A}$.
    它表示“事件$A$不发生”.
\end{definition}


像集合那样, 事件具有如下的运算规律:
\begin{itemize}
    \item 交换律
          \begin{itemize}
              \item $AB=BA$, $A\cup B=B\cup A$
          \end{itemize}
    \item 结合律
          \begin{itemize}
              \item $(AB)C=A(BC)$, $(A\cup B)\cup C=A\cup(B\cup C)$
          \end{itemize}
    \item 分配律
          \begin{itemize}
              \item $A(B\cup C)=AB\cup AC$, $A(B-C)=AB-AC$
          \end{itemize}
    \item 对偶律
          \begin{itemize}
              \item $\overline{AB}=\overline{A}\cup\overline{B}$, $\overline{A\cup B}=\overline{A}\,\overline{B}$
          \end{itemize}
\end{itemize}

下面来考察一些常见的化简运算关系的等式:

\begin{proposition}
    对任意两个事件$A$和$B$, 总有$ A-B=A-AB$.
\end{proposition}

\begin{proposition}
    事件$A$、$B$ \textbf{对立}当且仅当$A$、$B$\textbf{互斥}且$A\cup B=\Omega$.
\end{proposition}
\begin{example}
    设$A,B$为两个事件, 则有
    \begin{itemize}
        \item $A\overline{B}=A-B=A-AB$;
        \item $A=AB\cup A\overline{B}$.
    \end{itemize}
\end{example}

\begin{solution}
    用事件运算的分配律:
    \begin{itemize}
        \item $A\overline{B}=A(\Omega-B)=A\Omega-AB=A-AB$;
        \item $AB\cup A\overline{B}=A(B\cup\overline{B})=A\Omega=A$.
    \end{itemize}
\end{solution}

\begin{example}
    $A$, $B$, $C$ 表示事件
    \begin{itemize}
        \item $A$发生: $A$;
        \item 仅$A$发生: $A\cap \bar{B}\cap \bar{C}$;
        \item 恰有一个发生:$A \bar B \bar C\cup \bar AB\bar C\cup \bar A\bar BC$;
        \item 至少有一个发生:$A\cup B\cup C$;
        \item 至多有一个发生:$\bar A\bar B\bar C\cup A \bar B \bar C \cup \bar AB\bar C\cup \bar A\bar BC$;
        \item 都不发生:$\bar A\bar B\bar C$;
        \item 不全部发生: $\overline{ABC}=\bar A\cup \bar B\cup \bar C$.
    \end{itemize}
\end{example}

\begin{takeaway}
{
    可以使用集合描述事件, 离散数学中学过的集合的运算将允许我们对于事件进行化简和操作.
}
\end{takeaway}