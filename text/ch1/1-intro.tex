\section{概率中的基本概念}
\subsection{随机实验}

考虑某项试验, 其结果在某一组条件下会有若干种不同的结局 (现象) $\omega_1, \cdots, \omega_N$ . 关于这些结局具体是什么并不重要, 而是把他们抽象为一组点. 我们把这些结局 $\omega_1, \cdots, \omega_N$ 称做基本事件, 而把一切结局的全体
$$
\Omega=\left\{\omega_1, \cdots, \omega_N\right\}
$$

称做基本的事件空间, 或样本空间.

\begin{example}
    对于 “掷一枚硬币”, 基本事件空间由两个点组成:
$$
\Omega=\{\mathrm{Z}, \mathrm{F}\},
$$

其中 Z 表示出现 “正面”, 而 $\mathrm{F}$ 表示出现 “反面”. (这时假设, 诸如 “硬币在棱上立着”, “硬币丢失”.$\cdots \cdots$ 的情况不会出现.) 也就是假设不出现 “正面” 就出现 “反面”.

    将一枚硬币重复掷 $n$ 次, 基本事件空间为
$$
\Omega=\left\{\omega: \omega=\left(a_1, \cdots, a_n\right)\right\}, a_i=\text { Z或 } \mathrm{F},
$$

且基本事件的总数 $N(\Omega)=2^n$.
\end{example}

除基本事件空间的概念外,现在引进重要概念事件. 事件的概念, 是建立所考察试验的各种概率模型 (“理论”) 的基础. 在试验的结果中, 试验者一般并不关心究竟出现了哪种具体的结局, 而关心出现的结局属于一切结局集合的哪个子集. 满足试验条件的一切子集 $A \subseteq \Omega$, 分为两种类型: “结局 $\omega \in A$ ” 或 “结局 $\omega \notin A$ ”. 我们称这样的子集 $A$ 为事件.

\begin{example}
    将一枚硬币重复掷三次, 一切可能结局的空间 $\Omega$, 由 8 个点构成:
$$
\Omega=\{000,001,010,011,100,101,110,111\}
$$

其中 0 和 1 分别表示郑出 “正面” 和 “反面”. 如果由 “一组条件” 可以记录 (确定、测量等) 所有 3 次郑硬币的结果, 则例如
$$
A=\{000,001,010,100\}
$$

就是事件: “将一枚硬币重复掷三次” 正面至少出现两次. 假如由 “一组条件” 只能确定第一次郑出的结果, 则 $A$ 已经不能称为事件, 因为关于 “试验的具体结局 $\omega$ 是否属于 $A "$, 既不能肯定也不能否定.
\end{example}

在随机试验中,当事件中的一个样本点出现时,称该事件发生.

\subsection{事件的关系与运算}

\begin{definition}[事件的关系]
    若事件$A$发生时, 事件$B$一定发生. 则称事件$A$包含于事件$B$(或事件$B$ \textbf{包含} $A$), 记作
    $$A\subset B \ (\text{或}B\supset A)$$

    对任意事件$A$, 有$\emptyset \subset A\subset \Omega$.

    若$A\subset B$, 且$B\subset A$, 则称事件$A$与$B$ \textbf{相等}, 记作$A=B$.

\end{definition}

下面来考察事件的运算:

\begin{definition}[事件的并]

    “事件$A$、$B$至少有一个发生”
    称为事件$A$与$B$的\textbf{和}或\textbf{并}(union), 记作
    $$A\cup B \ (\text{或}A+B)$$
    也就是
    $$A\cup B=\{\omega | \omega\in A \ \text{or}\ \omega\in B\}$$
\end{definition}

\begin{remark}
    使用数学归纳法, 事件的并可以推广到多个的情形:如$n$个事件的并
    $$\bigcup_{i=1}^{n} A_i =\text{“事件$A_1, \cdots, A_n$至少有一个发生”}$$
    可数个事件的并
    $$\bigcup_{i=1}^{\infty} A_i =\text{“事件$A_1, A_2, \cdots$至少有一个发生”}$$
\end{remark}

\begin{definition}
    “事件$A$、$B$同时发生”
    称为事件$A$与$B$的\textbf{积}或\textbf{交}(intersection), 记作
    $$AB \ (\text{或}A\cap B)$$
    也就是$$A\cap B=\{\omega | \omega\in A \ \text{and}\ \omega\in B\}$$
\end{definition}

\begin{remark}
    事件的交可以推广到多个的情形:如$n$个事件的交
    $$\bigcap_{i=1}^{n} A_i =\text{“事件$A_1, \cdots, A_n$全都发生”}$$
    可数个事件的交
    $$\bigcap_{i=1}^{\infty} A_i =\text{“事件$A_1, A_2, \cdots$全都发生”}$$
\end{remark}

\begin{definition}
    “事件$A$发生, 但$B$不发生”
    称为事件$A$与$B$的\textbf{差}, 记作
    $$A-B$$
    也就是
    $$A- B=\{\omega | \omega\in A \ \text{and}\ \omega\notin B\}$$
\end{definition}

像集合那样, 我们同样可以引入事件的关系:

\begin{definition}
    若$AB=\emptyset$, 则称$A$与$B$ \textbf{互斥}(或称$A$与$B$ \textbf{不相容}), %(mutually exclusive)
    即$A$与$B$不可能同时发生.
\end{definition}

\begin{definition}
    称$\Omega-A$为事件$A$的\textbf{对立}事件(或称$A$的\textbf{补}), 记为$\overline{A}$.
    它表示“事件$A$不发生”.
\end{definition}


像集合那样, 事件具有如下的运算规律:
\begin{itemize}
    \item 交换律
          \begin{itemize}
              \item $AB=BA$, $A\cup B=B\cup A$
          \end{itemize}
    \item 结合律
          \begin{itemize}
              \item $(AB)C=A(BC)$, $(A\cup B)\cup C=A\cup(B\cup C)$
          \end{itemize}
    \item 分配律
          \begin{itemize}
              \item $A(B\cup C)=AB\cup AC$, $A(B-C)=AB-AC$
          \end{itemize}
    \item 对偶律
          \begin{itemize}
              \item $\overline{AB}=\overline{A}\cup\overline{B}$, $\overline{A\cup B}=\overline{A}\,\overline{B}$
          \end{itemize}
\end{itemize}

下面来考察一些常见的化简运算关系的等式:

\begin{proposition}
    对任意两个事件$A$和$B$, 总有$ A-B=A-AB$.
\end{proposition}

\begin{proposition}
    事件$A$、$B$ \textbf{对立}当且仅当$A$、$B$\textbf{互斥}且$A\cup B=\Omega$.
\end{proposition}
\begin{example}
    设$A,B$为两个事件, 则有
    \begin{itemize}
        \item $A\overline{B}=A-B=A-AB$;
        \item $A=AB\cup A\overline{B}$.
    \end{itemize}
\end{example}

\begin{solution}
    用事件运算的分配律:
    \begin{itemize}
        \item $A\overline{B}=A(\Omega-B)=A\Omega-AB=A-AB$;
        \item $AB\cup A\overline{B}=A(B\cup\overline{B})=A\Omega=A$.
    \end{itemize}
\end{solution}

\begin{example}
    $A$, $B$, $C$ 表示事件
    \begin{itemize}
        \item $A$发生: $A$;
        \item 仅$A$发生: $A\cap \bar{B}\cap \bar{C}$;
        \item 恰有一个发生:$A \bar B \bar C\cup \bar AB\bar C\cup \bar A\bar BC$;
        \item 至少有一个发生:$A\cup B\cup C$;
        \item 至多有一个发生:$\bar A\bar B\bar C\cup A \bar B \bar C \cup \bar AB\bar C\cup \bar A\bar BC$;
        \item 都不发生:$\bar A\bar B\bar C$;
        \item 不全部发生: $\overline{ABC}=\bar A\cup \bar B\cup \bar C$.
    \end{itemize}
\end{example}

\begin{takeaway}
{
    可以使用集合描述事件, 离散数学中学过的集合的运算将允许我们对于事件进行化简和操作.
}
\end{takeaway}