\section{事件的独立性}

\begin{definition}
    若两事件$A$、$B$满足
    \begin{align*}
        P(AB)= P(A) P(B),
    \end{align*}
    则称\textbf{事件$A$、$B$相互独立}.%independent
\end{definition}

实际意义:若$P(B)>0$,则上式等价于
\begin{align*}
    P(A|B)= P(A),
\end{align*}
即\textbf{事件$A$的概率不受事件$B$发生与否的影响}. 也就是事件$B$没有给我们任何的信息.

\begin{remark}
    “两个事件互斥”和“两个事件相互独立”是不同的概念:
    \begin{itemize}
        \item 互斥 $\Rightarrow$ $P(A\cup B)=P(A)+P(B)$;
        \item 独立 $\Rightarrow$ $P(AB)=P(A)P(B)$.
    \end{itemize}
    但两者也有关系:如果$P(A)>0$且$P(B)>0$,则两者不可能既是互斥的又是独立的.
\end{remark}

我们接下来看多个事件的独立性:

\begin{definition}
    称$n(n\ge 2)$个事件$A_1, A_2, \cdots, A_n$相互独立,如果对任意一组指标
    \begin{align*}
        1\le i_1<i_2< \cdots <i_k\le n\quad (k\ge 2)
    \end{align*}
    都有
    \begin{align*}
        P(A_{i_1}A_{i_2}\cdots A_{i_k})=P(A_{i_1})P(A_{i_2})\cdots  P(A_{i_k}).
    \end{align*}
\end{definition}

发现若$A$与$B$相互独立,且$B$与$C$相互独立,则$A$与$C$ \textbf{未必}相互独立.
\begin{example}
    从全体有两个孩子的家庭中随机选择一个家庭,并考虑下面三个事件:
    \begin{itemize}
        \item $A$为“第一个孩子是男孩”,
        \item $B$为“两个孩子不同性别”,
        \item $C$为“第一个孩子是女孩”.
    \end{itemize}
    容易验证$A$与$B$相互独立,$B$与$C$相互独立,但是$A$与$C$ \textbf{不独立}.

    同样, 三个两两独立的也不一定都独立.

    伯恩斯坦四面体问题:一个正四面体有三面各涂上红,白,黑三种颜色。第四面同时涂上三种颜色。这四个面等概率出现在底面。以A, B, C分别表示四面体底面出现红,白,黑三种颜色的事件。问A, B, C是否相互独立?

    $$
        P(A)=P(B)=P(C)=2/4=1/2,
    $$
    $$
        P(AB)=P(BC)=P(AC)=1/4,
    $$
    $$
        P(ABC)=1/4\neq P(A)P(B)P(C).
    $$
\end{example}
