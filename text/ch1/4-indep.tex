\section{事件的独立性}

\begin{definition}
    若两事件$A$、$B$满足
    \begin{align*}
        P(AB)= P(A) P(B),
    \end{align*}
    则称\textbf{事件$A$、$B$相互独立}. %independent
\end{definition}

实际意义:若$P(B)>0$, 则上式等价于
\begin{align*}
    P(A|B)= P(A),
\end{align*}
即\textbf{事件$A$的概率不受事件$B$发生与否的影响}.  也就是事件$B$没有给我们任何的信息.

\begin{remark}
    ``两个事件互斥''和``两个事件相互独立''是不同的概念:
    \begin{itemize}
        \item 互斥 $\Rightarrow$ $P(A\cup B)=P(A)+P(B)$; 
        \item 独立 $\Rightarrow$ $P(AB)=P(A)P(B)$. 
    \end{itemize}
    但两者也有关系:如果$P(A)>0$且$P(B)>0$, 则两者不可能既是互斥的又是独立的. 
\end{remark}

我们接下来看多个事件的独立性:

\begin{definition}
    称$n(n\ge 2)$个事件$A_1, A_2, \cdots, A_n$相互独立, 如果对任意一组指标
    \begin{align*}
        1\le i_1<i_2< \cdots <i_k\le n\quad (k\ge 2)
    \end{align*}
    都有
    \begin{align*}
        P(A_{i_1}A_{i_2}\cdots A_{i_k})=P(A_{i_1})P(A_{i_2})\cdots  P(A_{i_k}).
    \end{align*}
\end{definition}

发现若$A$与$B$相互独立, 且$B$与$C$相互独立, 则$A$与$C$ \textbf{未必}相互独立. 
\begin{example}
    从全体有两个孩子的家庭中随机选择一个家庭, 并考虑下面三个事件:
    \begin{itemize}
        \item $A$为``第一个孩子是男孩'', 
        \item $B$为``两个孩子不同性别'', 
        \item $C$为``第一个孩子是女孩''. 
    \end{itemize}
    容易验证$A$与$B$相互独立, $B$与$C$相互独立, 但是$A$与$C$ \textbf{不独立}. 

    同样, 三个两两独立的也不一定都独立.

    伯恩斯坦四面体问题:一个正四面体有三面各涂上红, 白, 黑三种颜色. 第四面同时涂上三种颜色. 这四个面等概率出现在底面. 以A, B, C分别表示四面体底面出现红, 白, 黑三种颜色的事件. 问A, B, C是否相互独立?

    $$
        P(A)=P(B)=P(C)=2/4=1/2,
    $$
    $$
        P(AB)=P(BC)=P(AC)=1/4,
    $$
    $$
        P(ABC)=1/4\neq P(A)P(B)P(C).
    $$
\end{example}


\subsection{多知道一点: 朴素的Bayes分类器}
我们希望对事情做分类. 

我们有 $n$ 个训练示例的集合, 每个对象具有一系列特征: $\left(X_1, X_2, \cdots, X_m\right)$ , 对于每个 $X_i$ 都在一组特定区域取值. 
每个 $D_i$ 由表达其特征的一组值(方便起见写作向量形式)
$$
x_i=\left(x_{i 1}, x_{i 2}, \cdots, x_{i m}\right)
$$
对象可能的分类集合 $C=\left\{C_1, C_2, \cdots, C_t\right\} . C\left(D_i\right)$ 是 $D_i$ 的分类. 
形式化的说, 我们就有了下面的一组集合供我们训练. 
$$\left\{\left(D_1, C\left(D_1\right)\right),\left(D_2, C\left(D_2\right), \cdots,\left(D_n, C\left(D_n\right)\right)\right\}\right.$$

假设我们有一个足够大的训练集, 然后, 对于每个向量\( y=\left(y_1, \cdots, y_m\right) \)和每一个$c_j$, 我们完全可以使用训练集计算一个这样的经验条件概率: 具有特征向量$y$的对象被分类为$C_j$.

根据条件概率的定义, 我们知道$p_{y,j}$的计算公式
$$
P_{y, j}=\frac{P\left\{\left(\forall i, x_i=y_i\right) \wedge c\left(D_i\right)=c_i\right\}}{P\left\{\forall i, x_i=y_i\right\}}
$$

当一个具有特征$x^*$的对象过来的时候, 我们计算$P(c(D^*)=c_j | x^* =(x_1^*,\cdots, x_n^*))$的估计. 最后, 我们返回一个向量$(p_{x^*, 1},p_{x^*, 2},\cdots, p_{x^*, n})$,表示它被划分到各个类的概率大小.

这种方法的难处在于我们需要大量的准确的样本. 即使我们的特征只能取得0和1两个值, 如果有$m$个特征, 我们也要获得$2^m$个特征值与之对应. 

如果我们的这些特征相互独立, 我们就可以加快这个进程: 
$$
\begin{aligned}
P\left(c\left(D^*\right)=c_j \mid x^*\right) & =\frac{P\left(x^* \mid c\left(D^*\right)=c_j\right) \cdot P\left(c\left(D^*\right)=c_j\right)}{P\left(x^*\right)} \\
& =\frac{\prod_{k=1}^m P\left(x_k^*=x_i \mid c\left(D^*\right)=c_j\right) \cdot P\left(c\left(D^*\right)=c_j\right)}{P\left(x^*\right)} .
\end{aligned}
$$
其中$x_k^*$是$x^*$的第$k$个分量. 由于每个特征的可能数量一定, 在$m$个特征的时候, 我们只需要学习$O(m|C|)$的概率估计. 

训练过程很简单: 
\begin{itemize}
    \item 对于每一类$c_j$, 看一看被分类为$c_j$与总共的占比为多少, 并用此计算$$
    \hat{P}\left(c\left(D^*\right)=c_j\right)=\frac{\left|\left\{i \mid c\left(D_i\right)=c_j\right\}\right|}{|D|}
    $$这里$\hat P$表示我们算的是经验概率.
    \item 对于每一个特征$X_k$和对应的特征值$x_k$, 我们注意这个特征值$x_k$被分到$c_j$那一类的占比. 也就是$$
    \hat{P}\left(x_k^*=x_k \mid c\left(D^*\right)=c_j\right)=\frac{\left|\left\{i: x_k^i=x_k, c\left(D_i\right)=c_j\right\}\right|}{\left\{i \mid c\left(D_i\right)=c_j\right\} \mid} .
    $$
\end{itemize}

一旦我们训练好了分类器, 当一个具有特征向量$x^*=(x_1^*, \cdots, x_m^*)$新的对象$D_i^*$来了的时候, 对于每一个$c_j$, 我们就可以计算$$
\left(\prod_{k=1}^m \hat{P}\left(x_k^*=x_k \mid c\left(D^*\right)=c_j\right)\right) \hat{P}\left(c\left(D^*\right)=c_j\right)
$$
并且取最大值, 得到它最可能在的分类. 

总结一下, 我们的算法如\cref{algo:Bayes-class}所示. 

\begin{algorithm}
    \caption{朴素Bayes分类器算法}
    \label{algo:Bayes-class}
    \KwData{所有分类的集合$C$, 特征以及对应的特征值$F_1, F_2, \cdots, F_m$, 用于分类项目的训练集$D$}

    训练阶段
    \begin{itemize}
        \item [1] 对于每一类$c\in C$, 特征$k=1,2,\cdots, m$, 以及特征值$x_k\in F_k$, 计算$$
        \hat{P}\left(x_k^*=x_k \mid c\left(D^*\right)=c\right)=\frac{\left|\left\{i: x_k^i=x_k, c\left(D_i\right)=c\right\}\right|}{\left\{i : c\left(D_i\right)=c\right\} \mid} .
        $$
        \item [2] 对于每一类$c\in C$, 计算$$
        \hat{P}\left(c\left(D^*\right)=c\right)=\frac{\left|\left\{i : c\left(D_i\right)=c\right\}\right|}{|D|} .
        $$
    \end{itemize}

    给具有特征向量$x^*=x=\left(x_1, \ldots, x_m\right)$的新的对象$D^*$归一个类: 
    \begin{itemize}
        \item [1] 最有可能的分类: $$c\left(D^*\right)=\arg \max _{c_j \in C}\left(\prod_{k=1}^m \hat{P}\left(x_k^*=x_k \mid c\left(D^*\right)=c_j\right)\right) \hat{P}\left(c\left(D^*\right)=c_j\right) .$$
        \item [2] 计算可能的分类分布:$$\hat{P}\left(c\left(D^*\right)=c_j\right)=\frac{\left(\prod_{k=1}^m \hat{P}\left(x_k^*=x_k \mid c\left(D^*\right)=c_j\right)\right) \hat{P}\left(c\left(D^*\right)=c_j\right)}{\hat{P}\left(x^*=x\right)}$$
    \end{itemize}
\end{algorithm}
