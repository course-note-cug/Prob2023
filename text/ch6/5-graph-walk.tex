\section{Markov链与无向图中的随机游走}

\subsection{2-SAT和3-SAT问题}

在离散数学课中, 我们学习了布尔表达式. 并且我们知道每一个命题都可以化为合取范式(或者析取范式). 这些范式中的每一个组成部分(我们先叫做语句)是由一些布尔变量的值或者它的否定的析取(合取)组成的. 接着, 把这些语句合取(析取), 就得到了对应的范式. 

给定一个范式, 我们自然想要求它的可满足性. 但是, 这个问题的复杂度已经被证明是 NP-hard 的. 如果我们发现每个语句中有$k$个布尔表达式, 并且要寻求它的可满足性, 我们把这个问题叫做$k$-SAT问题. 

比如, 2-SAT说的就是每个语句中只有2个布尔变量. 比如
$$\left(x_1 \vee \overline{x_2}\right) \wedge\left(\overline{x_1} \vee \overline{x_3}\right) \wedge\left(x_1 \vee x_2\right) \wedge\left(x_4 \vee \overline{x_3}\right) \wedge\left(x_4 \vee \overline{x_1}\right).$$

要寻找一个可满足的表达, 一个自然的想法就是: 我们先随便填上一些随机的赋值, 然后寻找公式中没有被满足的部分, 然后对这个没有满足的地方进行更改, 使得它满足这个赋值. 对于2-SAT问题, 我们就有两种改变我们的布尔变量的方式. 具体见如下的\cref{algo:2sat-random}. 在这个算法中, $n$表示变量的个数, $m$是一个整数参数, 表示了算法以正确结果终止的概率. 

\begin{algorithm}
    \caption{随机的2-SAT算法}
    \label{algo:2sat-random}

    \begin{itemize}
        \item [1.] 把布尔变量随机赋值
        \item [2.] 最多重复$2mn^2$次, 如果下面的条件都满足, 结束这一步: 
        \begin{itemize}
            \item [a)]随便选择一个不满足的语句
            \item [b)]在这个不满足的语句中, 随机选取一个变量, 改变它的真值. 
        \end{itemize}
        \item [3.] 如果找到了一个可满足的式子, 那么返回. 
        \item [4.] 如果没有满足, 报告这个公式是不可满足的. 
    \end{itemize}


\end{algorithm}

比如, 在上面的例子中, 如果我们一开始把所有的变量赋予假(F), 那么$\left(x_1 \vee x_2\right)$就不会被满足. 于是算法选择这一条语句, 随机选择一个变量(好比说$x_1$)然后设置$x_1$为真. 这之后, $\left(x_4 \vee \overline{x_1}\right)$就又没有满足, 所以算法会把那个语句中的某一个变量也改变真值, 等等... 

如果算法最后真正找到了一个可满足的表达式, 那么这个答案一定是正确的. 但如果我们的算法没有成功地找到一个可满足的赋值, 这个语句就一定不可满足吗? 这个问题需要仔细讨论, 我们待会再说. 我们现在先假定这个公式确实可以满足, 并且我们的算法想运行多久都可以. 

我们主要对于第二步的循环感兴趣. 为方便起见, 我们把每次算法更改一个变量的真值的这个过程叫做``一步''. 因为2-SAT有$O(n^2)$个语句, 也就是每一步可能要花费$O(n^2)$的时间. 虽然我们可以用更快速的方法实现, 但是这里我们先不做讨论. 令$S$表示这$n$个变量中可满足的赋值之一, 令$A_i$表示在算法执行了第$i$步之后变量的赋值, 令$X_i$表示当前的$A_i$与$S$中变量的赋值相同的个数. 当$X_i=n$的时候, 我们的算法就会报告一个可满足的赋值并终止. 实际上, $X_i$有可能小于$n$, 因为它找到了另一个可满足的表达式. 但是为了方便我们分析最坏情况, 我们还是要求当$X_i=n$的时候算法才终止. 从$X_i<n$开始, 我们考虑$X_i$的值如何随着时间变化, 尤其注意它耗费了多长时间到达$n$. 

首先, 注意到如果$X_i=0$, 那么我们无论如何更改, 我们都有$X_{i+1}=1$. 也就是 $$P\left(X_{i+1}=1 \mid X_i=0\right)=1.$$

我们假设有在前$n-1$步中, 有$1 \leq X_i \leq n-1$. 每一步中, 我们选取一个不被满足的语句. 由于$S$的赋值可以满足整个语句, $A_i$和$S$对于当前不满足的这个语句而言, 至少有一个是不同的. 因为语句的变量不超过两个, 所以我们增加匹配数的概率至少是1/2 -- 这是因为如果更改的恰好是