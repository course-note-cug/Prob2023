\section{随机过程简介}

通常来讲, 随机过程研究的是无穷多个随机变量, 并将他们当做一个整体来探讨. 更正式地说, 随机过程就是
\begin{definition}
    给定无穷集$T\subset (-\infty, +\infty)$. 如果对于每一个$t$, 对应一个随机变量$X_t$, 则称随机变量簇$\{X_t,t\in T\}$是随机过程. 有时候简称为过程. (或写作$X(t)$)
\end{definition}

\begin{example}
    用$X_t$表示某邮箱每天从0时刻到$t$时刻所接到的信件的个数. 那么$\{X(t), t\in [0, +\infty)\}$是随机过程.
\end{example}

\begin{example}
    水中的花粉在做布朗运动. 这时候, 如果记$X_t$为花粉在时刻$t$所在的位置的坐标, 那么$\{X_t,t\in [0, +\infty)\}$便是一个随机过程. 
\end{example}

一般我们用$E$表示这些$X_t$所取的值组成的集合, $E$叫做状态空间. 如果$X_t = x$, 我们称随机过程$\{X_t, t\in T\}$在时刻$t$处于状态$X$. 当$t$是可列无穷集的时候, $\{X_t, t\in T\}$称为离散时间的随机过程. 比如上面的例子$T=\{0,1,2,\}$ 当$T$是一个区间(包括无穷区间)的时候, $\{X_t, t\in T\}$叫做连续时间的随机过程. 比较常见的是$T=[0,+\infty)$.

给定$T$中的$n$个数$t_1, t_2, \cdots, t_n$, 记$(X_{t_1}, X_{t_2}, \cdots, X_{t_n})$的分布函数为$F_{t_1, t_2, \cdots, t_n}(x_1, x_2, \cdots, x_n)$. 这种分布函数的全体$\{F_{t_1, t_2, \cdots, t_n}(x_1, x_2, \cdots, x_n), n\geq 1, t_1,t_2,\cdots, t_n\in T\}$称为有限维分布簇. 这个分布簇刻画了随机过程的概率的特性. 

随机过程其实可以看做时间$t$和可能的结果$\omega$的二元函数. 如果我们固定$t$, 那么我们就得到了在某一个特定的时间下可能的分布; 相反, 如果我们固定$\omega$, 那么$X_t(\omega)$就会是$t$的函数. 有时候简写为$X(t)$. 这个函数就是随机过程的一个``实现''. 我们在一个时间段上面对于随机过程进行观察, 得到的记录就是这样的一个随机过程的实现的一段. 

接下来我们来定义两个随机过程的等价性. 

\begin{definition}
    设两个过程$\{X_t, t\in T\}, \{Y_t, t\in T\}$是随机等价的, 若$P(X_t = Y_t)=1$. 
\end{definition}