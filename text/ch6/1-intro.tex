\section{数理统计的基本概念}

\paragraph{1. 引入: 什么是数理统计(摘自\cite{probpku})}

Freedman在他的《统计学》中生动的描述了统计学研究什么: 
\begin{quote}
    (数理)统计学是什么? 统计学\footnote{``统计学'' 与 ``数理统计学''实质是同一个学科. 通常, 在统计研究者强调数学方法时, 将 ``统计学'' 前面加上一个``数理''的形容词.}是对令人图惑费解的问题作出数学设想的艺术. 应该怎样设计实验来演定新药的疗效? 什么东西引起父母与孩子之间的相像, 并且那种力量有多强? 通货澎胀率如何测定? 失业率呢? 它们怎样联系起来? 赌场为什么在轮盘賭上得益? 盖洛普民意测验怎么能够使用仅仅几千人的样本预测选举结果?
\end{quote}
可以看出, 统计学研究的对象十分广泛. 与先前学过的概率论相比, 统计学更加关心由我们的``数据''是如何得到``模型''的. 而非概率论关心的由``模型''得到``数据''. 

首先我们的研究对象是数据. 什么是数据呢? 广义地讲, 数据培贸我们在实际工作中的记录. 例如某工厂为考查灯泡的使用寿命, 随机地抽取了 18 台产品做试验, 测得寿命数据 (单位: h) 如下:
$$
\begin{aligned}
& 17,29,50,68,100,130,140,270,280,340, \\
& 410,450,520,620,190,210,800,1100 .
\end{aligned}
$$

这 18 个寿命数据就是我们的研究对象. 若不对这些数据进行合理的抽象, 就不可能对这些数据进行深层次的分析, 进而从中获得更多的信息. 在数据处理时我们通常用 $x$ 表示数据, 这里的 $x$ 既可以是一个数,也可以是一个向量或其他的量. 当明确表示向量时, 我们用 $\vec x$ 表示之. 这时数据的主要形式是 $\vec x=\left(x_1, \cdots, x_n\right)$.

在实际问题中, 有时候单一个字母是不够用于表达数据的. 例如, 在连续 10 天的气象记录中, 得到 $m_1, \cdots, m_{10}, M_1, \cdots, M_{10}$, 其中 $m_i(i=1, \cdots, 10)$ 是每天的最低气温, $M_i(i=1, \cdots, 10)$ 是每天的最高气温. 此时的数据为 $\vec x=\left\{\left(m_i, M_i\right), i=1, \cdots, 10\right\}$. 在学习统计学的时候, 用 $\vec x=\left(x_1, \cdots\right.$, $\left.x_n\right)$ 表示数据是最方便并且能够抓住数据本质的一种方法. 本节开头引入的寿命数据可表达成 $\vec x=\left(x_1, \cdots, x_{18}\right)$ 或 $\vec x=\left(x_1, \cdots, x_n\right), n=18$.

引入数据的概念以后, 我们要记住统计工作的核心任务是对数据进行分析, 进而对所考查的问题作出推断. 在寿命数据的问题中我们的任务是考察该厂生产的电子产品的使用寿命. 我们收集到的 18 台电子产品的寿命数据是该厂生产的一部分产品的数据. 此处特别强调, 我们的目的是要了解该厂生产的电子产品的使用寿命, 而不是这 18 台产品的使用寿命. 这 18 台产品的使用寿命是已经明摆着的数据, 不必再进行细究. 为了研究产品的使用寿命, 我们必须弄明白, 什么是工厂生产的电子产品的使用寿命, 而且还要弄清楚这 18 台产品的寿命与该厂生产的电子产品的使用寿命之间的联系. 最后我们要以这 18 台产品的位用寿命为依据,对该厂生产的电子产品的使用寿命作某些推断. 

根据经验知, 一个工厂生产的产品的使用寿命是带随机性的. 因此, 我们把一个工厂所生产的电子产品的使用寿命 $X$ 看成一个随机变量. 作这样的拙象以后, 可以把人们思想中直观的概念精确化成为一个数学概念. 若没有这种抽象, 就不可能对工厂生产的电子产品的使用寿命进行精确地研究. 什么是我们所需要的信息? 随机变量的某些持征是我们最关心的. 例如, $X$ 的期望 $\mathrm{E}(X), \mathrm{E}(X)$ 越大, 说明产品的使用寿命越长. $\mathrm{E}(X)$ 的大小说明该厂生产的产品质量. 除了 $\mathrm{E}(X), X$ 的标准差 $\sigma(X)$ $=\sqrt{\operatorname{var}(X)}$ 也是一个很重要的指标. 当然 $X$ 的分布体现了工厂所生产产品的使用寿命的全部信息, 因此若我们要了解工厂生产的电子产品的使用寿命, 只需了解随机变量 $X$ 的分布. 而数据又是什么? 它与随机变量 $X$ 的关系是什么? 从数据形成的过程可知, 电子产品的寿命 $x_1$ 是工厂生产的某台产品的寿命, 它是 $X$ 的一个观察值, 也可以看做是与 $X$ 同分布的随机变量 $X_1$ 的观察值. 同样 $x_n(n=2, \cdots, 18)$ 是与 $X$ 同分布的随机变量 $X_n(n=2, \cdots, 18)$ 的观察值. 这样, $x=\left(x_1, \cdots, x_{18}\right)$ 是 $\boldsymbol{X}=$ $\left(X_1, \cdots, X_{18}\right)$ 的观察值. 有经验的实际工作者一定会明白,我们收集 18 台数据的目的是为了了解工厂所生产产品的质量, 所以在采样时一定不会为某种利益去故意选择好的产品或坏的产品进行检查. 因此所选的产品一定是代表工厂产品质量的随机变量 $X$ 的观察值, 而且这 18 台产品也是相互独立地采样而得到的. 用数学的语言来描述, $X_1, \cdots$, $X_{18}$ 为相互独立且同分布的随机变量序列, 而其共同分布与 $X$ 的分布相同. 由于数据 $\boldsymbol{x}$ 与 $X$ 有这样一层关系, 我们就指望从 $\boldsymbol{x}$ 得到 $X$ 的分布信息.

\paragraph{2. 基本的概念和定义}

~

由上面的例子, 我们可以看出: 

\begin{definition}
    在数理统计中, 把研究对象的全体称为总体, 组成总体的每个元素称为个体. 总体中随机抽取若干个个体构成的集合称为总体的样本. 样本所含个体的个数称为样本容量. 
\end{definition}

其中一类很重要的问题是, 根据以往的经验, 我们知道它大概是什么分布, 但是没有办法确定它的参数. 比如, 在考察某电子设备的使用寿命的时候. 我们通过观察猜测 $X$ 服从指数分布, 即
$$
X \sim \frac{1}{\theta} \exp \left\{-\frac{x}{\theta}\right\}, \quad x>0, \theta>0 .
$$
但现在我们并不知道$\theta$是多少. 有没有办法让我们得到这个$\theta$的值呢? 这就是参数估计的概念. 