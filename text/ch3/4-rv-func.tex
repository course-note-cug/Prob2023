\section{两个随机变量的函数的分布}

我们先来看一个问题: $X,Y$是相互独立的离散型随机变量, 等概率地取$[0,3]$区间的整数. 问$X+Y=3$的概率是多少? 

解答也不难. 考虑一共有$(0, 3), (1, 2), (2, 1), (3, 0)$四种情况, 相加即可. 那么我们把这个$X+Y$看做新的随机变量, 应该如何求? 实际上我们只要枚举就好了.
$$
\begin{aligned}
P_W(w) & =P(X+Y=w) \\
& =\sum_x P(X=x)  P(Y=w-x) \\
& =\sum_x P_X(x) P_Y(w-x) . 
\end{aligned}
$$

我们现在来系统的考察对于任意的分布函数, 几个常见的两个随机变量参与运算之后的概率密度. 

\paragraph{(一) $Z=X+Y$ 的分布}

设 $(X, Y)$ 是二维连续型随机变量, 它具有概率密度 $f_{X,Y}(x, y)$.那么连续型随机变量$Z=X+Y$的概率密度是多少?

大致思路: 
    \begin{itemize}
        \item 先来求 $Z=X+Y$ 的分布函数 $F_Z(z)$
        \item $F_Z(z)=P(Z \leq z)=\iint_{x+y \leq z} f_{X,Y}(x, y) \mathrm{d} x \mathrm{~d} y$
    \end{itemize}
    $$\begin{aligned}
        F_Z(z)&=\int_{-\infty}^{\infty}\left[\int_{-\infty}^{\purple {z-y}} f({\red{x}}, y) \mathrm{d} {\red{x}}\right] \mathrm{d} y \\ 
        &\stackrel{{\red x}:=\teal{u-y}}{\stackrel{\rule{1.5cm}{0.4pt}}{\rule{1.5cm}{0.4pt}}}
        \int_{-\infty}^{\infty}\left[\int_{-\infty}^z f({\teal {u-y}}, y) \mathrm{d} u\right] \mathrm{d} y\\
        &=\int_{-\infty}^z\left[\int_{-\infty}^{\infty} f(u-y, y) \mathrm{d} y\right] \mathrm{d} u
    \end{aligned}$$

    这里的换元法实际上是使用给定的不等关系$x+y\leq z$为了消去$x$, 减少变量个数. 

    根据定义, 求导得到: 
    $f_{X+Y}(z)=\int_{-\infty}^{\infty} f(u-y, y) \mathrm{d} y=\int_{-\infty}^{\infty} f(z-y, y) \mathrm{d} y.$

我们用定理的形式总结这一事实: 
\begin{theorem}
    设 $(X, Y)$ 是二维连续型随机变量, 它具有概率密度 $f_{X,Y}(x, y)$. 则 $Z=X+Y$ 仍为连续型随机变量, 其概率密度为
    $$f_{X+Y}(z)=\int_{-\infty}^{\infty} f(z-y, y) \mathrm{d} y$$
    或
    $$
f_{X+Y}(z)=\int_{-\infty}^{\infty} f(x, z-x) \mathrm{d} x
$$
\end{theorem}

如果$X,Y$相互独立的话, 上述公式可以进一步地写作
\begin{itemize}
    \item $f_{X+Y}(z)=\int_{-\infty}^{\infty} f_X(z-y) f_Y(y) \mathrm{d} y$;
    \item $f_{X+Y}(z)=\int_{-\infty}^{\infty} f_X(x) f_Y(z-x) \mathrm{d} x$
\end{itemize}

\begin{asidebox}
    \newword{卷积}{convolution}: 刚刚常见的操作其实有一个名字: 卷积. 比如我们在求两个多项式的乘积的时候, $(\sum_{i=0}^{\infty} a_i x^i)(\sum_{j=0}^{\infty} b_j x^j).$

    我们可以用这样的公式计算: 

    $$
    \begin{aligned}
        \sum_{k=0}^{\infty}\left(\sum_{i+j=k} a_i b_j\right) x^k
        =\sum_{k=0}^{\infty}\left(\sum_{i=0}^k a_i b_{k-i}\right) x^k
    \end{aligned}
    $$

    刚刚的那个问题同样和这个问题有类似的性质, 只是求和号变为了积分号. 

\end{asidebox}

为了方便起见, 这两个公式称为 $f_X$ 和 $f_Y$ 的卷积公式, 记为 $f_X * f_Y$, 即
$$
f_X * f_Y=\int_{-\infty}^{\infty} f_X(z-y) f_Y(y) \mathrm{d} y=\int_{-\infty}^{\infty} f_X(x) f_Y(z-x) \mathrm{d} x
$$

\begin{asidebox}
    直观理解卷积: 先把纸片翻一下, 然后平移, 最后对应相乘相加. 如\cref{fig:convolution}.
\end{asidebox}

\begin{figure}
    \center
    % \usepackage[usenames,dvipsnames]{pstricks}
% \usepackage{pstricks-add}
% \usepackage{epsfig}
% \usepackage{pst-grad} % For gradients
% \usepackage{pst-plot} % For axes
% \usepackage[space]{grffile} % For spaces in paths
% \usepackage{etoolbox} % For spaces in paths
% \makeatletter % For spaces in paths
% \patchcmd\Gread@eps{\@inputcheck#1 }{\@inputcheck"#1"\relax}{}{}
% \makeatother
% 
\psscalebox{0.7 0.7} % Change this value to rescale the drawing.
{
\begin{pspicture}(0,11.045)(26.246119,21.935)
\definecolor{colour2}{rgb}{0.0,0.0,0.5019608}
\definecolor{colour3}{rgb}{0.0,0.5019608,0.5019608}
\definecolor{colour5}{rgb}{0.8,0.2,0.2}
\psline[linecolor=colour2, linewidth=0.04, fillstyle=crosshatch, hatchwidth=0.028222222, hatchangle=0.0, hatchsep=0.1411111, arrowsize=0.05291667cm 2.0,arrowlength=1.4,arrowinset=0.0]{->}(6.462786,17.445)(6.462786,20.445)
\psline[linecolor=colour2, linewidth=0.04, fillstyle=crosshatch, hatchwidth=0.028222222, hatchangle=0.0, hatchsep=0.1411111, arrowsize=0.05291667cm 2.0,arrowlength=1.4,arrowinset=0.0]{->}(6.3961196,17.445)(12.29612,17.445)
\psline[linecolor=colour2, linewidth=0.04, fillstyle=crosshatch, hatchwidth=0.028222222, hatchangle=0.0, hatchsep=0.1411111, dotsize=0.07055555cm 2.0]{-*}(7.3961196,17.445)(7.3961196,18.445)(7.3961196,18.445)
\psline[linecolor=colour2, linewidth=0.04, fillstyle=crosshatch, hatchwidth=0.028222222, hatchangle=0.0, hatchsep=0.1411111, dotsize=0.07055555cm 2.0]{-*}(8.396119,17.445)(8.396119,18.445)
\psline[linecolor=colour2, linewidth=0.04, fillstyle=crosshatch, hatchwidth=0.028222222, hatchangle=0.0, hatchsep=0.1411111, dotsize=0.07055555cm 2.0]{-*}(9.396119,17.445)(9.396119,18.445)
\psline[linecolor=colour2, linewidth=0.04, fillstyle=crosshatch, hatchwidth=0.028222222, hatchangle=0.0, hatchsep=0.1411111, dotsize=0.07055555cm 2.0]{-*}(10.396119,17.445)(10.396119,18.445)
\rput[bl](7.0961194,18.745){$1/4$}
\rput[bl](8.096119,18.745){$1/4$}
\rput[bl](9.096119,18.745){$1/4$}
\rput[bl](10.096119,18.745){$1/4$}
\rput[bl](5.1961193,20.445){\textcolor{colour2}{$p_X(x)$
}}
\rput[bl](12.396119,17.345){\textcolor{colour2}{$x$}}
\psline[linecolor=colour3, linewidth=0.04, fillstyle=crosshatch, hatchwidth=0.028222222, hatchangle=0.0, hatchsep=0.1411111, arrowsize=0.05291667cm 2.0,arrowlength=1.4,arrowinset=0.0]{->}(6.462786,12.545)(6.462786,15.545)
\psline[linecolor=colour3, linewidth=0.04, fillstyle=crosshatch, hatchwidth=0.028222222, hatchangle=0.0, hatchsep=0.1411111, arrowsize=0.05291667cm 2.0,arrowlength=1.4,arrowinset=0.0]{->}(6.4961195,12.545)(12.29612,12.545)
\psline[linecolor=colour3, linewidth=0.04, fillstyle=crosshatch, hatchwidth=0.028222222, hatchangle=0.0, hatchsep=0.1411111, dotsize=0.07055555cm 2.0]{-*}(8.396119,12.545)(8.396119,14.645)
\psline[linecolor=colour3, linewidth=0.04, fillstyle=crosshatch, hatchwidth=0.028222222, hatchangle=0.0, hatchsep=0.1411111, dotsize=0.07055555cm 2.0]{-*}(10.396119,12.545)(10.396119,13.345)
\rput[bl](8.396119,14.445){\textcolor{colour3}{$2/3$}}
\rput[bl](9.896119,13.445){\textcolor{colour3}{$1/3$}}
\rput[bl](5.162786,15.545){\textcolor{colour5}{$p_X(x)$
}}
\rput[bl](12.396119,12.445){\textcolor{colour3}{$x$}}
\rput[bl](7.296119,20.245){\textcolor{colour2}{$p_X(x)=1/4(x=1,2,3,4)$}}
\rput[bl](7.1961193,15.045){\textcolor{colour3}{$p_Y(y)=\begin{cases}1/3, y=4;  2/3, y=2; 0, ow\end{cases}$}}
\rput[bl](7.296119,17.045){1}
\rput[bl](8.29612,17.045){2}
\rput[bl](9.29612,17.045){3}
\rput[bl](10.29612,17.045){4}
\rput[bl](7.296119,12.145){\textcolor{colour3}{1}}
\rput[bl](8.29612,12.145){\textcolor{colour3}{2}}
\rput[bl](9.29612,12.145){\textcolor{colour3}{3}}
\rput[bl](10.29612,12.145){\textcolor{colour3}{4}}
\rput[bl](7.296119,21.545){$P(X+Y=3)=?$}
\psline[linecolor=colour2, linewidth=0.04, fillstyle=crosshatch, hatchwidth=0.028222222, hatchangle=0.0, hatchsep=0.1411111, arrowsize=0.05291667cm 2.0,arrowlength=1.4,arrowinset=0.0]{->}(18.89612,17.178333)(18.89612,20.178333)
\psline[linecolor=colour2, linewidth=0.04, fillstyle=crosshatch, hatchwidth=0.028222222, hatchangle=0.0, hatchsep=0.1411111, arrowsize=0.05291667cm 2.0,arrowlength=1.4,arrowinset=0.0]{->}(18.89612,17.178333)(25.79612,17.178333)
\psline[linecolor=colour2, linewidth=0.04, fillstyle=crosshatch, hatchwidth=0.028222222, hatchangle=0.0, hatchsep=0.1411111, dotsize=0.07055555cm 2.0]{-*}(20.89612,17.178333)(20.89612,18.178333)(20.89612,18.178333)
\psline[linecolor=colour2, linewidth=0.04, fillstyle=crosshatch, hatchwidth=0.028222222, hatchangle=0.0, hatchsep=0.1411111, dotsize=0.07055555cm 2.0]{-*}(21.89612,17.178333)(21.89612,18.178333)
\psline[linecolor=colour2, linewidth=0.04, fillstyle=crosshatch, hatchwidth=0.028222222, hatchangle=0.0, hatchsep=0.1411111, dotsize=0.07055555cm 2.0]{-*}(22.89612,17.178333)(22.89612,18.178333)
\psline[linecolor=colour2, linewidth=0.04, fillstyle=crosshatch, hatchwidth=0.028222222, hatchangle=0.0, hatchsep=0.1411111, dotsize=0.07055555cm 2.0]{-*}(23.89612,17.178333)(23.89612,18.178333)
\rput[bl](20.596119,18.478333){$1/4$}
\rput[bl](21.596119,18.478333){$1/4$}
\rput[bl](22.596119,18.478333){$1/4$}
\rput[bl](23.596119,18.478333){$1/4$}
\rput[bl](18.69612,20.178333){\textcolor{colour2}{$p_X(x)$
}}
\rput[bl](25.89612,17.078333){\textcolor{colour2}{$x$}}
\psline[linecolor=colour5, linewidth=0.04, fillstyle=crosshatch, hatchwidth=0.028222222, hatchangle=0.0, hatchsep=0.1411111, arrowsize=0.05291667cm 2.0,arrowlength=1.4,arrowinset=0.0]{->}(6.358333,12.544683)(6.358333,15.544683)
\psline[linecolor=colour5, linewidth=0.04, fillstyle=crosshatch, hatchwidth=0.028222222, hatchangle=0.0, hatchsep=0.1411111, arrowsize=0.05291667cm 2.0,arrowlength=1.4,arrowinset=0.0]{->}(6.3961196,12.545)(0.10416634,12.544683)
\psline[linecolor=colour5, linewidth=0.04, fillstyle=crosshatch, hatchwidth=0.028222222, hatchangle=0.0, hatchsep=0.1411111, dotsize=0.07055555cm 2.0]{-*}(4.466666,12.544683)(4.466666,14.644684)
\psline[linecolor=colour5, linewidth=0.04, fillstyle=crosshatch, hatchwidth=0.028222222, hatchangle=0.0, hatchsep=0.1411111, dotsize=0.07055555cm 2.0]{-*}(2.483333,12.544683)(2.483333,13.344684)
\rput[bl](4.466666,14.444683){\textcolor{colour5}{$2/3$}}
\rput[bl](2.2041664,13.644684){\textcolor{colour5}{$1/3$}}
\rput[bl](5.4999995,15.444683){\textcolor{colour5}{$p_X(x)$
}}
\rput[bl](0.0,12.444683){\textcolor{colour5}{$x$}}
\rput[bl](20.79612,19.978333){\textcolor{colour2}{$p_X(x)=1/4(x=1,2,3,4)$}}
\rput[bl](20.79612,16.778334){1}
\rput[bl](21.79612,16.778334){2}
\rput[bl](22.79612,16.778334){3}
\rput[bl](23.79612,16.778334){4}
\rput[bl](5.3124995,12.144684){\textcolor{colour5}{1}}
\rput[bl](4.270833,12.144684){\textcolor{colour5}{2}}
\rput[bl](3.2291663,12.144684){\textcolor{colour5}{3}}
\rput[bl](2.1874998,12.144684){\textcolor{colour5}{4}}
\psarc[linecolor=colour5, linewidth=0.04, dimen=outer, arrowsize=0.05291667cm 2.0,arrowlength=1.4,arrowinset=0.0]{->}(6.2294526,13.078333){3.3}{60.786674}{118.705956}
\psline[linecolor=red, linewidth=0.04, fillstyle=crosshatch, hatchwidth=0.028222222, hatchangle=0.0, hatchsep=0.1411111, arrowsize=0.05291667cm 2.0,arrowlength=1.4,arrowinset=0.0]{->}(23.658333,12.678017)(23.658333,15.678017)
\psline[linecolor=red, linewidth=0.04, fillstyle=crosshatch, hatchwidth=0.028222222, hatchangle=0.0, hatchsep=0.1411111, arrowsize=0.05291667cm 2.0,arrowlength=1.4,arrowinset=0.0]{->}(23.658333,12.678017)(16.470833,12.678017)
\psline[linecolor=red, linewidth=0.04, fillstyle=crosshatch, hatchwidth=0.028222222, hatchangle=0.0, hatchsep=0.1411111, dotsize=0.07055555cm 2.0]{-*}(20.833332,12.678017)(20.833332,14.778017)
\psline[linecolor=red, linewidth=0.04, fillstyle=crosshatch, hatchwidth=0.028222222, hatchangle=0.0, hatchsep=0.1411111, dotsize=0.07055555cm 2.0]{-*}(18.85,12.678017)(18.85,13.478017)
\rput[bl](20.833332,14.578017){\textcolor{red}{$2/3$}}
\rput[bl](18.570833,13.778017){\textcolor{red}{$1/3$}}
\rput[bl](23.866667,15.678017){\textcolor{red}{$p_X(x)$
}}
\rput[bl](16.366667,12.578017){\textcolor{red}{$x$}}
\rput[bl](21.679167,12.278017){\textcolor{red}{1}}
\rput[bl](20.637499,12.278017){\textcolor{red}{2}}
\rput[bl](19.595833,12.278017){\textcolor{red}{3}}
\rput[bl](18.554167,12.278017){\textcolor{red}{4}}
\psline[linecolor=black, linewidth=0.04, arrowsize=0.05291667cm 2.0,arrowlength=1.4,arrowinset=0.0]{->}(19.19612,16.045)(21.096119,16.045)(21.096119,16.045)
\rput[bl](21.162786,15.911667){3}
\rput[bl](5.4961195,11.045){(a)}
\rput[bl](20.89612,11.345){(b)}
\end{pspicture}
}


    \caption{直观理解卷积}
    \label{fig:convolution}
\end{figure}

\paragraph{(二)$Z=Y/X,Z=XY$的分布}

设 $(X, Y)$ 是二维连续型随机变量, 它具有概率密度 $f_{X,Y}(x, y)$.连续型随机变量$Z=Y/X$的概率密度是多少?

我们还是还是首先设出$F_{Y/X}(z)=P(Y/X\leq z)$. 但是这里由于函数的不连续性, 需要分两种情况: $x<0, x>0$分别考虑:

$$
\begin{aligned}
F_{Y / X}(z) & =P(Y / X \leq z)=\iint_{\stackrel{y / x \leq z}{x<0}} f(x, y) \mathrm{d} y \mathrm{~d} x+\iint_{\stackrel{y / x \leq z}{x>0}} f(x, y) \mathrm{d} y \mathrm{~d} x \\
& =\int_{-\infty}^0\left[\int_{z x}^{\infty} f(x, y) \mathrm{d} y\right] \mathrm{d} x+\int_0^{\infty}\left[\int_{-\infty}^{z x} f(x, y) \mathrm{d} y\right] \mathrm{d} x \\
& \varsub{y:=xu}{1cm} \int_{-\infty}^0\left[\int_z^{-\infty} x f(x, x u) \mathrm{d} u\right] \mathrm{d} x+\int_0^{\infty}\left[\int_{-\infty}^z x f(x, x u) \mathrm{d} u\right] \mathrm{d} x \\
& =\int_{-\infty}^0\left[\int_{-\infty}^z(-x) f(x, x u) \mathrm{d} u\right] \mathrm{d} x+\int_0^{\infty}\left[\int_{-\infty}^z x f(x, x u) \mathrm{d} u\right] \mathrm{d} x \\
& =\int_{-\infty}^z\left[\int_{-\infty}^{+\infty}|x| f(x, x u) \mathrm{d} u\right] \mathrm{d} x
\end{aligned}
$$

遵循同样的模式, 同样可以求出: $Z=XY$的概率分布. 
$$
\begin{aligned} & F_{XY}(z)=P(XY\leq z)\\
 & =\iint_{\substack{xy\leq z\\
 x<0}}f(x,y)\dd y\dd x+\iint_{\substack{xy\leq z\\
 x>0}}f(x,y)\dd y\dd x\\
 & =\int_{-\infty}^{0}\left(\int_{z/x}^{+\infty}f(x,y)\dd y\right)\dd x+\int_{0}^{+\infty}\left(\int_{-\infty}^{z/x}f(x,y)\dd y\right)\dd x\\
 & \varsub{y:=u/x}{1.5cm}\int_{-\infty}^{0}\left(\int_{z/x}^{+\infty}f\left(x,\frac{u}{x}\right)d\left(\frac{u}{x}\right)\right)\dd x+\int_{0}^{+\infty}\left(\int_{-\infty}^{z/x}f\left(x,\left(\frac{u}{x}\right)\dd x\right)\right)\\
 & =\int_{-\infty}^{0}\left(\left(\frac{1}{x}\right)\int_{z}^{-\infty}f\left(x,\frac{u}{x}\right)\dd u\right)\dd x+\int_{0}^{+\infty}\left(\frac{1}{x}\int_{-\infty}^{z}f\left(x,\frac{u}{x}\right)\dd u\right)\dd x\\
 & =\int_{0}^{z}\left(\int_{-\infty}^{\infty}\frac{1}{|x|}f\left(x,\frac{u}{x}\right)\dd u\right)\dd x
\end{aligned}
$$

\paragraph{(三) $M=\min \{X, Y\}$的分布}

$X, Y$ 是两个\emph{相互独立}的随机变量, 它们的分布函数分别为 $F_X(x)$ 和$F_Y(y)$.求 $M=\min \{X, Y\}$的分布函数.


   $$
\begin{aligned}
F_{\min }(z) & =P(N \leq z\}=1-P\{N>z) \\
& =1-P(X>z, Y>z\}=1-P(X>z) \cdot P\{Y>z)
\end{aligned}
$$

也就是
$$
F_{\min }(z)=1-\left[1-F_X(z)\right]\left[1-F_Y(z)\right] .
$$

上述的三个情况都可以推广到$n$维的情形.