\section{相互独立的随机变量}

在离散的情形, 我们定义两个事件是独立的当且仅当$P(AB)=P(A)P(B)$. 那么对于随机变量, 我们也给出同样的定义: 

\begin{definition}[相互独立的随机变量]
    设 $\red{F_{X,Y}(x, y)}$ 及 $\teal{F_X(x), F_Y(y)}$ 分别是二维随机变量 $(X, Y)$ 的分布函数及边缘分布函数. 若对于所有 $x, y$ 有
    $$
\begin{gathered}
    P\{X \leqslant x, Y \leqslant y\}=P\{X \leqslant x\} P\{Y \leqslant y\}, \\
\red{F_{X,Y}(x, y)}=\teal{F_X(x) F_Y(y)},
\end{gathered}
$$
则称随机变量 $X$ 和 $Y$ 是相互独立的.
\end{definition}

实际上, 刚刚的定义给了我们一点提示: 如果$(X, Y)$ 是连续型随机变量,$f_{X,Y}(x, y), f_X(x), f_Y(y)$是概率密度以及边缘概率密度, 那么: 
\begin{itemize}
    \item $X$ 和 $Y$ 相互独立 $\iff$ $f_{X,Y}(x, y)=f_X(x) f_Y(y)$ 几乎处处成立. 
    \item (除了在平面上面积为0的集合)
\end{itemize}

也就是说, 如果 $(X, Y)$ 是离散型随机变量:
    \begin{itemize}
        \item $P\left\{X=x_i, Y=y_j\right\}=P\left\{X=x_i\right\} P\left\{Y=y_j\right\}$.
    \end{itemize}

\subsection{推广到$n$维}

对于多个事件的独立性, 我们知道任意的一个事件的子集都必须满足$P(A_1\cdots A_k)=P(A_1)\cdots P(A_n)$. 高维的情形也是类似的: 

\begin{definition*}

    设$(X_1,X_2,\dots,X_n)$的分布函数$F(x_1,x_2,\dots,x_n)$为已知,则$(X_1,X_2,\dots,X_n)$的$k(1 \leq k < n )$维边缘分布函数就随之确定,例如$(X_1,X_2,\dots,X_n)$关于$X_1$、关于$(X_1,X_2)$的边缘分布函数分别为
    \[F_{X_1}(x_1) = F(x_1,\infty,\infty,\dots,\infty),\]
    \[F_{X_1,X_2}(x_1,x_2) = F(x_1,x_2,\infty,\infty,\dots,\infty)\].
又若$f(x_1,x_2,\dots,x_n)$是$(X_1,X_2,\dots,X_n)$的概率密度,则$(X_1,X_2,\dots,X_n)$关于$X_1$、关于$(X_1,X_2)$的边缘概率密度分别为
    \[f_{X_1}(x_1) = \int_{-\infty}^{\infty}\int_{-\infty}^{\infty}
    \dots \int_{-\infty}^{\infty}f(x_1,x_2,\dots,x_n)d{x_2}d{x_3}\dots d{x_n}\].

    \[f_{X_1,X_2}\left(x_1,x_2\right) = \int_{-\infty}^{\infty} \int_{-\infty}^{\infty} \dots \int_{-\infty}^{\infty}f\left(x_1,x_2,\dots,x_n\right)d{x_3}d{x_4} \dots {x_n},\]
    \quad 若对于所有的$x_1,x_2,\dots,x_n$有
    \[F\left(x_1,x_2,\dots,x_n\right) = F_{X_1}\left(x_1\right)F_{X_2}\left(x_2\right)\dots F_{X_n}\left(x_n\right),\]
    则称$X_1,X_2,\dots,X_n$是相互独立的.\\
    若对于所有的$x_1,x_2,\dots,x_m$;$y_1,y_2,\dots,y_n$有
    \[F\left(x_1,x_2,\dots,x_m,y_1,y_2,\dots,y_n\right) = F_1\left(x_1,x_2,\dots,x_m\right)F_2\left(y_1,y_2,\dots,y_n\right),\]
    其中$F_1$,$F_2$,F依次为随机变量$\left(X_1,X_2,\dots,X_m\right)$,$\left(Y_1,Y_2,\dots,Y_n\right)$和$\left(X_1,X_2,\dots,X_m,Y_1,Y_2,\dots,Y_n\right)$的分布函数,则称随机变量$\left(X_1,X_2,\dots,X_m\right)$和$\left(Y_1,Y_2,\dots,Y_n\right)$是相互独立的.
\end{definition*}


\begin{theorem}
    设$\left(X_1,X_2,\dots,X_m\right)$和$\left(Y_1,Y_2,\dots,Y_n\right)$相互独立,则$X_i\left(i = 1,2,\dots,m\right)$和$Y_j\left(j = 1,2,\dots,n\right)$相互独立,又若$h,g$是连续函数,则$h\left(X_1,X_2,\dots,X_m\right)$和$g\left(Y_1,Y_2,\dots,Y_n\right)$相互独立.
\end{theorem}
