\section{二维随机变量及其分布函数}

有时候样本空间不一定受一维的影响. 例如研究学龄前的儿童发育情况, 对这一地区的儿童进行抽查.
比如观察到: 身高$H$, 体重$W$.样本空间: $S=\{e\}=\{$ 某地区的全部学龄前儿童 $\}$, 那么$H(e)$ 和 $W(e)$ 是 定义在 $S$ 上的两个随机变量.
又如观察炮弹着陆点, 是由横坐标$x$, 纵坐标$y$确定的. 那么, 对于二维的随机变量, 我们应该如何分析? 

\begin{definition}[二维随机变量]
  一般地, 设 $E$ 是一个随机试验, 它的样本空间是 $S=\{e\}$, 设 $X=X(e)$ 和 $Y=Y(e)$ 是定义在 $S$上的随机变量, 由它们构成的一个向量 $(X, Y)$,叫做\emph{二维随机向量}或\emph{二维随机变量}. 第二章讨论的随机变量也叫一维随机变量.
\end{definition}

最终的二维随机变量$(X,Y)$:
    \begin{itemize}
        \item 与$X$有关, 与$Y$有关
        \item 与$X,Y$\emph{之间的关系}有关
    \end{itemize}
  因此通常将$(X,Y)$作为一个整体来研究.

  \begin{remark}
    这个实际上可以推广. 对于$n$维的随机变量也是有类似的记号.  
  \end{remark}

  

  像上面的例子一样, 我们同样可以定义随机变量的分布函数, 表示累积量的关系, 便于后面的问题分析. 
  \begin{definition}[多维变量的分布函数]
    \label{def:cumdist}
    设 $(X, Y)$ 是二维随机变量, 对于任意实数 $x, y$, 二元函数:
    $$
        F(x, y)=P\{(X \leqslant x) \cap(Y \leqslant y)\} := P\{X \leqslant x, Y \leqslant y\}
    $$
    称为二维随机变量 $(X, Y)$ 的分布函数, 或称为随机变量 $X$ 和 $Y$ 的\emph{联合分布函数}.
\end{definition}

二维的情形下, 如何求出落入小区间的概率? 即: 随机点 $(X, Y)$ 落在矩形域 $\{(x, y)\mid \left.x_1<x \leqslant x_2, y_1<y \leqslant y_2\right\}$ 的概率是什么?
  根据图像: 
  $$
        \begin{aligned}
             & P\left\{x_1<X \leqslant x_2, y_1<Y \leqslant y_2\right\}                                            \\
             & \quad=F\left(x_2, y_2\right)-F\left(x_2, y_1\right)+F\left(x_1, y_1\right)-F\left(x_1, y_2\right) .
        \end{aligned}
    $$

  \subsection{分布函数的性质}
  像一维的时候那样, 同样发现分布函数要满足这些基本的性质:

    \begin{proposition}[分步函数的性质]
      \label{prop:distfunc}
      分布函数满足如下的性质: 
      \begin{itemize}
        \item \emph{单调.}$F(x, y)$ 是变量 $x$ 和 $y$ 的不减函数
              \begin{itemize}
                  \item 对于任意固定的 $y$, 当 $x_2>x_1$ 时 $F\left(x_2, y\right) \geqslant F\left(x_1, y\right)$;
                  \item 对于任意固定的 $x$, 当 $y_2>y_1$ 时 $F\left(x, y_2\right) \geqslant F\left(x, y_1\right)$.
              \end{itemize}
        \item $0 \leqslant F(x, y) \leqslant 1 $
              \begin{itemize}
                  \item $\forall$固定的 $y, F(-\infty, y)=0$,
                  \item $\forall$固定的 $x, F(x,-\infty)=0$,
                  \item $F(-\infty,-\infty)=0, F(\infty, \infty)=1$.
              \end{itemize}
        \item \emph{右连续.}$F(x+0, y)=F(x, y), F(x, y+0)=F(x, y)$, 即 $F(x, y)$ 关于 $x$ 右连续,关于 $y$ 也右连续.
        \item \emph{非负.}对于任意 $\left(x_1, y_1\right),\left(x_2, y_2\right), x_1<x_2, y_1<y_2$,$F\left(x_2, y_2\right)-F\left(x_2, y_1\right)+F\left(x_1, y_1\right)-F\left(x_1, y_2\right) \geqslant 0$.
    \end{itemize}
    \end{proposition}

    \subsection{离散型随机变量}
    我们先从离散的情形看到一些概念:

    \begin{definition}[离散型随机变量]
        如果二维随机变量 $(X, Y)$ 全部可能取到的值是有限对或可列无限多对, 则称 $(X, Y)$ 是\emph{离散型的随机变量}.
    \end{definition}

    $$
        \begin{array}{|c|ccccc|}
            \hline Y \ddots X & x_1     & x_2     & \cdots & x_i     & \cdots \\
            \hline y_1 & p_{11}  & p_{21}  & \cdots & p_{i 1} & \cdots \\
            y_2        & p_{12}  & p_{22}  & \cdots & p_{i 2} & \cdots \\
            \vdots     & \vdots  & \vdots  &        & \vdots  &        \\
            y_j        & p_{1 j} & p_{2 j} & \cdots & p_{i j} & \cdots \\
            \vdots     & \vdots  & \vdots  &        & \vdots  &        \\
            \hline
        \end{array}
    $$
    
    我们同样希望使用表格来``枚举''每一种情况的概率. 这种表格我们称为``联合分布律''. 

    \begin{definition}
      设二维离散型随机变量 $(X, Y)$ 所有可能取的值为 $\left(x_i, y_j\right), i, j=1,2, \cdots$, 记 $P\left\{X=x_i, Y=y_j\right\}=p_{i j}, i, j=1,2, \cdots$, 由概率的定义有
    $$
        p_{i j} \geqslant 0, \quad \sum_{i=1}^{\infty} \sum_{j=1}^{\infty} p_{i j}=1 .
    $$

    称 $P\left\{X=x_i, Y=y_j\right\}=p_{i j}, i, j=1,2, \cdots$ 为二维离散型随机变量 $(X, Y)$的\emph{分布律},或随机变量 $X$ 和 $Y$ 的\emph{联合分布律}.
    \end{definition}

    实际上, 这种问题我们可以在未来处理连续性问题的时候带来一个平滑的转化. 

    \begin{example}
      设随机变量 $X$ 在 $1,2,3,4$ 四个整数中等可能地取一个值, 另一个随机变量 $Y$ 在 $1 \sim X$ 中等可能地取一整数值. 试求 $(X, Y)$ 的分布律.
    \end{example}
    \begin{solution}
      {可由乘法公式求得 $(X, Y)$ 的分布律.}
        { $$
            \begin{gathered}
                P\{X=i, Y=j\}=P\{Y=j \mid X=i\} P\{X=i\}=\frac{1}{i} \cdot \frac{1}{4}, \\
                i=1,2,3,4, j \leqslant i .
            \end{gathered}
        $$}
        {于是 $(X, Y)$ 的分布律为}
        $$
            \begin{array}{l|cccc}
                \hline Y \backslash X   & 1           & 2           & 3            & 4            \\
                \hline 1 & \frac{1}{4} & \frac{1}{8} & \frac{1}{12} & \frac{1}{16} \\
                2        & 0           &\frac{1}{8}   &\frac{1}{12} &
                \frac{1}{16}\\
                3        & 0           & 0            &\frac{1}{12} & \frac{1}{16} \\
                4        & 0           & 0            & 0            & \frac{1}{16} \\
                \hline
            \end{array}
        $$
    \end{solution}
    有了联合分布律, 自然可以得到联合分布函数. 按照\cref{def:cumdist}写一下就会发现, 
    $$
        F(x, y)=\sum_{x_i \leqslant x} \sum_{y_j \leqslant y} p_{i j},
    $$
    
    下面我们把上面的\cref{def:cumdist}推广到连续情形.

    \subsection{二维连续型随机变量}
    \begin{definition}
      对于二维随机变量 $(X, Y)$ 的分布函数 $F(x, y)$, 如果存在非负的函数 $f(x, y)$ 使对于任意 $x, y$ 有
      $$
          F(x, y)=\int_{-\infty}^y \int_{-\infty}^x f(u, v) \mathrm{d} u \mathrm{~d} v,
      $$
      则称 $(X, Y)$ 是连续型的二维随机变量, 函数 $f(x, y)$ 称为二维随机变量 $(X, Y)$ 的\emph{概率密度},或称为随机变量 $X$ 和 $Y$ 的\emph{联合概率密度}. $F$称为随机变量$X, Y$的联合分布函数. 
  \end{definition}
  观察到, 只是上述的$\sum$换成了$\int$. 而且它同样遵循一些换元规则. 我们可以使用矩阵的记号描述之. 



  和离散的情形一样, 联合分布函数也满足一些基本性质 -- 和\cref{prop:distfunc}所指示的基本类似. 
    \begin{itemize}
      \item \emph{非负.}  $f(x, y) \geqslant 0$.
      \item $\int_{-\infty}^{\infty} \int_{-\infty}^{\infty} f(u, v) \mathrm{d} u \mathrm{~d} v=F(\infty, \infty)=1$.
      \item 设 $G$ 是 $x O y$ 平面上的区域, 点 $(X, Y)$ 落在 $G$ 内的概率为
      $$
          P\{(X, Y) \in G\}=\iint_G f(u, v) \mathrm{d} u \mathrm{~d} v .
      $$
      \item 若 $f(x, y)$ 在点 $(x, y)$ 连续, 则有
      $$
          \frac{\partial^2 F(x, y)}{\partial x \partial y}=f(x, y) .
      $$
    \end{itemize}

    实际上, 上述的第(4)条性质有一个直观的解释: 若 $f(x, y)$ 在点 $(x, y)$ 处连续, 则当 $\Delta x, \Delta y$ 很小时$P\{x<X \leqslant x+\Delta x, y<Y \leqslant y+\Delta y\} \approx f(x, y) \Delta x \Delta y$. 对于一个联合密度函数, 如果其表达式为$z=f(x, y)$, 直观来看就是表示空间中的一个曲面. 这个曲面有一些特别之处:介于它和 $x O y$ 平面的空间区域的体积为 1. 并且概率分布的值就是$P\{(X, Y) \in G\}$ 的值等于以 $G$ 为底, 以曲面 $z=f(x, y)$ 为顶面的柱体体积.

    使用极限的语言也容易证明上述的第四条性质: 
    $$
            \begin{aligned}
                 & \lim _{\substack{\Delta x \rightarrow 0^{+}                                                                 \\
                \Delta y \rightarrow 0^{+}}} \frac{P\{x<X \leqslant x+\Delta x, y<Y \leqslant y+\Delta y\}}{\Delta x \Delta y} \\
                 & = \lim _{\substack{\Delta x \rightarrow 0^{+}                                                               \\
                \Delta y \rightarrow 0^{+}}} \frac{1}{\Delta x \Delta y}[ F(x+\Delta x, y+\Delta y)-F(x+\Delta x, y)           \\
                 & \quad \quad  -F(x, y+\Delta y)+F(x, y)]                                                                     \\
                 & =\frac{\partial^2 F(x, y)}{\partial x \partial y}=f(x, y)
            \end{aligned}
        $$
  \begin{remark}
    * 在微积分中, 我们了解了换元法. 并且很多时候它可以带给我们方便. 这里我们同样介绍类似的方法. 例如${P}\left\{a_1<X_1 \leqslant b_1, a_2<X_2 \leqslant b_2\right\}=\int_{a_1}^{b_1} \int_{a_2}^{b_2} f\left(x_1, x_2\right) \mathrm{d} x_1 \mathrm{~d} x_2$. 有时候为了方便起见, 我们会进行变量替换: $X_1=a_{11} Y_1+a_{12} Y_2, \quad X_2=a_{21} Y_1+a_{22} Y_2$. 这个变量替换是可逆的, 因为其行列式$\Delta=a_{11} a_{22}-a_{12} a_{21}\neq 0$. 

    带入这样的变量代换, 就将原来的区域变换为了$\mathrm{P}\{\Omega\}=\iint_{\Omega_*} f\left(a_{11} y_1+a_{12} y_2, a_{21} y_1+a_{22} y_2\right) \cdot \Delta \mathrm{d} y_1 \mathrm{~d} y_2$

    区域 $\Omega_*$ 由所有这样的点 $\left(y_1, y_2\right)$ 组成: 它的像 $\left(x_1, x_2\right)$ 在 $\Omega$ 中. 因为事件 $\left(X_1, X_2\right) \in$ $\Omega$ 和 $\left(Y_1, Y_2\right) \in \Omega_*$ 是相等的, 由此 $\left(Y_1, Y_2\right)$ 的联合密度由下式给出:
$$
g\left(y_1, y_2\right)=f\left(a_{11} y_1+a_{12} y_2, a_{21} y_1+a_{22} y_2\right) \cdot \Delta
$$

实际上, 像上面的内容进行的变量代换可以写作线性方程组的形式. 
行向量的利用需要把由 $\mathbf{R}^r$ 到 $\mathbf{R}^m$ 的线性变换写成形式
$$
\boldsymbol{Y}=\boldsymbol{X} \boldsymbol{A}
$$

即
$$
y_k=\sum_{j=1}^r a_{j_k} x_j \quad k=1, \cdots, m
$$
因此, 引入矩阵记号对我们有很大的帮助. 
  \end{remark}

  \begin{asidebox}
    回顾矩阵的一些基本内容:   
    \begin{itemize}
      \item 访问:一个矩阵在访问元素的时候, 首先是行, 其次是列. 因此, $\alpha \times \beta$ 矩阵 $\boldsymbol{A}$ 含有 $\alpha$ 行和 $\beta$ 列, 它的元素记为 $a_{j_k}$, 第 1 个下标表示行, 第 2 个下标表示列. 可以认为表现为一个线性方程组.
      \item 矩阵乘法: 类似于线性方程组的代换. 一般的规则是: 如果有$\boldsymbol{A}$ 含有 $\alpha$ 行和 $\beta$ 列, 其元素为$a_{j_k}$; 如果 $\boldsymbol{B}$ 是含有元素 $b_{j_k}$ 的 $\beta \times \gamma$ 矩阵乘积 $A B$ 是含有元素 $a_{j_1} b_{1 k}+a_{j_2} b_{2 k}+\cdots+a_{j_\beta} b_{\beta k}$ 的 $\alpha \times \gamma$ 矩阵.
      \begin{itemize}
        \item 不一定满足交换律, 但是满足结合律. 
      \end{itemize}
      \item 只含有一行的 $1 \times \alpha$ 矩阵称为行向量, 只含有一列的矩阵称为列向量.
    \end{itemize}
    矩阵的内积: 两个行向量 $\boldsymbol{x}=\left(x_1, \cdots, x_\alpha\right)$ 和 $\boldsymbol{y}=\left(y_1, \cdots, y_\alpha\right)$ 的内积定义为
    $$\boldsymbol{x} \boldsymbol{y}^{\mathrm{T}}=\boldsymbol{y} \boldsymbol{x}^{\mathrm{T}}=\sum_{j=1}^\alpha x_j y_j$$

    二次型: 如果 $a_{j k}=a_{k j}$, 即 $\boldsymbol{A}^{\mathrm{T}}=\boldsymbol{A}$, 则方阵 $\boldsymbol{A}$ 是对称的, 与对称的 $r \times r$ 矩阵 $\boldsymbol{A}$ 有关的二次型定义为
$$
\boldsymbol{x} \boldsymbol{A} \boldsymbol{x}^{\mathrm{T}}=\sum_{j, k=1}^r a_{j k} x_j x_k,
$$

其中 $x_1, \cdots, x_r$ 是未定的. 如果对于所有的非零向量 $\boldsymbol{x}$ 有 $\boldsymbol{x} \boldsymbol{A} \boldsymbol{x}^{\mathrm{T}}>0$, 是称矩阵 $\boldsymbol{A}$ 是正定的. 由上述准则推出, 正定矩阵是可逆的.
    
  \end{asidebox}

  \begin{example}
    设二维随机变量 $(X, Y)$ 具有概率密度
    $$
        f(x, y)= \begin{cases}2 \mathrm{e}^{-(2 x+y)}, & x>0, y>0, \\ 0, & \text { 其他. }\end{cases}
    $$

    (1) 求分布函数 $F(x, y)$; (2) 求概率 $P\{Y \leqslant X\}$.
  \end{example}
  \begin{solution}
    (1) $\begin{aligned} F(x, y) & =\int_{-\infty}^y \int_{-\infty}^x f(u, v) \mathrm{d} u \mathrm{~d} v \\ & = \begin{cases}\int_0^y \int_0^x 2 \mathrm{e}^{-(2 u+v)} \mathrm{d} u \mathrm{~d} v, & x>0, y>0, \\ 0, & \text { 其他. }\end{cases} \end{aligned}$

            即有 $F(x, y)= \begin{cases}\left(1-\mathrm{e}^{-2 x}\right)\left(1-\mathrm{e}^{-y}\right), & x>0, y>0, \\ 0, & \text { 其他. }\end{cases}$

            (2) 将 $(X, Y)$ 看作是平面上随机点的坐标. 即有$\{Y \leqslant X\}=\{(X, Y) \in G\},$其中 $G$ 为 $x O y$ 平面上直线 $y=x$ 及其下方的部分. 于是
                    $$\begin{aligned} P\{Y \leqslant X\} & =P\{(X, Y) \in G\}=\iint_G f(u, v) \mathrm{d} u \mathrm{~d} v \\ & =\int_0^{\infty} \int_y^{\infty} 2 \mathrm{e}^{-(2 u+v)} \mathrm{d} u \mathrm{~d} v=\frac{1}{3} . \quad \end{aligned}.$$
  \end{solution}
  \subsection{$n$维随机变量及其分布函数}
  可以推广到$n>2$的情形.
    \begin{definition}
        设 $E$ 是一个随机试验, 它的样本空间是 $S=\{e\}$, 设 $X_1=$ $X_1(e), X_2=X_2(e), \cdots, X_n=X_n(e)$ 是定义在 $S$上的随机变量, 由它们构成的一个 $n$ 维向量 $\left(X_1, X_2, \cdots, X_n\right)$ 叫\emph{做 $n$ 维随机向量}或 \emph{$n$ 维随机变量}.
    \end{definition}

    \begin{definition}
        对于任意 $n$ 个实数 $x_1, x_2, \cdots, x_n, n$ 元函数
        $$
            F\left(x_1, x_2, \cdots, x_n\right)=P\left\{X_1 \leqslant x_1, X_2 \leqslant x_2, \cdots, X_n \leqslant x_n\right\}
        $$
        称为 $n$ 维随机变量 $\left(X_1, X_2, \cdots, X_n\right)$ 的分布函数或随机变量 $X_1, X_2, \cdots, X_n$ 的联合分布函数.
    \end{definition}
    其具有类似于二维随机变量的分布函数的性质.