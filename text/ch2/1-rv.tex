\section{随机变量}

我们已经知道了一个实值函数是什么: 如果对于每一个实数 $x$ 又有唯一的 $y$ 与之对应,
那么成为$y$ 为实变量$x$的函数. 这个定义可以推广到$x$不是实数的情形. 具体的, 如果
我们把这个想法定义在样本空间上, 这样的函数就是随机变量. 意味着

\begin{definition*}
    定义在样本空间上的函数就称为随机变量.
\end{definition*}

因为他们会为每一个样本点对应一个值, 在我们进行随机试验的时候, 这个值就好像是随机的一样.

\begin{definition}[随机变量]
    设$\Omega$是某随机试验的样本空间.如果对于每个$\omega\in\Omega$,都有一个实数$X(\omega)$与其对应,这样就得到一个定义在$\Omega$上的函数
    \begin{align*}
        X=X(\omega),
    \end{align*}
    称该函数为随机变量(random variable).
\end{definition}
随机变量一般用大写英文字母$X$、$Y$、$Z$或小写希腊字母$\xi$、$\eta$、$\gamma$来表示.
\begin{itemize}
    \item 大写字母: 一个实验中的值
    \item 小写字母: 某个具体实验中的取值
\end{itemize}

按照研究的顺序, 我们现在先研究离散型(只能取有限个或者可列个值), 然后使用微积分的技巧
研究连续型(取得某一区间内的任何数值). 在这期间, 我们借助一些手段(如定义概率分布函数等)
从而可以研究混合型(一部分连续, 一部分离散)的随机变量.