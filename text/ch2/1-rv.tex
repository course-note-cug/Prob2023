\section{随机变量}

我们已经知道了一个实值函数是什么: 如果对于每一个实数 $x$ 又有唯一的 $y$ 与之对应,
那么成为$y$ 为实变量$x$的函数. 这个定义可以推广到$x$不是实数的情形. 具体的, 如果
我们把这个想法定义在样本空间上, 这样的函数就是随机变量. 意味着

\begin{definition*}
    定义在样本空间上的函数就称为随机变量.
\end{definition*}

因为他们会为每一个样本点对应一个值, 在我们进行随机试验的时候, 这个值就好像是随机的一样.

\begin{definition}[随机变量]
    设$\Omega$是某随机试验的样本空间.如果对于每个$\omega\in\Omega$, 都有一个实数$X(\omega)$与其对应, 这样就得到一个定义在$\Omega$上的函数
    \begin{align*}
        X=X(\omega),
    \end{align*}
    称该函数为随机变量(random variable).
\end{definition}

随机变量一般用大写英文字母$X$、$Y$、$Z$或小写希腊字母$\xi$、$\eta$、$\gamma$来表示.
\begin{itemize}
    \item 大写字母: 一个实验中的值
    \item 小写字母: 某个具体实验中的取值
\end{itemize}

按照研究的顺序, 我们现在先研究离散型(只能取有限个或者可列个值), 然后使用微积分的技巧
研究连续型(取得某一区间内的任何数值). 在这期间, 我们借助一些手段(如定义概率分布函数等)
从而可以研究混合型(一部分连续, 一部分离散)的随机变量.

下面我们继续就着集合说一说随机变量的初步想法. 

我们接下来考虑在有限个试验结局的情形. 设 $(\Omega, \mathscr{A}, \mathbf{P})$ 是某具有有限个结局试验的概率模型, $N(\Omega)$ 是 $\Omega$ 中基本事件的个数, 而 $\mathscr{B}$ 是 $\Omega$ 中所有子集的代数. 

比如, 掷硬币. 

\begin{example}
    a) 对于接连两次掷硬币模型, 其基本事件空间为 $\Omega=\{\mathrm{ZZ}, \mathrm{ZF}, \mathrm{FZ}, \mathrm{FF}\}$, 其中 $\mathrm{Z}$ 为 正面, $\mathrm{F}$ 为 反面. 我们利用下面的表格定义随机变量 $X=X(\omega)$ 其中$\xi(\omega)$ 是对应于 $\omega$ 的 “正面” 出现的次数:
    \begin{tabular}{c|c|c|c|c}
        \hline$\omega$ & $\mathrm{ZZ}$ & $\mathrm{ZF}$ & $\mathrm{FZ}$ & $\mathrm{FF}$ \\
        \hline$\xi(\omega)$ & 2 & 1 & 1 & 0 \\
        \hline
        \end{tabular}

        b) 随机变量 $\xi$ 的另一简单的例子是某集合 $A \in \mathscr{A}$ 的示性函数 (亦称特征函数):
        $$
        \xi=I_A(\omega)
        $$其中, $$
        I_A(\omega)= \begin{cases}1, & \omega \in A \\ 0, & \omega \notin A\end{cases}
        $$
\end{example}

当实验者遇到描绘某些记载或读数的随机变量时, 则他关心的基本问题是, 该随机
变量取各个数值的概率如何. 换句话说, 他关心的不是概率 $\mathbf{P}$ 在 $(\Omega, \mathscr{A})$ 上的分布, 而是概率在随机变量之可能值的集合上的分布. 对于我们现在研究的情形, $\Omega$由有限个点构成, 则随机变量 $\xi$ 的值域 $X$ 也是有限的.

设 $X=\left\{x_1, \cdots, x_m\right\}$, 其中$x_1, \cdots, x_m$ 是 $\xi$ 的全部可能值.

记 $\mathscr{X}$ 为值域 $X$ 上一切子集的全体, 并设 $B \in \mathscr{X}$. 当 $X$ 是随机变量 $\xi$ 的值域时, 集合 $B$ 也可以视为某个事件.

在 $(X, \mathscr{B})$ 上考虑由随机变量 $\xi$ 按照,
$$
P_{\xi}(B)=\mathbf{P}\{\omega: \xi(\omega) \in B\}, \quad B \in \mathscr{B}
$$
产生的概率 $P_{\xi}(\cdot)$. 显然, 这些概率的值完全决定于:
$$
P_{\xi}\left(x_i\right)=\mathbf{P}\left\{\omega: \xi(\omega)=x_i\right\}, \quad x_i \in X
$$
其中组数 $\left\{P_{\xi}\left(x_1\right), \cdots, P_{\xi}\left(x_m\right)\right\}$ 称做随机变量 $\xi$ 的概率分布.

仿照事件的独立性的定义, 我们给出随机变量的独立性的定义. 同样, 这表示他们不彼此依赖. 

\begin{definition}
    设$\xi_1, \cdots, \xi_r$ 在 $\mathbb{R}^1$是一组  中 (有限) 集合 $X$ 上取值的随机称随机变量 $\xi_1, \cdots, \xi_r$ 为 (全体) 独立的, 如果对于任意 $x_1, \cdots, x_r \in X$,
$$
\mathbf{P}\left\{\xi_1=x_1, \cdots, \xi_r=x_r\right\}=\mathbf{P}\left\{\xi_1=x_1\right\} \cdots \mathbf{P}\left\{\xi_r=x_r\right\}
$$
\end{definition}

\begin{shaded}
    记 $\mathscr{X}$ 是 $X$ 中所有子集的代数, 上述的内容可以等价的写成: 对于任意 $B_1, \cdots, B_r \in \mathscr{X}$,
    $$
    \mathbf{P}\left\{\xi_1 \in B_1, \cdots, \xi_r \in B_r\right\}=\mathbf{P}\left\{\xi_1 \in B_1\right\} \cdots \mathbf{P}\left\{\xi_r \in B_r\right\}
    $$
\end{shaded}