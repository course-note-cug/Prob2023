\section{连续型随机变量}
连续型随机变量 $X$ 的取值范围为一个区间. 此时$X$取某个确定值的概率总是等于零. 我们发现
事件$X=a$可能发生, 但$P(X=a)=0$. 

这就启发我们使用极限的想法来考虑之. $X$在某个小区间内取值的概率可以大于零. 
将区间分为若干段, 研究$X$在各小区间取值的概率. 分组的频率直方图刻画随机变量的概率分布,
分组越细, 频率直方图就越接近一条连续曲线. 取极限, 就有:

\begin{definition}{概率密度函数}
    如果存在非负函数$f(x)$, 满足
    $$
        P(X\leq x)=\int_{-\infty}^{x}f(t)dt,
    $$
    则称$X$为连续型随机变量, 称$f(x)$为$X$的概率密度函数(pdf, probability density function).
\end{definition}

和概率分布函数一样, 具有如下的性质:
$$
    (1)\,f(x)\ge 0,\quad (2)\ \int_{-\infty}^{+\infty}f(x)dx=1.
$$

这样确实可以来表述问题. 但是对于一半离散, 一半连续的随机变量, 该怎么统一起来?
实际上, 我们希望``忽略''在转折点上面带来的影响. 这样, 我们可以使用上面的分布函数的
思想进行.

\begin{definition}
    对任意随机变量$X$(离散或连续), 称函数
    $$F_X(x):=P(X\le x),\quad x\in \mathbb{R}$$
    为$X$的分布函数. 
\end{definition}
这是一个对于概率密度函数(或者概率分布)的累计, 自然, 它应该满足如下的性质:

\begin{itemize}
    \item 对任意实数$a<b$, 总有$F(a)\le F(b)$;
    \item $0 \le F(x) \le 1$; 
    \item $F(-\infty)=0$, $F(+\infty)=1$. 
\end{itemize}
分别对应着: 概率不可能为负数; 所有的概率必须在0和1之间; 并且所有的可能性加起来必须等于1(整个$\Omega$的概率).
\begin{takeaway}
    我们可以考察每个小区间出现的概率.

    概率密度函数 的积分是 概率分布函数, 概率分布函数求导得到概率密度函数.

    不论是离散型的或非离散型的随机变量 $X$, 都可以借助分布函数$F(x)=P(X \leq x), \quad-\infty<x<\infty$
    来描述.
\end{takeaway}
\subsection{常见的连续性随机变量}

\paragraph{(一) 均匀分布}

这可能是最基本的一个分布, 这个分布在一个区间内每个点的取值相同:

\begin{definition}
    若随机变量$X$有概率密度
    \begin{align*}
        f(x)=\left\{\begin{array}{lcl}
                        \frac1{b-a} &  & x\in [a,b]       \\
                        0           &  & \mbox{otherwise}
                    \end{array}\right. ,\qquad (a<b)
    \end{align*}
    则称$X$服从区间$[a,b]$上的均匀分布:$X\sim U(a,b)$.
\end{definition}%


\paragraph{(二) 指数分布}

假设某医院平均每小时出生$\lambda$个婴儿, 也就是说
\begin{itemize}
    \item $1$小时内出生婴儿个数$N(1)$服从Poisson分布$P(\lambda)$. 
    \item $t$小时内出生婴儿个数$N(t)$服从Poisson分布$P(\lambda t)$. 
\end{itemize}
我们希望研究婴儿出生的时间间隔$X$的概率分布. 


当$t<0$时, 有$F(t)=P(X\le t)=0$.  当$t\ge0$时, 
\begin{align*}
    F(t) & =P(X\le t\}=1-P\{X>t)          \\
         & =1-P(N(t)=0)=1-e^{-\lambda t}.
\end{align*}

因此, 当$t<0$时, $f(t)=0$;  当$t\ge0$时, $f(t)=\lambda e^{-\lambda t}$. 

基于此, 我们给出指数分布的描述:
\begin{definition}
    如果随机变量$X$有以下概率密度
    \begin{align*}
        f(x)=\left\{\begin{array}{lcl}
                        \lambda e^{-\lambda x} &  & x\ge 0           \\
                        0                      &  & \mbox{otherwise}
                    \end{array}\right. ,
    \end{align*}
    其中$\lambda>0$, 则称$X$服从参数为$\lambda$的指数分布, 记为
    \[X\sim \text{Exp}(\lambda). \]
    其分布函数为
    \begin{align*}
        F(x)=\left\{\begin{array}{lcl}
                        1- e^{-\lambda x} &  & x\ge 0 \\
                        0                 &  & x<0
                    \end{array}\right. .
    \end{align*}
\end{definition}

指数分布经常作为时间间隔或等待时间的分布. 其有一个非常重要的特性: 无记忆性.

\begin{proposition}[指数分布的无记忆性]
    随机变量$X$服从参数为$\lambda$的指数分布. 设$s,t>0$, 即有
    \[ P(X>s+t|X>s\}=P\{X>t).\]
\end{proposition}

\begin{proof}
    证明无记忆性, 只要证明 $\forall s, t \geq 0$, 有 $P(X \leq s\}=P(X \leq s+t \mid X \geq t\}$容易计算, $P(X \leq s+t \mid X \geq t\}=\frac{P\{t<X \leq s+t)}{P\{X \geq t)}=\frac{\int_1^{s+t} \lambda e^{-\lambda x} d x}{\int_t^{+\infty} \lambda e^{-\lambda x} d x}=\frac{e^{-\lambda t}-e^{-\lambda(s+t)}}{e^{-\lambda t}}=1-e^{-\lambda s}$ $P\{X \leq s)=\int_0^s \lambda e^{-\lambda x} d x=1-e^{-\lambda s}$
\end{proof}

\paragraph{(三) 正态分布}

在《高等数学 II》中我们使用极坐标的换元法了解到了如下的事实:
$$
    \int_{-\infty}^{+\infty}e^{-x^2} dx =\sqrt \pi
$$

实际上, 这个性质比较好的函数稍加改造便可得到一个非常重要的分布函数 --
正态分布函数. 我们在后面会发现, 当样本总体是大量的随机变量求和的时候,
分布将不可避免地趋向于正态分布.

\begin{definition}
    如果随机变量$X$有以下概率密度
    \begin{align*}
        f(x)=\frac1{\sqrt{2\pi}\cdot\sigma}e^{-\frac{(x-\mu)^2}{2\sigma^2}},
    \end{align*}
    其中$\mu,\sigma$为常数且$\sigma>0$, 则称$X$服从正态分布, 简记为%(normal distribution)
    \begin{align*}
        X \sim N(\mu,\sigma^2).
    \end{align*}
    称$N(0,1)$为标准正态分布.
\end{definition}

带入数据, 我们发现标准正态分布的概率密度函数(PDF)为
\begin{align*}
    \varphi(x)=\frac1{\sqrt{2\pi}}e^{-\frac{x^2}{2}},
\end{align*}
前面奇怪的系数是保证它在做积分的时候满足规范化的条件.

在这个问题中, 其分布函数
\begin{align*}
    \Phi(x)=\frac1{\sqrt{2\pi}}\int_{-\infty}^xe^{-\frac{t^2}{2}}dt.
\end{align*}
是没有显式的积分结果的. 通常这个结果需要查表得到.

如果发现一个分布不是标准正态分布, 可以用线性变换把它变成标准正态分布.

\begin{proposition}
    随机变量$X\sim N(\mu,\sigma^2)$, 则 $Z=\frac{X-\mu}{\sigma}\sim N(0,1)$.
\end{proposition}


\begin{proof}
    { $Z=\frac{X-\mu}{\sigma}$ 的分布函数为$$
            \begin{aligned}
                P(Z \leq x\} & =P\left\{{(X-\mu)}/{\sigma} \leq x\right\}=P\{X \leq \mu+\sigma x)                                         \\
                                   & =\frac{1}{\sqrt{2 \pi} \sigma} \int_{-\infty}^{\mu+\alpha t} \mathrm{e}^{-\frac{(t-\mu)^2}{2 \sigma^2}} \mathrm{~d} t
            \end{aligned}
        $$
        做变量代换, 令 $\frac{t-\mu}{\sigma}:=u$, 得$$
            \small P(Z \leq x)=\frac{1}{\sqrt{2 \pi}} \int_{-\infty}^x \mathrm{e}^{-u^2 / 2} \mathrm{~d} u=\Phi(x),
        $$
        由此知 $Z=\frac{X-\mu}{\sigma} \sim N(0,1)$.}
\end{proof}

\begin{takeaway}
    常见的连续性概率分布: 
    \begin{itemize}
        \item 均匀分布: $X\sim U(a, b): $的PDF是$$
        f(x)=\left\{\begin{array}{ll}
        \frac{1}{b-a} & x \in[a, b] \\
        0 & \text { otherwise }
        \end{array}, \quad(a<b)\right.
        $$
        \item 指数分布${X} \sim \operatorname{Exp}(\lambda)$的PDF是$$
        f(x)= \begin{cases}\lambda e^{-\lambda x} & x \geq 0 \\ 0 & \text { otherwise }\end{cases}
        $$
        \item 正态分布$X\sim N(\mu,\sigma)$的PDF是$$
        f(x)=\frac{1}{\sqrt{2 \pi} \cdot \sigma} e^{-\frac{(x-\mu)^2}{2 \sigma^2}}
        $$
    \end{itemize}
\end{takeaway}

\subsection{条件概率密度函数}

一个概率密度函数的条件概率我们应当如何定义? 使用微积分的方法, 我们不研究单个点上面的情况, 我们转而研究一个``小区间''.

研究在给定 $\xi<X_1 \leq \xi+h$ 下事件 $X_2 \leq \eta$ 的条件概率. 也就是$$\mathrm{P}\left\{X_2 \leq \eta \mid \xi<X_1 \leq \xi+h\right\}=\frac{\int_{\xi}^{\xi+h}  \int_{-\infty}^\eta f(x, y) \mathrm{d} y\mathrm{~d} x}{\int_{\xi}^{\xi+h} f_X(x) \mathrm{d} x}$$, 把分子和分母同除以 $h$, 当 $h \rightarrow 0$ 时, 右边趋于$$U_{\xi}(\eta)=\frac{1}{f_X(\xi)} \int_{-\infty}^\eta f(\xi, y) \mathrm{d} y$$
所以, 对于固定的 $\xi$, 这是 $\eta$ 的一个分布函数, 它的密度为$$u_{\xi}(\eta)=\frac{1}{f_X(\xi)} f(\xi, \eta).$$ 我们把这个定义做这种情形下的条件概率. 

