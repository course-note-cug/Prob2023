\section{离散型随机变量的概率分布}

\begin{definition}[概率分布]
    若离散型随机变量$X$的所有可能值为$\{x_k\}$,分别对应概率$\{p_k\}$,则称
    $$P\{X=x_k\}=p_k, \quad k=1,2,\cdots$$
    为$X$的概率分布(分布律).
\end{definition}

为了方便起见, 可以把概率列表以得到直观描述:
\[\begin{array}{|c||c|c|c|c|c|}\hline
        X & x_1 & x_2 & \cdots & x_k & \cdots \\ \hline
        P & p_1 & p_2 & \cdots & p_k & \cdots \\ \hline
    \end{array}\]
这张表我们称作随机变量$X$的分布列.

由于每一个概率是非负的, 并且我们遍历的整个概率空间, 因此概率分布具有如下的性质:
$\displaystyle \left\{ \begin{array}{l} p_k\ge 0, \quad k=1,2,\cdots \\
        \sum\limits_{k}{p_k}=1\end{array} \right.$

有时候我们对于小于某一个特定值的变量的取值内容感兴趣.概率分布函数就是统计小于某一个
特定值发生的概率.
同时这种定义也可以让我们从一个具体的
点之中脱身. 我们的视野就可以看得更广了, 这种技巧在连续的时候很有用处.

\begin{definition}[概率分布函数]
    设离散型随机变量$X$的概率分布为
    \begin{align*}
        P\{X=x_k\}=p_k,\qquad k=1,2,\cdots.
    \end{align*}
    则定义$X$的分布函数为
    \begin{align*}
        F_X(x)=P\{X\le x\}=\sum_{x_k\le x} p_k,
    \end{align*}
    这里的和式是对所有满足$x_k\le x$的$p_k$求和.
\end{definition}

注意, 这样的定义下, 函数是右连续的. 也就是在一个间断点$a$的时候, 
$\lim_{x\to a^+}f(x)=f(a)$.

有了这样的一番定义之后, 我们来看几个离散型随机变量.

\begin{takeaway}
    
    我们可以考察每个情况出现的概率. 

    概率分布 的前缀和是 概率分布函数, 概率分布函数的差分是概率分布. 
\end{takeaway}

\subsection{经典的离散型随机变量}

\paragraph{(一) 0-1分布(两点分布)}

\begin{example}
    $100$件产品中,有$98$件是正品,$2$件是次品,今从中随机地抽取一件,若规定
    \[X=\left\{ \begin{array}{ll} 1, & \text{取到正品;} \\
             0,      & \text{取到次品;}\end{array} \right.\]
    则随机变量$X$的概率分布表为
    $\begin{array}{|c|c|c|}\hline
            X & 0    & 1    \\ \hline
            P & 0.02 & 0.98 \\ \hline
        \end{array}$.
\end{example}

\begin{definition}[两点分布(0-1分布)]
    若随机变量$X$的概率分布为
    $\begin{array}{|c||c|c|}\hline
            X & 0   & 1 \\ \hline
            P & 1-p & p \\ \hline
        \end{array}$,
    则称$X$服从参数为$p$的两点分布(或 0-1分布).记为$$X\sim B(1,p)$$
\end{definition}

\paragraph{(二-$\varepsilon$) 几何分布}

\begin{example}
    抛一枚硬币, 硬币出现正面的概率为$p$, 请问前$k-1$次抛出反面, 第$k$次出现正面的概率.
\end{example}

\begin{definition}
    若随机变量$X$的概率分布为
    \[P\{X=k\}=p(1-p)^{k-1},\quad k=1,2,3\cdots,\]
    则称$X$服从参数为$p$的几何分布,记为$X\sim G(p)$.
\end{definition}

几何分布是接下来的例子要说的二项分布, 当次数恰好等于1时的特例.

\paragraph{(二) 二项分布}

\begin{example}
    若某射手每次射击命中的概率均为$p$,现进行$n$次独立射击,求恰有$k$次命中的概率.
\end{example}
% 需要 pifont 包
\def\1{\ding{51}} % 勾
\def\0{\ding{55}} % 叉
先研究射击次数$n=4$的特殊情形.此时有


{$$
    \begin{tabular}{|c|l|}
        \hline
        $k=0$ & \0\0\0\0                                                   \\
        \hline
        $k=1$ & \1\0\0\0, \0\1\0\0, \0\0\1\0, \0\0\0\1                     \\
        \hline
        $k=2$ & \1\1\0\0, \1\0\1\0, \1\0\0\1, \0\1\1\0, \0\1\0\1, \0\0\1\1 \\
        \hline
        $k=3$ & \1\1\1\0, \1\1\0\1, \1\0\1\1, \0\1\1\1                     \\
        \hline
        $k=4$ & \1\1\1\1                                                   \\
        \hline
    \end{tabular}
$$}

\begin{definition}[$n$重Bernoulli实验]
    只有两种可能结果的试验称为Bernoulli试验.将一Bernoulli试验独立重复$n$次称为$n$重Bernoulli试验.
\end{definition}
        
我们发现, 上面的问题就是$k$重Bernoulli实验恰好成功$k$次的概率. 根据计数的技巧, 我们发上发现:

\begin{theorem}[Bernoulli定理]
    设一次试验中事件$A$发生的概率为$p(0<p<1)$,则$n$重Bernoulli试验中,事件$A$恰好发生$k(0\le k\le n)$ 次的概率为
    \begin{align*}
        b(k;n,p)={n\choose k} p^k (1-p)^{n-k}.
    \end{align*}
\end{theorem}

对二项分布$B(n,p)$,当$n$充分大、$p$很小时,(但是保证$np_n=\lambda$不变)形成的函数曲线是什么?
\begin{align*}
                   & {n \choose k}p_n^k(1-p_n)^{n-k}                                                                                                                                                    \\
    ={}            & \frac{n(n-1)\cdots (n-k+1)}{k!}(\frac{\lambda}{n})^k(1-\frac{\lambda}{n})^{n-k}                                                                                                    \\
    ={}            & \frac{\lambda^k}{k!}\textcolor{teal}{\Big(1\cdot (1-\frac{1}{n})\cdots (1-\frac{k-1}{n})\Big)}\textcolor{red}{(1-\frac{\lambda}{n})^n}\textcolor{blue}{(1-\frac{\lambda}{n})^{-k}} \\
    \rightarrow {} & \frac{\lambda^k}{k!}\cdot \textcolor{teal}{1}\cdot \textcolor{red}{e^{-\lambda}}\cdot \textcolor{blue}{1}
\end{align*}

因此我们发现, 当$n\to\infty$的时候, 有
\begin{theorem}
    设$\lambda>0$为一个常数,$n$为任意正整数,且$np_n=\lambda$,则对任意一个\textcolor{blue}{固定的}非负整数$k$,
    $$
        \lim_{n\rightarrow \infty}{n\choose k}p_n^k(1-p_n)^{n-k}=\frac{\lambda^k}{k!}e^{-\lambda}.
    $$
\end{theorem}
这就是Poisson分布.

\paragraph{(三) Poisson分布}

Poisson分布的定义如下.
\begin{definition}[Poisson分布]
    如果随机变量$X$服从以下分布律
    \[ P\{X=k\}=\frac{\lambda^k}{k!}e^{-\lambda},\quad k=0,1,\cdots \]
    其中$\lambda>0$,则称$X$服从参数为$\lambda$的Poisson分布,记为
    $X\sim P(\lambda).$
\end{definition}%
Poisson分布常与单位时间(或单位面积、单位产品等)上的计数过程相联系.