% !TEX root = main.tex
\documentclass{article}
\usepackage{amsmath, amsthm, amssymb, amsfonts}
\usepackage{thmtools}
\usepackage{graphicx}
\usepackage{setspace}
\usepackage{geometry}
\usepackage{float}
\usepackage{hyperref}
\usepackage[utf8]{inputenc}
\usepackage[english]{babel}
\usepackage{framed}
\usepackage[dvipsnames]{xcolor}
\usepackage{tcolorbox}

\colorlet{LightGray}{White!90!Periwinkle}
\colorlet{LightOrange}{Orange!15}
\colorlet{LightGreen}{Green!15}

\newcommand{\HRule}[1]{\rule{\linewidth}{#1}}

\declaretheoremstyle[name=Theorem,]{thmsty}
\declaretheorem[style=thmsty,numberwithin=section]{theorem}
\tcolorboxenvironment{theorem}{colback=LightGray}

\declaretheoremstyle[name=Proposition,]{prosty}
\declaretheorem[style=prosty,numberlike=theorem]{proposition}
\tcolorboxenvironment{proposition}{colback=LightOrange}

\declaretheoremstyle[name=Principle,]{prcpsty}
\declaretheorem[style=prcpsty,numberlike=theorem]{principle}
\tcolorboxenvironment{principle}{colback=LightGreen}

\declaretheoremstyle[name=Definition,]{prcpsty}
\declaretheorem[style=prcpsty,numberlike=theorem]{definition}
\tcolorboxenvironment{definition}{colback=white}

\usepackage{enumitem}

\setlist{nosep}

\setstretch{1.2}
\geometry{
    textheight=9in,
    textwidth=5.5in,
    top=1in,
    headheight=12pt,
    headsep=25pt,
    footskip=30pt
}

\usepackage{ctex}
% \usepackage[utf8]{inputenc}

\begin{document}

% \title{ \normalsize \textsc{}
		\\ [2.0cm]
		\HRule{1.5pt} \\
		\LARGE \textbf{{Lecture Notes}
		\HRule{2.0pt} \\ [0.6cm] \LARGE{2023秋 ~~~概率论与数理统计A} \vspace*{10\baselineskip}}
		}
\date{}
\author{\textbf{Author} \\ 
		AUGPath \\
		China University of Geosciences~(Wuhan) \\
		Department of Computer Science \\
		\today}

\maketitle
\newpage
\setcounter{tocdepth}{1}
\tableofcontents
\newpage


\section{概率中的基本概念}

\subsection{随机实验}

\begin{definition}
    随机试验是指: 
    \begin{itemize}
        \item 试验结果不止一个,但能明确所有的结果; 
        \item 试验前不能预知出现哪种结果。
    \end{itemize}
\end{definition}

\begin{definition}
    随机事件: 对随机试验的结果的陈述称为\textbf{随机事件},简称\textbf{事件}. 
\end{definition}

\begin{definition}
    样本空间: 随机试验的每个可能结果$\omega$称为一个\textbf{样本点}. 全体样本点组成的集合$\Omega$称为\textbf{样本空间}. 
    $$
        \Omega:=\{\omega\}.
    $$
    每个\textbf{随机事件}都可用样本空间的某个子集表示,通常记为$A, B, C$等. 
    在随机试验中,当事件中的一个样本点出现时,称该事件\textbf{发生}. 
\end{definition}

\subsection{事件的关系与运算}

\begin{definition}[事件的关系] 
    若事件$A$发生时,事件$B$一定发生. 则称事件$A$包含于事件$B$(或事件$B$ \textbf{包含} $A$),记作
    $$A\subset B \ (\text{或}B\supset A)$$

    对任意事件$A$,有$\emptyset \subset A\subset \Omega$. 

    若$A\subset B$,且$B\subset A$,则称事件$A$与$B$ \textbf{相等},记作$A=B$. 
    
\end{definition}

下面来考察事件的运算: 

\begin{definition}[事件的并]
    
        “事件$A$、$B$至少有一个发生”
    称为事件$A$与$B$的\textbf{和}或\textbf{并}(union),记作
    $$A\cup B \ (\text{或}A+B)$$
    也就是
    $$A\cup B=\{\omega | \omega\in A \ \text{or}\ \omega\in B\}$$
\end{definition}

\begin{remark}
    使用数学归纳法, 事件的并可以推广到多个的情形:如$n$个事件的并
    $$\bigcup_{i=1}^{n} A_i =\text{“事件$A_1, \cdots, A_n$至少有一个发生”}$$
    可数个事件的并
    $$\bigcup_{i=1}^{\infty} A_i =\text{“事件$A_1, A_2, \cdots$至少有一个发生”}$$
\end{remark}

\begin{definition}
        “事件$A$、$B$同时发生”
    称为事件$A$与$B$的\textbf{积}或\textbf{交}(intersection),记作
    $$AB \ (\text{或}A\cap B)$$
    也就是$$A\cap B=\{\omega | \omega\in A \ \text{and}\ \omega\in B\}$$
\end{definition}

\begin{remark}
    事件的交可以推广到多个的情形:如$n$个事件的交
    $$\bigcap_{i=1}^{n} A_i =\text{“事件$A_1, \cdots, A_n$全都发生”}$$
    可数个事件的交
    $$\bigcap_{i=1}^{\infty} A_i =\text{“事件$A_1, A_2, \cdots$全都发生”}$$
\end{remark}

\begin{definition}
    “事件$A$发生,但$B$不发生”
称为事件$A$与$B$的\textbf{差},记作
$$A-B$$
也就是
$$A- B=\{\omega | \omega\in A \ \text{and}\ \omega\notin B\}$$
\end{definition}

像集合那样, 我们同样可以引入事件的关系: 

\begin{definition}
    若$AB=\emptyset$,则称$A$与$B$ \textbf{互斥}(或称$A$与$B$ \textbf{不相容}),%(mutually exclusive)
    即$A$与$B$不可能同时发生. 
\end{definition}

\begin{definition}
    称$\Omega-A$为事件$A$的\textbf{对立}事件(或称$A$的\textbf{补}),记为$\overline{A}$. 
    它表示“事件$A$不发生”. 
\end{definition}


像集合那样, 事件具有如下的运算规律: 
\begin{itemize}
    \item 交换律
    \begin{itemize}
        \item $AB=BA$,$A\cup B=B\cup A$
    \end{itemize}
    \item 结合律
    \begin{itemize}
        \item $(AB)C=A(BC)$,$(A\cup B)\cup C=A\cup(B\cup C)$
    \end{itemize}
    \item 分配律
    \begin{itemize}
        \item $A(B\cup C)=AB\cup AC$,$A(B-C)=AB-AC$
    \end{itemize}
    \item 对偶律
    \begin{itemize}
        \item $\overline{AB}=\overline{A}\cup\overline{B}$,$\overline{A\cup B}=\overline{A}\,\overline{B}$
    \end{itemize}
\end{itemize}

下面来考察一些常见的化简运算关系的等式: 

\begin{proposition}
        对任意两个事件$A$和$B$,总有$ A-B=A-AB$. 
\end{proposition}

\begin{proposition}
    事件$A$、$B$ \textbf{对立}当且仅当$A$、$B$\textbf{互斥}且$A\cup B=\Omega$. 
\end{proposition}
\begin{example}
    设$A,B$为两个事件,则有
        \begin{itemize}
            \item $A\overline{B}=A-B=A-AB$; 
            \item $A=AB\cup A\overline{B}$. 
        \end{itemize}
\end{example}

\begin{solution}
    用事件运算的分配律:
    \begin{itemize}
        \item $A\overline{B}=A(\Omega-B)=A\Omega-AB=A-AB$; 
        \item $AB\cup A\overline{B}=A(B\cup\overline{B})=A\Omega=A$. 
    \end{itemize}
\end{solution}

\begin{example}
    $A$, $B$, $C$ 表示事件
    \begin{itemize}
        \item $A$发生: $A$; 
        \item 仅$A$发生: $A\cap \bar{B}\cap \bar{C}$; 
        \item 恰有一个发生:$A \bar B \bar C\cup \bar AB\bar C\cup \bar A\bar BC$;
        \item 至少有一个发生:$A\cup B\cup C$;
        \item 至多有一个发生:$\bar A\bar B\bar C\cup A \bar B \bar C \cup \bar AB\bar C\cup \bar A\bar BC$;
        \item 都不发生:$\bar A\bar B\bar C$;
        \item 不全部发生: $\overline{ABC}=\bar A\cup \bar B\cup \bar C$.
    \end{itemize}
\end{example}

\section{事件的概率}

事件的\textbf{概率}:刻画试验中随机事件发生的\textbf{可能性大小}. 

\subsection{概率的统计定义}
\begin{definition}
    设在$n$次试验中,事件$A$发生了$m$次,则称
    \begin{align*}
        f_n(A):=\frac{m}{n}
    \end{align*}
    为事件$A$发生的\textbf{频率}(frequency). 
\end{definition}

\begin{definition*}%[概率的统计定义]
    在相同条件下重复进行的试验中,若随着试验次数$n$的增加,
    事件$A$发生的频率稳定在某一常数$p$附近,
    则称$p$为事件$A$的\textbf{概率},记作$P(A)=p$. 
\end{definition*}
也就是概率是频率的稳定值. 实际应用中常将大量重复试验中事件的频率作为概率的近似估计. 

\begin{proposition*}
    频率的性质: 
    \begin{itemize}
        \item $0\le f_n(A)\le 1$; 
        \item $f_n(\Omega)=1,\ f_n(\emptyset)=0$; 
        \item 若事件$A_1, A_2, \cdots, A_k$两两互斥,则
              $$f_n \left( \bigcup_{i=1}^k A_i \right)=\sum_{i=1}^k f_n(A_i)$$
    \end{itemize}
\end{proposition*}

由于上述的性质, 我们给出概率的数学公理化定义: 
\begin{definition}[概率的公理化定义]
    设$\Omega$是样本空间,定义概率空间$(\Omega,\mathcal{F},P)$。对每个事件$A\in \mathcal{F}$定义一个实数$P(A)$与之对应。
    集合函数$P$满足以下条件:
    \begin{itemize}
        \item 非负性:对任意事件$A$,均有$P(A)\ge 0$; 
        \item 规范性:$P(\Omega)=1$; 
        \item 可加性:若事件序列$\{A_n\}_{n\ge 1}$两两互斥,则
              $$P \left( \bigcup_{n=1}^{\infty} A_n \right)=\sum_{n=1}^{\infty} P(A_n)$$
    \end{itemize}
    则称$P(A)$为事件$A$的\textbf{概率}(probability). 
\end{definition}

\subsection{概率的加法公式}
\begin{proposition}[加法公式]
    若两个事件$A,B$互斥,则
    $$P(A\cup B)=P(A)+P(B).$$
\end{proposition}

\begin{remark}
    由加法公式可得到如下性质:
    \begin{itemize}
        \item 对任意事件$A$,有
              $P(A)=1-P\left(\overline{A}\right).$
        \item 对任意两个事件$A,B$,有
              $$P(A\cup B)=P(A)+P(B)-P(AB).$$
    \end{itemize}
\end{remark}

\begin{remark}
    若三个事件$A_1, A_2, A_3$两两互斥,则
    $$\pmb P(A_1 \cup A_2 \cup A_3) = P(A_1)+P(A_2)+P(A_3).$$
    对任意三个事件$A_1, A_2, A_3$,有
    \begin{align*}
        \pmb P(A_1 \cup A_2 \cup A_3) & \pmb=\color{blue}P(A_1)+P(A_2)+P(A_3)              \\
                                      & \phantom=\color{red}-P(A_1A_2)-P(A_1A_3)-P(A_2A_3) \\
                                      & \phantom=\color{blue}+P(A_1A_2A_3).
    \end{align*}%
\end{remark}

\begin{remark}
    更一般地, 可以使用容斥原理计算: 
    若$n$个事件$A_1, A_2, \cdots, A_n$两两互斥,则
    $$\pmb P\left( \bigcup_{i=1}^n A_i \right)=\sum_{i=1}^n P(A_i).$$
    \vspace{0.2in}
    对任意$n$个事件$A_1, A_2, \cdots, A_n$,有
    $$\pmb P\left( \bigcup_{i=1}^n A_i \right)=\sum_{k=1}^n \left[ (-1)^{k+1} \sum_{1\le i_1\le \cdots\le i_k \le n} P(A_{i_1}\cdots A_{i_k}) \right].$$
\end{remark}



\end{document}
