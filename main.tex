% !TEX root = main.tex
\documentclass{article}
\usepackage{amsmath, amsthm, amssymb, amsfonts}
\usepackage{thmtools}
\usepackage{graphicx}
\usepackage{setspace}
\usepackage{geometry}
\usepackage{float}
\usepackage{hyperref}
\usepackage[utf8]{inputenc}
\usepackage[english]{babel}
\usepackage{framed}
\usepackage[dvipsnames]{xcolor}
\usepackage{tcolorbox}

\colorlet{LightGray}{White!90!Periwinkle}
\colorlet{LightOrange}{Orange!15}
\colorlet{LightGreen}{Green!15}

\newcommand{\HRule}[1]{\rule{\linewidth}{#1}}

\declaretheoremstyle[name=Theorem,]{thmsty}
\declaretheorem[style=thmsty,numberwithin=section]{theorem}
\tcolorboxenvironment{theorem}{colback=LightGray}

\declaretheoremstyle[name=Proposition,]{prosty}
\declaretheorem[style=prosty,numberlike=theorem]{proposition}
\tcolorboxenvironment{proposition}{colback=LightOrange}

\declaretheoremstyle[name=Principle,]{prcpsty}
\declaretheorem[style=prcpsty,numberlike=theorem]{principle}
\tcolorboxenvironment{principle}{colback=LightGreen}

\declaretheoremstyle[name=Definition,]{prcpsty}
\declaretheorem[style=prcpsty,numberlike=theorem]{definition}
\tcolorboxenvironment{definition}{colback=white}

\usepackage{enumitem}

\setlist{nosep}

\setstretch{1.2}
\geometry{
    textheight=9in,
    textwidth=5.5in,
    top=1in,
    headheight=12pt,
    headsep=25pt,
    footskip=30pt
}

\usepackage{ctex}
% \usepackage[utf8]{inputenc}

\begin{document}

% \title{ \normalsize \textsc{}
		\\ [2.0cm]
		\HRule{1.5pt} \\
		\LARGE \textbf{{Lecture Notes}
		\HRule{2.0pt} \\ [0.6cm] \LARGE{2023秋 ~~~概率论与数理统计A} \vspace*{10\baselineskip}}
		}
\date{}
\author{\textbf{Author} \\ 
		AUGPath \\
		China University of Geosciences~(Wuhan) \\
		Department of Computer Science \\
		\today}

\maketitle
\newpage
\setcounter{tocdepth}{1}
\tableofcontents
\newpage


\part{概率论中基本的概念}
\section{概率中的基本概念}
\subsection{随机实验}

考虑某项试验, 其结果在某一组条件下会有若干种不同的结局 (现象) $\omega_1, \cdots, \omega_N$ . 关于这些结局具体是什么并不重要, 而是把他们抽象为一组点. 我们把这些结局 $\omega_1, \cdots, \omega_N$ 称做基本事件, 而把一切结局的全体
$$
\Omega=\left\{\omega_1, \cdots, \omega_N\right\}
$$

称做基本的事件空间, 或样本空间.

\begin{example}
    对于 “掷一枚硬币”, 基本事件空间由两个点组成:
$$
\Omega=\{\mathrm{Z}, \mathrm{F}\},
$$

其中 Z 表示出现 “正面”, 而 $\mathrm{F}$ 表示出现 “反面”. (这时假设, 诸如 “硬币在棱上立着”, “硬币丢失”.$\cdots \cdots$ 的情况不会出现.) 也就是假设不出现 “正面” 就出现 “反面”.

    将一枚硬币重复掷 $n$ 次, 基本事件空间为
$$
\Omega=\left\{\omega: \omega=\left(a_1, \cdots, a_n\right)\right\}, a_i=\text { Z或 } \mathrm{F},
$$

且基本事件的总数 $N(\Omega)=2^n$.
\end{example}

除基本事件空间的概念外,现在引进重要概念事件. 事件的概念, 是建立所考察试验的各种概率模型 (“理论”) 的基础. 在试验的结果中, 试验者一般并不关心究竟出现了哪种具体的结局, 而关心出现的结局属于一切结局集合的哪个子集. 满足试验条件的一切子集 $A \subseteq \Omega$, 分为两种类型: “结局 $\omega \in A$ ” 或 “结局 $\omega \notin A$ ”. 我们称这样的子集 $A$ 为事件.

\begin{example}
    将一枚硬币重复掷三次, 一切可能结局的空间 $\Omega$, 由 8 个点构成:
$$
\Omega=\{000,001,010,011,100,101,110,111\}
$$

其中 0 和 1 分别表示郑出 “正面” 和 “反面”. 如果由 “一组条件” 可以记录 (确定、测量等) 所有 3 次郑硬币的结果, 则例如
$$
A=\{000,001,010,100\}
$$

就是事件: “将一枚硬币重复掷三次” 正面至少出现两次. 假如由 “一组条件” 只能确定第一次郑出的结果, 则 $A$ 已经不能称为事件, 因为关于 “试验的具体结局 $\omega$ 是否属于 $A "$, 既不能肯定也不能否定.
\end{example}

在随机试验中,当事件中的一个样本点出现时,称该事件发生.

\subsection{事件的关系与运算}

\begin{definition}[事件的关系]
    若事件$A$发生时, 事件$B$一定发生. 则称事件$A$包含于事件$B$(或事件$B$ \textbf{包含} $A$), 记作
    $$A\subset B \ (\text{或}B\supset A)$$

    对任意事件$A$, 有$\emptyset \subset A\subset \Omega$.

    若$A\subset B$, 且$B\subset A$, 则称事件$A$与$B$ \textbf{相等}, 记作$A=B$.

\end{definition}

下面来考察事件的运算:

\begin{definition}[事件的并]

    “事件$A$、$B$至少有一个发生”
    称为事件$A$与$B$的\textbf{和}或\textbf{并}(union), 记作
    $$A\cup B \ (\text{或}A+B)$$
    也就是
    $$A\cup B=\{\omega | \omega\in A \ \text{or}\ \omega\in B\}$$
\end{definition}

\begin{remark}
    使用数学归纳法, 事件的并可以推广到多个的情形:如$n$个事件的并
    $$\bigcup_{i=1}^{n} A_i =\text{“事件$A_1, \cdots, A_n$至少有一个发生”}$$
    可数个事件的并
    $$\bigcup_{i=1}^{\infty} A_i =\text{“事件$A_1, A_2, \cdots$至少有一个发生”}$$
\end{remark}

\begin{definition}
    “事件$A$、$B$同时发生”
    称为事件$A$与$B$的\textbf{积}或\textbf{交}(intersection), 记作
    $$AB \ (\text{或}A\cap B)$$
    也就是$$A\cap B=\{\omega | \omega\in A \ \text{and}\ \omega\in B\}$$
\end{definition}

\begin{remark}
    事件的交可以推广到多个的情形:如$n$个事件的交
    $$\bigcap_{i=1}^{n} A_i =\text{“事件$A_1, \cdots, A_n$全都发生”}$$
    可数个事件的交
    $$\bigcap_{i=1}^{\infty} A_i =\text{“事件$A_1, A_2, \cdots$全都发生”}$$
\end{remark}

\begin{definition}
    “事件$A$发生, 但$B$不发生”
    称为事件$A$与$B$的\textbf{差}, 记作
    $$A-B$$
    也就是
    $$A- B=\{\omega | \omega\in A \ \text{and}\ \omega\notin B\}$$
\end{definition}

像集合那样, 我们同样可以引入事件的关系:

\begin{definition}
    若$AB=\emptyset$, 则称$A$与$B$ \textbf{互斥}(或称$A$与$B$ \textbf{不相容}), %(mutually exclusive)
    即$A$与$B$不可能同时发生.
\end{definition}

\begin{definition}
    称$\Omega-A$为事件$A$的\textbf{对立}事件(或称$A$的\textbf{补}), 记为$\overline{A}$.
    它表示“事件$A$不发生”.
\end{definition}


像集合那样, 事件具有如下的运算规律:
\begin{itemize}
    \item 交换律
          \begin{itemize}
              \item $AB=BA$, $A\cup B=B\cup A$
          \end{itemize}
    \item 结合律
          \begin{itemize}
              \item $(AB)C=A(BC)$, $(A\cup B)\cup C=A\cup(B\cup C)$
          \end{itemize}
    \item 分配律
          \begin{itemize}
              \item $A(B\cup C)=AB\cup AC$, $A(B-C)=AB-AC$
          \end{itemize}
    \item 对偶律
          \begin{itemize}
              \item $\overline{AB}=\overline{A}\cup\overline{B}$, $\overline{A\cup B}=\overline{A}\,\overline{B}$
          \end{itemize}
\end{itemize}

下面来考察一些常见的化简运算关系的等式:

\begin{proposition}
    对任意两个事件$A$和$B$, 总有$ A-B=A-AB$.
\end{proposition}

\begin{proposition}
    事件$A$、$B$ \textbf{对立}当且仅当$A$、$B$\textbf{互斥}且$A\cup B=\Omega$.
\end{proposition}
\begin{example}
    设$A,B$为两个事件, 则有
    \begin{itemize}
        \item $A\overline{B}=A-B=A-AB$;
        \item $A=AB\cup A\overline{B}$.
    \end{itemize}
\end{example}

\begin{solution}
    用事件运算的分配律:
    \begin{itemize}
        \item $A\overline{B}=A(\Omega-B)=A\Omega-AB=A-AB$;
        \item $AB\cup A\overline{B}=A(B\cup\overline{B})=A\Omega=A$.
    \end{itemize}
\end{solution}

\begin{example}
    $A$, $B$, $C$ 表示事件
    \begin{itemize}
        \item $A$发生: $A$;
        \item 仅$A$发生: $A\cap \bar{B}\cap \bar{C}$;
        \item 恰有一个发生:$A \bar B \bar C\cup \bar AB\bar C\cup \bar A\bar BC$;
        \item 至少有一个发生:$A\cup B\cup C$;
        \item 至多有一个发生:$\bar A\bar B\bar C\cup A \bar B \bar C \cup \bar AB\bar C\cup \bar A\bar BC$;
        \item 都不发生:$\bar A\bar B\bar C$;
        \item 不全部发生: $\overline{ABC}=\bar A\cup \bar B\cup \bar C$.
    \end{itemize}
\end{example}

\begin{takeaway}
{
    可以使用集合描述事件, 离散数学中学过的集合的运算将允许我们对于事件进行化简和操作.
}
\end{takeaway}
% !TEX root = main.tex
\section{事件的概率}

事件的\textbf{概率}:刻画试验中随机事件发生的\textbf{可能性大小}.

\subsection{概率的统计定义}
\begin{definition*}
    设在$n$次试验中, 事件$A$发生了$m$次, 则称
    \begin{align*}
        f_n(A):=\frac{m}{n}
    \end{align*}
    为事件$A$发生的\textbf{频率}(frequency).
\end{definition*}

\begin{definition*}%[概率的统计定义]
    在相同条件下重复进行的试验中, 若随着试验次数$n$的增加, 
    事件$A$发生的频率稳定在某一常数$p$附近, 
    则称$p$为事件$A$的\textbf{概率}, 记作$P(A)=p$.
\end{definition*}
也就是概率是频率的稳定值. 实际应用中常将大量重复试验中事件的频率作为概率的近似估计.

\begin{proposition*}
    频率的性质:
    \begin{itemize}
        \item $0\le f_n(A)\le 1$;
        \item $f_n(\Omega)=1,\ f_n(\emptyset)=0$;
        \item 若事件$A_1, A_2, \cdots, A_k$两两互斥, 则
              $$f_n \left( \bigcup_{i=1}^k A_i \right)=\sum_{i=1}^k f_n(A_i)$$
    \end{itemize}
\end{proposition*}

由于上述的性质, 我们给出概率的数学公理化定义:
\begin{definition}[概率的公理化定义]
    \label{def:prob}
    设$\Omega$是样本空间, 定义概率空间$(\Omega,\mathcal{F},P)$. 对每个事件$A\in \mathcal{F}$定义一个实数$P(A)$与之对应. 
    集合函数$P$满足以下条件:
    \begin{itemize}
        \item 非负性:对任意事件$A$, 均有$P(A)\ge 0$;
        \item 规范性:$P(\Omega)=1$;
        \item 可加性:若事件序列$\{A_n\}_{n\ge 1}$两两互斥, 则
              $$P \left( \bigcup_{n=1}^{\infty} A_n \right)=\sum_{n=1}^{\infty} P(A_n)$$
    \end{itemize}
    则称$P(A)$为事件$A$的\textbf{概率}(probability).
\end{definition}


\mn{事件可以先简单认为就是$\Omega$一堆子集构成的集合, 当然有一些条件需要满足. 对于初学者而言, 下面的就先不用看了. 这些内容只是为了那些学习过离散数学并且知道这种情形的作用的同学准备的. }
这里的事件用集合的语言描述, 考虑集合 $A \subseteq \Omega$ 的某个集系 $\mathscr{A}_0$, 则利用集合运算 $\cup, \cap$ 与 $\backslash$ 可以由 $\mathscr{A}_0$ 构造新集系, 其中元素也是事件. 给这些事件补充上必然事件 $\Omega$ 和不可能事件 $\varnothing$, 得集系 $\mathscr{A}$, 则 $\mathcal{A}$ 是代数. 所谓 “代数” 即 $\Omega$ 的这样的集系, 满足
\begin{itemize}
    \item [1)] $\Omega \in \mathcal{F}$,
    \item    [2)] 若 $A \in \mathcal{F}, B \in \mathcal{F}$, 则集合 $A \cup B, A \cap B, A \backslash B$ 也都属于 $\mathcal{F}$.
\end{itemize}

例如这些内容
\begin{example}
    a) $\mathscr{A}=\{\Omega, \varnothing\}$集系由 $\Omega$ 和空集 $\varnothing$ 构成, 称做平凡代数;

b) $\mathscr{A}=\{A, \bar{A}, \Omega, \varnothing\}$事件 $A$ 产生的集系;

c) $\mathscr{A}=\{A: A \subseteq \Omega\}$ $\Omega$ 全部子集的集系 (包括空集 $\varnothing$ ).
\end{example}

这些事件代数可以按分割的方式得到: 我们称集系
$$
\mathscr{D}=\left\{D_1, \cdots, D_n\right\}
$$

构成集合 $\Omega$ 的一个分割, 而 $D_1, \cdots, D_n$ 是该分割的原子, 如果 $D_1, \cdots, D_n$ 非空且两两不相容, 而它们的和等于 $\Omega$ :
$$
D_1+\cdots+D_n=\Omega .
$$

例如, 假定集合 $\Omega$ 由 3 个点构成: $\Omega=\{1,2,3\}$, 则存在 5 个不同的分割:
$$
\begin{array}{ll}
\mathscr{D}_1=\left\{D_1\right\} & D_1=\{1,2,3\} \\
\mathscr{D}_2=\left\{D_1, D_2\right\} & D_1=\{1,2\}, D_2=\{3\} \\
\mathscr{D}_3=\left\{D_1, D_2\right\} & D_1=\{1,3\}, D_2=\{2\} \\
\mathscr{D}_4=\left\{D_1, D_2\right\} & D_1=\{2,3\}, D_2=\{1\} \\
\mathscr{D}_5=\left\{D_1, D_2, D_3\right\} & D_1=\{1\}, D_2=\{2\} ; D_3=\{3\} .
\end{array}
$$

\begin{wrapfigure}{l}{0.6\textwidth}
    % \usepackage[usenames,dvipsnames]{pstricks}
% \usepackage{pstricks-add}
% \usepackage{epsfig}
% \usepackage{pst-grad} % For gradients
% \usepackage{pst-plot} % For axes
% \usepackage[space]{grffile} % For spaces in paths
% \usepackage{etoolbox} % For spaces in paths
% \makeatletter % For spaces in paths
% \patchcmd\Gread@eps{\@inputcheck#1 }{\@inputcheck"#1"\relax}{}{}
% \makeatother
% 
\psscalebox{0.6 0.6}{
    \begin{pspicture}(0,-5.8)(12.3,2.2)
    \definecolor{colour0}{rgb}{0.9019608,0.9019608,0.9019608}
    \definecolor{colour1}{rgb}{0.9490196,0.9490196,0.9490196}
    \psellipse[linecolor=black, linewidth=0.04, dimen=outer](6.0,1.1)(3.1,1.1)
    \psdots[linecolor=black, dotsize=0.2](4.3,1.1)
    \psdots[linecolor=black, dotsize=0.2](5.8,1.1)
    \psdots[linecolor=black, dotsize=0.2](7.5,1.1)
    \rput[bl](4.4,0.7){1}
    \rput[bl](6.1,1.0){2}
    \rput[bl](7.6,0.7){3}
    \rput[bl](2.8,1.9){$\Omega$}
    % \psframe[linecolor=white, linewidth=0.04, fillstyle=gradient, gradlines=2000, gradbegin=colour0, gradend=colour1, dimen=outer](12.3,-1.1)(0.0,-5.8)
    \rput[bl](0.8,-2.3){$\sigma(\mathcal D_4)$}
    \psellipse[linecolor=black, linewidth=0.04, dimen=outer](3.85,-3.15)(1.55,0.65)
    \psdots[linecolor=black, dotsize=0.2](2.9,-3.1)
    \psdots[linecolor=black, dotsize=0.2](4.6,-3.1)
    \rput[bl](3.2,-3.2){2}
    \rput[bl](4.7,-3.5){3}
    \rput[bl](3.3,-4.2){$D_1$}
    \psellipse[linecolor=black, linewidth=0.04, dimen=outer](7.35,-3.15)(1.55,0.65)
    \psdots[linecolor=black, dotsize=0.2](7.2,-3.1)
    \rput[bl](7.5,-3.2){1}
    \rput[bl](6.8,-4.2){$D_2$}
    \psellipse[linecolor=black, linewidth=0.04, dimen=outer](6.3,-3.25)(5.3,1.55)
    \rput[bl](10.0,-3.3){$\emptyset$}
    \end{pspicture}
}

     
    \caption{集合代数}
    \label{fig:set-alg}
\end{wrapfigure}

如果考虑 $\mathscr{D}$ 中一切集合的并连同空集 $\varnothing$, 则得到的集系是代数, 称做 $\mathscr{D}$ 产生的代数, 记作 $\sigma(\mathscr{D})$. 于是, 代数 $\sigma(\mathscr{D})$ 的元素由空集 $\varnothing$ 与分割 $\mathscr{D}$ 之原子中集合的和组成.
这样, 如果 $\mathscr{D}$ 是 $\Omega$ 的某一分割, 则它与代数 $\mathscr{B}=\sigma(\mathscr{D})$ 一一对应. 如\cref{fig:set-alg}.



逆命题也正确. $\mathscr{B}$ 是有限空间 $\Omega$ 的子集的代数, 则存在唯一分割 $\mathscr{D}$, 其原子是代数 $\mathscr{B}$ 的元素, 并且 $\mathscr{B}=\sigma(\mathscr{D})$. 事实上, 假设集合 $\mathscr{D} \in \mathscr{B}$ 并且具有性质:对于任意 $B \in \mathscr{B}$, 集合 $D \cap B$ 要么与 $D$ 重合, 要么是空集. 那么, 这样集合 $D$ 的全体组成分割 $\mathscr{D}$ 并且具有所要求的性质 $B=\sigma(\mathscr{D})$. 



\begin{takeaway}
    概率是在样本空间上面定义的一个函数, 满足: 
    \begin{itemize}
        \item (1)非负性: 每个事件的概率必须大于等于0; 
        \item (2)规范性
        所有的事件概率``总和''等于1; 
        \item (3) 可加性: 互斥事件的概率可以直接相加.
    \end{itemize}
\end{takeaway}
    


\subsection{概率的加法公式}
我们已经在互斥的时候, 规定了其概率的过程. 下面我们来看一看不互斥的情形.
\begin{proposition}[加法公式]
    若两个事件$A,B$互斥, 则
    $$P(A\cup B)=P(A)+P(B).$$
\end{proposition}

\begin{remark}
    由加法公式可得到如下性质:
    \begin{itemize}
        \item 对任意事件$A$, 有
              $P(A)=1-P\left(\overline{A}\right).$
        \item 对任意两个事件$A,B$, 有
              $$P(A\cup B)=P(A)+P(B)-P(AB).$$
    \end{itemize}
\end{remark}

\begin{asidebox}
    我们实际上可以借此瞥见容斥原理.
    \begin{remark}
        若三个事件$A_1, A_2, A_3$两两互斥, 则
        $$\pmb P(A_1 \cup A_2 \cup A_3) = P(A_1)+P(A_2)+P(A_3).$$
        对任意三个事件$A_1, A_2, A_3$, 有
        \begin{align*}
            \pmb P(A_1 \cup A_2 \cup A_3) & \pmb=\color{blue}P(A_1)+P(A_2)+P(A_3)              \\
                                          & \phantom=\color{red}-P(A_1A_2)-P(A_1A_3)-P(A_2A_3) \\
                                          & \phantom=\color{blue}+P(A_1A_2A_3).
        \end{align*}%
    \end{remark}

    \begin{remark}
        更一般地, 可以使用容斥原理计算:
        若$n$个事件$A_1, A_2, \cdots, A_n$两两互斥, 则
        $$\pmb P\left( \bigcup_{i=1}^n A_i \right)=\sum_{i=1}^n P(A_i).$$
        \vspace{0.2in}
        对任意$n$个事件$A_1, A_2, \cdots, A_n$, 有
        $$\pmb P\left( \bigcup_{i=1}^n A_i \right)=\sum_{k=1}^n \left[ (-1)^{k+1} \sum_{1\le i_1\le \cdots\le i_k \le n} P(A_{i_1}\cdots A_{i_k}) \right].$$
    \end{remark}


\end{asidebox}

\subsection{古典概型模型}
\begin{definition}
    如果一个随机试验具有以下特点:
    \begin{itemize}%\setcounter{enumi}{2}
        \item 样本空间只含有限多个样本点; 
        \item 各样本点出现的可能性相等, 
    \end{itemize}
    则称此随机试验是古典型的. 此时对每个事件$A\subset \Omega$, 
    \begin{align*}
        P(A)=\frac{\mbox{事件$A$包含的样本点数}}{\mbox{样本点的总数}}=\frac{n(A)}{n(\Omega)}
    \end{align*}
    称为事件$A$的\textbf{古典概率}. 
\end{definition}

根据上述的定义, 我们可以立即得出$P(\emptyset)=0$, $P(\Omega)=1$. 






\subsection{几何概型}

\begin{definition}
    设样本空间为有限区域$\Omega$,若样本点落入$\Omega$内的任何区域$G$中的概率与区域$G$的测度成正比, 则样本点落入$G$内的概率为:
    $$
        p=\frac{\vert G\vert}{\vert \Omega \vert}
    $$
\end{definition}

我们可以先简单地把``测度''理解为面积. 并且我们发现, 如果$P(A)=0$, 那么$A$不一定是不可能事件. 比如正方形区域中的一个点, 一个点的面积为0. 因此正方形中选一个点的概率总为0. 






\section{条件概率}

\begin{definition}
    设$P(A)>0$, 称
    \begin{align*}
        P(B|A):=\frac{P(AB)}{P(A)}
    \end{align*}
    为在\textbf{事件$A$发生条件下, 事件$B$的条件概率}.(记作 $P(B \mid A)$ )
\end{definition}

比如, 在古典概率模型中, 
\begin{align*}
    P(B|A)=\frac{\mbox{事件$AB$包含的样本点数}}{\mbox{事件$A$包含的样本点数}}=\frac{n(AB)}{n(A)}.
\end{align*}

条件概率也是概率, 因此它也具有概率的性质:

$$
\begin{aligned}
& \mathbf{P}(A \mid A)=1 \\
& \mathbf{P}(\varnothing \mid A)=0 \\
& \mathbf{P}(B \mid A)=1, \quad B \supseteq A \\
& \mathbf{P}\left(B_1+B_2 \mid A\right)=\mathbf{P}\left(B_1 \mid A\right)+\mathbf{P}\left(B_2 \mid A\right), B_1,B_2\text{互斥}
\end{aligned}
$$

更抽象的, 我们有

\begin{proposition}
    设$P(A)>0$, 则
    \begin{itemize}
        \item 对任意事件$B$, 均有$P(B|A)\ge 0$;
        \item $P(\Omega|A)=1$;
        \item 若事件序列$\{B_n\}_{n\ge 1}$两两互斥, 则
              $$P\left( \left. \bigcup_{n=1}^{\infty} B_n \right| A\right)=\sum_{n=1}^{\infty} P(B_n|A)$$
              .
    \end{itemize}
\end{proposition}

由这些性质可见, 对于固定的事件 $A$, 在概率空间 $(\Omega \cap A, \mathscr{B} \cap A)$ 上的条件概率 $\mathbf{P}(\cdot \mid A)$, 以及在空间 $(\Omega, \mathscr{C})$ 上的概率 $\mathbf{P}(\cdot)$ 具有同样的性质, 其中
$$
\mathscr{A} \cap A=\{B \cap A: B \in \mathscr{B}\}
$$

\subsection*{全概率公式}

由于条件概率, 我们有概率的乘法公式:

\begin{theorem}[乘法公式]
    由条件概率的定义, 得到
    \begin{itemize}
        \item 如果$P(A)>0$, 则有$P(AB)=P(A)P(B|A)$.
        \item 如果$P(B)>0$, 则有$P(AB)=P(B)P(A|B)$.
    \end{itemize}
\end{theorem}

我们同样可以把这个性质推广到$n$个物品的时候.

\begin{corollary}
    如果$P(A_1 A_2\cdots A_{n-1})>0$, 则有\textbf{乘法公式}
    \begin{align*}
          & P(A_1A_2\cdots A_n)                                                \\
        = & P(A_1)P(A_2|A_1)P(A_3|A_1 A_2)\cdots P(A_n|A_1 A_2\cdots A_{n-1}).
    \end{align*}
\end{corollary}



另一个简单而重要的公式称做全概率公式, 是利用条件概率计算复合事件的基本工具. 我们首先希望对样本空间进行划分. 然后再进行求解:

\begin{definition}
    设$\Omega$为某试验的样本空间, $A_1, A_2, \cdots, A_n$为一组事件.如果以下条件成立:
    \begin{itemize}
        \item $A_1, A_2, \cdots, A_n$两两互斥, 
        \item $A_1 \cup A_2 \cup \cdots \cup A_n=\Omega$, 
    \end{itemize}
    则称$A_1, A_2, \cdots , A_n$为样本空间$\Omega$的一个\textbf{划分}.
    %或称$A_1, A_2, \cdots, A_n$为一个完备事件组.
\end{definition}


有了这样的分类之后, 我们就可以给出全概率了.

\begin{theorem}[全概率公式]
    如果$A_1, A_2, \cdots, A_n$ 是样本空间的划分, 且都有正概率, 则对任意事件$B$有
    \begin{align*}
        P(B)=\sum_{i=1}^n P(A_i) P(B|A_i).
    \end{align*}
\end{theorem}

\begin{proof}
    考虑基本事件空间 $\Omega$ 的某个分割 $\mathscr{D}=\left\{A_1, \cdots, A_n\right\}$, 且 $\mathbf{P}\left(A_i\right)>0(i=1$, $2, \cdots, n)$. (这样的分割又称做不相容事件的完全事件组.) 显然,
$$
B=B A_1+\cdots+B A_n,
$$

因此
$$
\mathbf{P}(B)=\sum_{i=1}^n \mathbf{P}\left(B A_i\right),
$$

其中
$$
\mathbf{P}\left(B A_i\right)=\mathbf{P}\left(B \mid A_i\right) \mathbf{P}\left(A_i\right)
$$
\end{proof}


\begin{example}
    假设要对研究生论文抄袭现象进行社会调查, 我们设计两个具有\textbf{相同答案}的问题:
    \begin{itemize}
        \item 你的生日是否在7月1日以前?
        \item 你做论文时是否有过抄袭行为?
    \end{itemize}
    同时提供给受访者一个放有等量红球和白球的袋子, 
    受访者在不被观察的情况下从袋子中随机取一个球观察颜色后放回.
    如果是红球回答第一个问题, 白球回答第二个问题.

    假定受访者有150人, 统计出共有60个回答“是”.问:有抄袭行为的比率是多少?
\end{example}

\begin{solution}
    事件$A$表示抽到白球, 事件$B$表示回答是, 则有
    $$P(B)=P(A)P(B|A)+P(\overline{A})P(B|\overline{A}).$$
    代入已知的概率, 得到
    $$\frac{60}{150}=\frac12\cdot P(B|A)+\frac12\cdot\frac12 $$
    求得
    $$P(B|A)=\frac{3}{10}$$
\end{solution}

\subsection{Bayes公式}
设事件 $A$ 和 $B$ 的概率大于 $0: \mathbf{P}(A)>0, \mathbf{P}(B)>0$, 利用乘法公式, 我们可以先看$B$而非$A$, 得到: 
$$\mathbf{P}(A B)=\mathbf{P}(A \mid B) \mathbf{P}(B).$$

和刚刚得到的$\mathbf{P}(A B)=\mathbf{P}(B \mid A) \mathbf{P}(A)$对比, 得到了$$\mathbf{P}(A \mid B)=\frac{\mathbf{P}(A) \mathbf{P}(B \mid A)}{\mathbf{P}(B)}.$$

\begin{theorem}[Bayes定理]
    设$0<P(A)<1$, $P(B)>0$, 则有
    $$P(A|B)=\frac{P(AB)}{P(B)}
        =\frac{P(A)P(B|A)}{P(A)P(B|A)+P(\overline{A})P(B|\overline{A})}$$
\end{theorem}

\begin{webaside}
    著名的科普视频频道主3Blue1Brown曾经对Bayes定律进行了可视化. 可以参考\href{https://www.bilibili.com/video/BV1R7411a76r}{Bilibili: BV1R7411a76r}.
\end{webaside}
假如事件组 $A_1, \cdots, A_n$ 是 $\Omega$ 的一个分割,那么有

\begin{corollary}
    如果$A_1, A_2, \cdots A_n$ 是样本空间的一个划分, 且都有正概率, 则对任意正概率的事件$B$有
    \[
        P(A_i|B)=\frac{P(A_i)P(B|A_i)}{P(A_1)P(B|A_1)+\cdots+P(A_n)P(B|A_n)}.%,\ i=1,\cdots,n
    \]
\end{corollary}

    实际上, 在统计应用中, 事件 $A_1, \cdots, A_n$ 组成事件组 $\left(A_1+\cdots+A_n=\Omega\right)$, 常称做 “假设” 或 “假说”, 而 $\mathbf{P}\left(A_i\right)$ 称做假设 $A_i$ 的\emph{先验}概率 $\left.\right)$. 条件概率 $\mathbf{P}\left(A_i \mid B\right)$ 称做假设 $A_i$ 在事件 $B$ 出现后的\emph{后验}概率. 


\begin{exercise}
    假设匣中有两枚硬币: $A_1$ 是一对称的硬币, “正面” $\mathrm{Z}$ 出现的概率等于 $1 / 2$, 而 $A_2$ 是一枚不对称的硬币, “正面” $\mathrm{Z}$ 出现的概率等于 $1 / 3$. 随意选出一枚硬币并将其投掷, 结果掷出正面. 问抽到硬币为对称硬币的概率如何?
\end{exercise}

\begin{solution}
    建立相应的概率模型. 这里自然取集合 $\Omega=\left\{A_1 \mathrm{Z}, A_1 \mathrm{~F}, A_2 \mathrm{Z}, A_2 \mathrm{~F}\right\}$, 可以描绘选取和投掷的结局, 其中 $A_1 \mathrm{Z}$ 表示 “选中硬币” $A_1$, 结果掷出正面 $\mathrm{Z}$, 等等, 而 $\mathrm{F}$ 表示硬币掷出反面. 根据条件, 所考虑结局的概率应该是:
$$
\mathbf{P}\left(A_1\right)=\mathbf{P}\left(A_2\right)=\frac{1}{2}
$$

和
$$
\mathbf{P}\left(\mathrm{Z} \mid A_1\right)=\frac{1}{2}, \quad \mathbf{P}\left(\mathrm{Z} \mid A_2\right)=\frac{1}{3} .
$$

这些条件唯一决定各结局的概率:
$$
\mathbf{P}\left(A_1 \mathrm{Z}\right)=\frac{1}{4}, \mathbf{P}\left(A_1 \mathrm{~F}\right)=\frac{1}{4}, \mathbf{P}\left(A_2 \mathrm{Z}\right)=\frac{1}{6}, \mathbf{P}\left(A_2 \mathrm{~F}\right)=\frac{1}{3}
$$

那么, 根据贝叶斯公式, 所求的概率为
$$
\mathbf{P}\left(A_1 \mid \mathrm{Z}\right)=\frac{\mathbf{P}\left(A_1\right) \mathbf{P}\left(\mathrm{Z} \mid A_1\right)}{\mathbf{P}\left(A_1\right) \mathbf{P}\left(\mathrm{Z} \mid A_1\right)+\mathbf{P}\left(A_2\right) \mathbf{P}\left(\mathrm{Z} \mid A_2\right)}=\frac{3}{5}
$$
\end{solution}

\begin{exercise}
    袋子中有10个白球, 5个黑球. 现掷一枚均匀的骰子. 掷出几点就从袋中取几个球. 若已知取出的球全为白球, 求掷出3点的概率. 
\end{exercise}

\begin{solution}
    原问题的意思是在取出的球全为白球的条件下, 掷出三点的概率. 设$B=\{\text{取出的球全是白球}\}$, $A=\{\text{掷出}i\text{点}\}(i=1,2,\cdots, 6)$.

    \begin{align*}
        P(A_3 | B) &= \frac{P(A_3)P(B|A_3)}{P(A_1)P(B|A_1)+P(A_2)P(B|A_2)+\cdots+P(A_6)P(B|A_6)} \\
        &= \frac{\frac16 \times \frac{\binom 53}{\binom {15}3}}{\sum_{i=1}^5 \frac16\times \frac{\binom 5i}{\binom{15}i}+\frac16\times 0}=0.4835
    \end{align*}
\end{solution}
\section{事件的独立性}

\begin{definition}
    若两事件$A$、$B$满足
    \begin{align*}
        P(AB)= P(A) P(B),
    \end{align*}
    则称\textbf{事件$A$、$B$相互独立}. %independent
\end{definition}

实际意义:若$P(B)>0$, 则上式等价于
\begin{align*}
    P(A|B)= P(A),
\end{align*}
即\textbf{事件$A$的概率不受事件$B$发生与否的影响}.  也就是事件$B$没有给我们任何的信息.

\begin{remark}
    ``两个事件互斥''和``两个事件相互独立''是不同的概念:
    \begin{itemize}
        \item 互斥 $\Rightarrow$ $P(A\cup B)=P(A)+P(B)$; 
        \item 独立 $\Rightarrow$ $P(AB)=P(A)P(B)$. 
    \end{itemize}
    但两者也有关系:如果$P(A)>0$且$P(B)>0$, 则两者不可能既是互斥的又是独立的. 
\end{remark}

我们接下来看多个事件的独立性:

\begin{definition}
    称$n(n\ge 2)$个事件$A_1, A_2, \cdots, A_n$相互独立, 如果对任意一组指标
    \begin{align*}
        1\le i_1<i_2< \cdots <i_k\le n\quad (k\ge 2)
    \end{align*}
    都有
    \begin{align*}
        P(A_{i_1}A_{i_2}\cdots A_{i_k})=P(A_{i_1})P(A_{i_2})\cdots  P(A_{i_k}).
    \end{align*}
\end{definition}

发现若$A$与$B$相互独立, 且$B$与$C$相互独立, 则$A$与$C$ \textbf{未必}相互独立. 
\begin{example}
    从全体有两个孩子的家庭中随机选择一个家庭, 并考虑下面三个事件:
    \begin{itemize}
        \item $A$为``第一个孩子是男孩'', 
        \item $B$为``两个孩子不同性别'', 
        \item $C$为``第一个孩子是女孩''. 
    \end{itemize}
    容易验证$A$与$B$相互独立, $B$与$C$相互独立, 但是$A$与$C$ \textbf{不独立}. 

    同样, 三个两两独立的也不一定都独立.

    伯恩斯坦四面体问题:一个正四面体有三面各涂上红, 白, 黑三种颜色. 第四面同时涂上三种颜色. 这四个面等概率出现在底面. 以A, B, C分别表示四面体底面出现红, 白, 黑三种颜色的事件. 问A, B, C是否相互独立?

    $$
        P(A)=P(B)=P(C)=2/4=1/2,
    $$
    $$
        P(AB)=P(BC)=P(AC)=1/4,
    $$
    $$
        P(ABC)=1/4\neq P(A)P(B)P(C).
    $$
\end{example}


\subsection{多知道一点: 朴素的Bayes分类器}
我们希望对事情做分类. 

我们有 $n$ 个训练示例的集合, 每个对象具有一系列特征: $\left(X_1, X_2, \cdots, X_m\right)$ , 对于每个 $X_i$ 都在一组特定区域取值. 
每个 $D_i$ 由表达其特征的一组值(方便起见写作向量形式)
$$
x_i=\left(x_{i 1}, x_{i 2}, \cdots, x_{i m}\right)
$$
对象可能的分类集合 $C=\left\{C_1, C_2, \cdots, C_t\right\} . C\left(D_i\right)$ 是 $D_i$ 的分类. 
形式化的说, 我们就有了下面的一组集合供我们训练. 
$$\left\{\left(D_1, C\left(D_1\right)\right),\left(D_2, C\left(D_2\right), \cdots,\left(D_n, C\left(D_n\right)\right)\right\}\right.$$

假设我们有一个足够大的训练集, 然后, 对于每个向量\( y=\left(y_1, \cdots, y_m\right) \)和每一个$c_j$, 我们完全可以使用训练集计算一个这样的经验条件概率: 具有特征向量$y$的对象被分类为$C_j$.

根据条件概率的定义, 我们知道$p_{y,j}$的计算公式
$$
P_{y, j}=\frac{P\left\{\left(\forall i, x_i=y_i\right) \wedge c\left(D_i\right)=c_i\right\}}{P\left\{\forall i, x_i=y_i\right\}}
$$

当一个具有特征$x^*$的对象过来的时候, 我们计算$P(c(D^*)=c_j | x^* =(x_1^*,\cdots, x_n^*))$的估计. 最后, 我们返回一个向量$(p_{x^*, 1},p_{x^*, 2},\cdots, p_{x^*, n})$,表示它被划分到各个类的概率大小.

这种方法的难处在于我们需要大量的准确的样本. 即使我们的特征只能取得0和1两个值, 如果有$m$个特征, 我们也要获得$2^m$个特征值与之对应. 

如果我们的这些特征相互独立, 我们就可以加快这个进程: 
$$
\begin{aligned}
P\left(c\left(D^*\right)=c_j \mid x^*\right) & =\frac{P\left(x^* \mid c\left(D^*\right)=c_j\right) \cdot P\left(c\left(D^*\right)=c_j\right)}{P\left(x^*\right)} \\
& =\frac{\prod_{k=1}^m P\left(x_k^*=x_i \mid c\left(D^*\right)=c_j\right) \cdot P\left(c\left(D^*\right)=c_j\right)}{P\left(x^*\right)} .
\end{aligned}
$$
其中$x_k^*$是$x^*$的第$k$个分量. 由于每个特征的可能数量一定, 在$m$个特征的时候, 我们只需要学习$O(m|C|)$的概率估计. 

训练过程很简单: 
\begin{itemize}
    \item 对于每一类$c_j$, 看一看被分类为$c_j$与总共的占比为多少, 并用此计算$$
    \hat{P}\left(c\left(D^*\right)=c_j\right)=\frac{\left|\left\{i \mid c\left(D_i\right)=c_j\right\}\right|}{|D|}
    $$这里$\hat P$表示我们算的是经验概率.
    \item 对于每一个特征$X_k$和对应的特征值$x_k$, 我们注意这个特征值$x_k$被分到$c_j$那一类的占比. 也就是$$
    \hat{P}\left(x_k^*=x_k \mid c\left(D^*\right)=c_j\right)=\frac{\left|\left\{i: x_k^i=x_k, c\left(D_i\right)=c_j\right\}\right|}{\left\{i \mid c\left(D_i\right)=c_j\right\} \mid} .
    $$
\end{itemize}

一旦我们训练好了分类器, 当一个具有特征向量$x^*=(x_1^*, \cdots, x_m^*)$新的对象$D_i^*$来了的时候, 对于每一个$c_j$, 我们就可以计算$$
\left(\prod_{k=1}^m \hat{P}\left(x_k^*=x_k \mid c\left(D^*\right)=c_j\right)\right) \hat{P}\left(c\left(D^*\right)=c_j\right)
$$
并且取最大值, 得到它最可能在的分类. 

总结一下, 我们的算法如\cref{algo:Bayes-class}所示. 

\begin{algorithm}
    \caption{朴素Bayes分类器算法}
    \label{algo:Bayes-class}
    \KwData{所有分类的集合$C$, 特征以及对应的特征值$F_1, F_2, \cdots, F_m$, 用于分类项目的训练集$D$}

    训练阶段
    \begin{itemize}
        \item [1] 对于每一类$c\in C$, 特征$k=1,2,\cdots, m$, 以及特征值$x_k\in F_k$, 计算$$
        \hat{P}\left(x_k^*=x_k \mid c\left(D^*\right)=c\right)=\frac{\left|\left\{i: x_k^i=x_k, c\left(D_i\right)=c\right\}\right|}{\left\{i : c\left(D_i\right)=c\right\} \mid} .
        $$
        \item [2] 对于每一类$c\in C$, 计算$$
        \hat{P}\left(c\left(D^*\right)=c\right)=\frac{\left|\left\{i : c\left(D_i\right)=c\right\}\right|}{|D|} .
        $$
    \end{itemize}

    给具有特征向量$x^*=x=\left(x_1, \ldots, x_m\right)$的新的对象$D^*$归一个类: 
    \begin{itemize}
        \item [1] 最有可能的分类: $$c\left(D^*\right)=\arg \max _{c_j \in C}\left(\prod_{k=1}^m \hat{P}\left(x_k^*=x_k \mid c\left(D^*\right)=c_j\right)\right) \hat{P}\left(c\left(D^*\right)=c_j\right) .$$
        \item [2] 计算可能的分类分布:$$\hat{P}\left(c\left(D^*\right)=c_j\right)=\frac{\left(\prod_{k=1}^m \hat{P}\left(x_k^*=x_k \mid c\left(D^*\right)=c_j\right)\right) \hat{P}\left(c\left(D^*\right)=c_j\right)}{\hat{P}\left(x^*=x\right)}$$
    \end{itemize}
\end{algorithm}


\part{一维随机变量}
\section{随机变量}

我们已经知道了一个实值函数是什么: 如果对于每一个实数 $x$ 又有唯一的 $y$ 与之对应, 
那么成为$y$ 为实变量$x$的函数. 这个定义可以推广到$x$不是实数的情形. 具体的, 如果
我们把这个想法定义在样本空间上, 这样的函数就是随机变量. 意味着

\begin{definition*}
    定义在样本空间上的函数就称为随机变量.
\end{definition*}

因为他们会为每一个样本点对应一个值, 在我们进行随机试验的时候, 这个值就好像是随机的一样. 

\begin{definition}[随机变量]
    设$\Omega$是某随机试验的样本空间.如果对于每个$\omega\in\Omega$,都有一个实数$X(\omega)$与其对应,这样就得到一个定义在$\Omega$上的函数
    \begin{align*}
    X=X(\omega),
    \end{align*}
    称该函数为随机变量(random variable).
\end{definition}
随机变量一般用大写英文字母$X$、$Y$、$Z$或小写希腊字母$\xi$、$\eta$、$\gamma$来表示.
\begin{itemize}
    \item 大写字母: 一个实验中的值
    \item 小写字母: 某个具体实验中的取值
\end{itemize}

按照研究的顺序, 我们现在先研究离散型(只能取有限个或者可列个值), 然后使用微积分的技巧
研究连续型(取得某一区间内的任何数值). 在这期间, 我们借助一些手段(如定义概率分布函数等)
从而可以研究混合型(一部分连续, 一部分离散)的随机变量. 

\section{离散型随机变量的概率分布}

\begin{definition}[概率分布]
    若离散型随机变量$X$的所有可能值为$\{x_k\}$,分别对应概率$\{p_k\}$,则称
$$P\{X=x_k\}=p_k, \quad k=1,2,\cdots$$
为$X$的概率分布.
\end{definition}

为了方便起见, 可以把概率列表以得到直观描述: 
\[\begin{array}{|c||c|c|c|c|c|}\hline
X & x_1 & x_2 & \cdots & x_k & \cdots \\ \hline
P & p_1 & p_2 & \cdots & p_k & \cdots \\ \hline
\end{array}\]

由于每一个概率是非负的, 并且我们遍历的整个概率空间, 因此概率分布具有如下的性质:
$\displaystyle \left\{ \begin{array}{l} p_k\ge 0, \quad k=1,2,\cdots \\
\sum\limits_{k}{p_k}=1 \end{array} \right.$

有时候我们对于小于某一个特定值的变量的取值内容感兴趣.概率分布函数就是统计小于某一个
特定值发生的概率. 
 同时这种定义也可以让我们从一个具体的
点之中脱身. 我们的视野就可以看得更广了, 这种技巧在连续的时候很有用处. 

\begin{definition}[概率分布函数]
    设离散型随机变量$X$的概率分布为
    \begin{align*}
    P\{X=x_k\}=p_k,\qquad k=1,2,\cdots.
    \end{align*}
    则定义$X$的分布函数为
    \begin{align*}
    F_X(x)=P\{X\le x\}=\sum_{x_k\le x} p_k,
    \end{align*}
    这里的和式是对所有满足$x_k\le x$的$p_k$求和.
    \end{definition}

    

\end{document}
