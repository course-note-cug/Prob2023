% !TEX root = main.tex

\usepackage{amsmath, amsthm, amssymb, amsfonts}
\usepackage{thmtools}
\usepackage{graphicx}
\usepackage{setspace}
\usepackage{geometry}
\usepackage{float}
\usepackage{amsthm}
\usepackage{hyperref}
% \usepackage{mathabx}
\usepackage[utf8]{inputenc}
\usepackage[english]{babel}
\usepackage{framed}
\usepackage[dvipsnames]{xcolor}
\usepackage[skins,breakable]{tcolorbox}
\usepackage{awesomebox}
\usepackage{mathrsfs}  
\usepackage{xcolor}
\usepackage{wrapfig}
\usepackage{algorithm2e}
\RestyleAlgo{ruled}

\usepackage{pstricks-add}
\usepackage{epsfig}
\usepackage{pst-grad} % For gradients
\usepackage{pst-plot} % For axes
\usepackage[space]{grffile} % For spaces in paths
\usepackage{etoolbox} % For spaces in paths
\makeatletter % For spaces in paths
\patchcmd\Gread@eps{\@inputcheck#1 }{\@inputcheck"#1"\relax}{}{}
\makeatother

\colorlet{shadecolor}{blue!10}

\colorlet{LightGray}{White!90!Periwinkle}
\colorlet{LightOrange}{Orange!15}
\colorlet{LightGreen}{Green!15}

\newcommand{\HRule}[1]{\rule{\linewidth}{#1}}

\declaretheoremstyle[name=定理]{thmsty}
\declaretheorem[style=thmsty,numberwithin=section]{theorem}
\tcolorboxenvironment{theorem}{colback=LightGray,breakable}

\declaretheoremstyle[name=命题,]{prosty}
\declaretheorem[style=prosty,numberlike=theorem]{proposition}
\tcolorboxenvironment{proposition}{colback=LightOrange,breakable}

\declaretheoremstyle[name=原则,]{prcpsty}
\declaretheorem[style=prcpsty,numberlike=theorem]{principle}
\tcolorboxenvironment{principle}{colback=LightGreen,breakable}

\declaretheoremstyle[name=定义,]{prcpsty}
\declaretheorem[style=prcpsty,numberlike=theorem]{definition}
\tcolorboxenvironment{definition}{colback=white,breakable}

\declaretheoremstyle[name=推论,]{prcpsty}
\declaretheorem[style=prcpsty,numberlike=theorem]{corollary}
\tcolorboxenvironment{corollary}{colback=white,breakable}

\newtheorem{example}{例子}[section]

\usepackage{xsim}
\loadxsimstyle{layouts} 
\DeclareExerciseEnvironmentTemplate{custom}{%
  \begin{tcolorbox}[boxrule = 0pt]
  \tcbox[on line,colback=teal,colframe=teal,coltext=white,size=small]{%
    \faToggleOn\sffamily\bfseries\
    \XSIMmixedcase{\GetExerciseName}
    \GetExerciseProperty{counter}%
  }\quad
}{\hfill\textbf{$\#$}\end{tcolorbox}}

\xsimsetup{
  exercise/within = part,
  exercise/template = margin,
  exercise/name = 练习
}


\newenvironment{remark}{%
  \par\medskip
  \noindent
  \textbf{注:}
}{%
  \par\medskip
}

\renewenvironment{solution}{%
  \par\medskip
  \noindent
  \textbf{解答:}
}{%
  \par\medskip
}

\newenvironment{solution*}{%
  \par\medskip
  \noindent 
  \color{gray}\small\textbf{提示或解答:}
}{%
  \par\medskip
}

\newenvironment{definition*}{%
  \par\medskip
  \noindent
  \textbf{定义:}
}{%
  \par\medskip
}

\newenvironment{lemma}{%
  \par\medskip
  \noindent
  \textbf{引理:}
}{%
  \par\medskip
}

\newenvironment{proposition*}{%
  \par\medskip
  \noindent
  \textbf{性质: }
}{%
  \par\medskip
}

\newtcolorbox{asidebox}{
  colback=gray!10,
  colframe=gray!60,
  fonttitle=\bfseries,
  title={Aside},
  breakable=true
}

\newtcolorbox{webaside}{
  colback=cyan!10,
  colframe=cyan!60,
  fonttitle=\bfseries,
  title={Web Demonstrate Aside},
  breakable=true
}

\usepackage{enumitem}

\setlist{nosep}

\setstretch{1.2}
\geometry{
    textheight=9in,
    textwidth=5.5in,
    top=1in,
    headheight=12pt,
    headsep=25pt,
    footskip=30pt
}

\usepackage{environ}
\usepackage[tikz]{bclogo}
\usepackage{tikz}
\usetikzlibrary{calc}
\NewEnviron{takeaway}
  {\par\medskip\noindent
  \begin{tikzpicture}
    \node[inner sep=0pt] (box) {\parbox[t]{.99\textwidth}{%
      \begin{minipage}{.3\textwidth}
      \centering\tikz[scale=5]\node[scale=3,rotate=30]{\bclampe};
      \end{minipage}%
      \begin{minipage}{.65\textwidth}
      \textbf{Takeaway Message}\par\smallskip
      \BODY
      \end{minipage}\hfill}%
    };
    \draw[red!75!black,line width=3pt] 
      ( $ (box.north east) + (-5pt,3pt) $ ) -- ( $ (box.north east) + (0,3pt) $ ) -- ( $ (box.south east) + (0,-3pt) $ ) -- + (-5pt,0);
    \draw[red!75!black,line width=3pt] 
      ( $ (box.north west) + (5pt,3pt) $ ) -- ( $ (box.north west) + (0,3pt) $ ) -- ( $ (box.south west) + (0,-3pt) $ ) -- + (5pt,0);
  \end{tikzpicture}\par\medskip%
}

\usepackage{marginnote}
\renewcommand*{\marginfont}{\color{gray}\ttfamily\small}
\usepackage{setspace}
\newcounter{paranum}[section]
\newcommand{\Par}[1]{\vspace{10pt}\noindent\textbf{\refstepcounter{paranum}\theparanum. }\textbf{#1}~~}
\newcommand{\lec}[1]{\reversemarginpar\marginnote{{\textbf{#1}}}}
\newcommand{\mn}[1]{\marginnote{{#1}}}
\renewcommand{\algorithmcfname}{算法}

\newcommand{\stirling}[2]{\left\{{#1 \atop #2}\right\}}
\newcommand{\binomt}[2]{\left(\left({#1 \atop #2}\right)\right)}
\newcommand{\pf}[4]{#1_{#2}^{#3_{#4}}}
\newcommand{\pl}[4]{#1_{#2}{#3^{#4}}}
\newcommand{\ty}[3]{{#1} \equiv {#2} ~(\bmod {#3})}
\newcommand{\Z}{{\mathbb Z}}
\newcommand{\one}{\mathbf{1}}
\newcommand{\varsub}[2]{\stackrel{#1}{\stackrel{\rule{#2}{0.4pt}}{\rule{#2}{0.4pt}}}}
\newcommand{\dd}{\mathrm{d}}
\newcommand{\Ep}[1]{\mathbb E\left(#1\right)}

\renewcommand{\red}[1]{{{\color{red}#1}}}
\newcommand{\teal}[1]{{{\color{teal}#1}}}
\renewcommand{\blue}[1]{{{\color{blue}#1}}}
\newcommand{\purple}[1]{{{\color{purple}#1}}}
\DeclareMathOperator{\var}{Var}
\newcommand{\E}{\mathbb E}
\newcommand{\like}{$\blacktriangleright$}
\newcommand{\exrate}[1]{{[#1]~}}
\newcommand{\newword}[2]{{\textbf{#1(#2)}\index{#1}}}
\newcommand{\newenword}[1]{{\textbf{#1}\index{#1}}}
\usepackage{ctex}
\usepackage{pifont}
\usepackage{cleveref}
% \usepackage[utf8]{inputenc}
\input{crefchn}
\begin{document}

\title{ \normalsize \textsc{}
		\\ [2.0cm]
		\HRule{1.5pt} \\
		\LARGE \textbf{{Lecture Notes}
		\HRule{2.0pt} \\ [0.6cm] \LARGE{2023秋 ~~~概率论与数理统计A} \vspace*{10\baselineskip}}
		}
\date{}
\author{\textbf{Author} \\ 
		AUGPath \\
		China University of Geosciences~(Wuhan) \\
		Department of Computer Science \\
		\today}

\maketitle
\newpage
\setcounter{tocdepth}{1}
\tableofcontents
\newpage


\part{概率论中基本的概念}
\section{概率中的基本概念}
\subsection{随机实验}

\begin{definition}
    随机试验是指:
    \begin{itemize}
        \item 试验结果不止一个, 但能明确所有的结果;
        \item 试验前不能预知出现哪种结果. 
    \end{itemize}
\end{definition}

\begin{definition}
    随机事件: 对随机试验的结果的陈述称为\textbf{随机事件}, 简称\textbf{事件}.
\end{definition}

\begin{definition}
    样本空间: 随机试验的每个可能结果$\omega$称为一个\textbf{样本点}. 全体样本点组成的集合$\Omega$称为\textbf{样本空间}.
    $$
        \Omega:=\{\omega\}.
    $$
    每个\textbf{随机事件}都可用样本空间的某个子集表示, 通常记为$A, B, C$等.
    在随机试验中, 当事件中的一个样本点出现时, 称该事件\textbf{发生}.
\end{definition}

\subsection{事件的关系与运算}

\begin{definition}[事件的关系]
    若事件$A$发生时, 事件$B$一定发生. 则称事件$A$包含于事件$B$(或事件$B$ \textbf{包含} $A$), 记作
    $$A\subset B \ (\text{或}B\supset A)$$

    对任意事件$A$, 有$\emptyset \subset A\subset \Omega$.

    若$A\subset B$, 且$B\subset A$, 则称事件$A$与$B$ \textbf{相等}, 记作$A=B$.

\end{definition}

下面来考察事件的运算:

\begin{definition}[事件的并]

    “事件$A$、$B$至少有一个发生”
    称为事件$A$与$B$的\textbf{和}或\textbf{并}(union), 记作
    $$A\cup B \ (\text{或}A+B)$$
    也就是
    $$A\cup B=\{\omega | \omega\in A \ \text{or}\ \omega\in B\}$$
\end{definition}

\begin{remark}
    使用数学归纳法, 事件的并可以推广到多个的情形:如$n$个事件的并
    $$\bigcup_{i=1}^{n} A_i =\text{“事件$A_1, \cdots, A_n$至少有一个发生”}$$
    可数个事件的并
    $$\bigcup_{i=1}^{\infty} A_i =\text{“事件$A_1, A_2, \cdots$至少有一个发生”}$$
\end{remark}

\begin{definition}
    “事件$A$、$B$同时发生”
    称为事件$A$与$B$的\textbf{积}或\textbf{交}(intersection), 记作
    $$AB \ (\text{或}A\cap B)$$
    也就是$$A\cap B=\{\omega | \omega\in A \ \text{and}\ \omega\in B\}$$
\end{definition}

\begin{remark}
    事件的交可以推广到多个的情形:如$n$个事件的交
    $$\bigcap_{i=1}^{n} A_i =\text{“事件$A_1, \cdots, A_n$全都发生”}$$
    可数个事件的交
    $$\bigcap_{i=1}^{\infty} A_i =\text{“事件$A_1, A_2, \cdots$全都发生”}$$
\end{remark}

\begin{definition}
    “事件$A$发生, 但$B$不发生”
    称为事件$A$与$B$的\textbf{差}, 记作
    $$A-B$$
    也就是
    $$A- B=\{\omega | \omega\in A \ \text{and}\ \omega\notin B\}$$
\end{definition}

像集合那样, 我们同样可以引入事件的关系:

\begin{definition}
    若$AB=\emptyset$, 则称$A$与$B$ \textbf{互斥}(或称$A$与$B$ \textbf{不相容}), %(mutually exclusive)
    即$A$与$B$不可能同时发生.
\end{definition}

\begin{definition}
    称$\Omega-A$为事件$A$的\textbf{对立}事件(或称$A$的\textbf{补}), 记为$\overline{A}$.
    它表示“事件$A$不发生”.
\end{definition}


像集合那样, 事件具有如下的运算规律:
\begin{itemize}
    \item 交换律
          \begin{itemize}
              \item $AB=BA$, $A\cup B=B\cup A$
          \end{itemize}
    \item 结合律
          \begin{itemize}
              \item $(AB)C=A(BC)$, $(A\cup B)\cup C=A\cup(B\cup C)$
          \end{itemize}
    \item 分配律
          \begin{itemize}
              \item $A(B\cup C)=AB\cup AC$, $A(B-C)=AB-AC$
          \end{itemize}
    \item 对偶律
          \begin{itemize}
              \item $\overline{AB}=\overline{A}\cup\overline{B}$, $\overline{A\cup B}=\overline{A}\,\overline{B}$
          \end{itemize}
\end{itemize}

下面来考察一些常见的化简运算关系的等式:

\begin{proposition}
    对任意两个事件$A$和$B$, 总有$ A-B=A-AB$.
\end{proposition}

\begin{proposition}
    事件$A$、$B$ \textbf{对立}当且仅当$A$、$B$\textbf{互斥}且$A\cup B=\Omega$.
\end{proposition}
\begin{example}
    设$A,B$为两个事件, 则有
    \begin{itemize}
        \item $A\overline{B}=A-B=A-AB$;
        \item $A=AB\cup A\overline{B}$.
    \end{itemize}
\end{example}

\begin{solution}
    用事件运算的分配律:
    \begin{itemize}
        \item $A\overline{B}=A(\Omega-B)=A\Omega-AB=A-AB$;
        \item $AB\cup A\overline{B}=A(B\cup\overline{B})=A\Omega=A$.
    \end{itemize}
\end{solution}

\begin{example}
    $A$, $B$, $C$ 表示事件
    \begin{itemize}
        \item $A$发生: $A$;
        \item 仅$A$发生: $A\cap \bar{B}\cap \bar{C}$;
        \item 恰有一个发生:$A \bar B \bar C\cup \bar AB\bar C\cup \bar A\bar BC$;
        \item 至少有一个发生:$A\cup B\cup C$;
        \item 至多有一个发生:$\bar A\bar B\bar C\cup A \bar B \bar C \cup \bar AB\bar C\cup \bar A\bar BC$;
        \item 都不发生:$\bar A\bar B\bar C$;
        \item 不全部发生: $\overline{ABC}=\bar A\cup \bar B\cup \bar C$.
    \end{itemize}
\end{example}

\begin{takeaway}
{
    可以使用集合描述事件, 离散数学中学过的集合的运算将允许我们对于事件进行化简和操作.
}
\end{takeaway}
\begin{shaded}
    % !TEX root = main.tex

\subsection*{多知道一点: 更多集合化的描述}

\paragraph{事件代数}

在离散数学中, 我们知道``代数'':
\begin{itemize}
    \item [1)] $\Omega \in \mathscr{A}$,
    \item    [2)] 若 $A \in \mathscr{A}, B \in \mathscr{A}$, 则集合 $A \cup B, A \cap B, A \backslash B$ 也都属于 $\mathscr{A}$.
    
\end{itemize}
考虑集合 $A \subseteq \Omega$ 的某个集合 $\mathscr{C}_0$, 则利用集合运算 $\cup, \cap$ 与 $\backslash$ 可以由 $\mathscr{A}_0$ 构造新集系, 其中元素也是事件. 给这些事件补充上必然事件 $\Omega$ 和不可能事件 $\varnothing$, 得集系 $\mathscr{A}$, 则 $\mathscr{A}$ 是代数. 

由以上的叙述, 可见作为事件系最好考虑本身是代数的集系. 以后, 我们正是考虑这样的集系.

\begin{example}
    1) $\mathscr{A}=\{\Omega, \varnothing\}$ 一 集系由 $\Omega$ 和空集 $\varnothing$ 构成, 称做平凡代数; 
    
    2) $\mathscr{A}=\{A, \bar{A}, \Omega, \varnothing\}$ 事件 $A$ 产生的集系;

    3) $\mathscr{A}=\{A: A \subseteq \Omega\}-\Omega$ 全部子集的集系 (包括空集 $\varnothing$ ).
\end{example}

\paragraph{分割} 我们称集合系
$$
\mathscr{D}=\left\{D_1, \cdots, D_n\right\}
$$

构成集合 $\Omega$ 的一个分割, 而 $D_1, \cdots, D_n$ 是该分割的原子, 如果 $D_1, \cdots, D_n$ 非空且两两不相容, 而它们的和等于 $\Omega$ :
$$
D_1+\cdots+D_n=\Omega .
$$

\end{shaded}
% !TEX root = main.tex
\section{事件的概率}

事件的\textbf{概率}:刻画试验中随机事件发生的\textbf{可能性大小}.

\subsection{概率的统计定义}
\begin{definition}
    设在$n$次试验中,事件$A$发生了$m$次,则称
    \begin{align*}
        f_n(A):=\frac{m}{n}
    \end{align*}
    为事件$A$发生的\textbf{频率}(frequency).
\end{definition}

\begin{definition*}%[概率的统计定义]
    在相同条件下重复进行的试验中,若随着试验次数$n$的增加,
    事件$A$发生的频率稳定在某一常数$p$附近,
    则称$p$为事件$A$的\textbf{概率},记作$P(A)=p$.
\end{definition*}
也就是概率是频率的稳定值. 实际应用中常将大量重复试验中事件的频率作为概率的近似估计.

\begin{proposition*}
    频率的性质:
    \begin{itemize}
        \item $0\le f_n(A)\le 1$;
        \item $f_n(\Omega)=1,\ f_n(\emptyset)=0$;
        \item 若事件$A_1, A_2, \cdots, A_k$两两互斥,则
              $$f_n \left( \bigcup_{i=1}^k A_i \right)=\sum_{i=1}^k f_n(A_i)$$
    \end{itemize}
\end{proposition*}

由于上述的性质, 我们给出概率的数学公理化定义:
\begin{definition}[概率的公理化定义]
    设$\Omega$是样本空间,定义概率空间$(\Omega,\mathcal{F},P)$。对每个事件$A\in \mathcal{F}$定义一个实数$P(A)$与之对应。
    集合函数$P$满足以下条件:
    \begin{itemize}
        \item 非负性:对任意事件$A$,均有$P(A)\ge 0$;
        \item 规范性:$P(\Omega)=1$;
        \item 可加性:若事件序列$\{A_n\}_{n\ge 1}$两两互斥,则
              $$P \left( \bigcup_{n=1}^{\infty} A_n \right)=\sum_{n=1}^{\infty} P(A_n)$$
    \end{itemize}
    则称$P(A)$为事件$A$的\textbf{概率}(probability).
\end{definition}

\begin{takeaway}
    概率是在样本空间上面定义的一个函数, 满足: (1)非负性: 每个事件的概率必须大于等于0; (2)规范性
    所有的事件概率``总和''等于1; (3) 可加性: 互斥事件的概率可以直接相加.
\end{takeaway}
    


\subsection{概率的加法公式}
我们已经在互斥的时候, 规定了其概率的过程. 下面我们来看一看不互斥的情形.
\begin{proposition}[加法公式]
    若两个事件$A,B$互斥,则
    $$P(A\cup B)=P(A)+P(B).$$
\end{proposition}

\begin{remark}
    由加法公式可得到如下性质:
    \begin{itemize}
        \item 对任意事件$A$,有
              $P(A)=1-P\left(\overline{A}\right).$
        \item 对任意两个事件$A,B$,有
              $$P(A\cup B)=P(A)+P(B)-P(AB).$$
    \end{itemize}
\end{remark}

\begin{asidebox}
    我们实际上可以借此瞥见容斥原理.
    \begin{remark}
        若三个事件$A_1, A_2, A_3$两两互斥,则
        $$\pmb P(A_1 \cup A_2 \cup A_3) = P(A_1)+P(A_2)+P(A_3).$$
        对任意三个事件$A_1, A_2, A_3$,有
        \begin{align*}
            \pmb P(A_1 \cup A_2 \cup A_3) & \pmb=\color{blue}P(A_1)+P(A_2)+P(A_3)              \\
                                          & \phantom=\color{red}-P(A_1A_2)-P(A_1A_3)-P(A_2A_3) \\
                                          & \phantom=\color{blue}+P(A_1A_2A_3).
        \end{align*}%
    \end{remark}

    \begin{remark}
        更一般地, 可以使用容斥原理计算:
        若$n$个事件$A_1, A_2, \cdots, A_n$两两互斥,则
        $$\pmb P\left( \bigcup_{i=1}^n A_i \right)=\sum_{i=1}^n P(A_i).$$
        \vspace{0.2in}
        对任意$n$个事件$A_1, A_2, \cdots, A_n$,有
        $$\pmb P\left( \bigcup_{i=1}^n A_i \right)=\sum_{k=1}^n \left[ (-1)^{k+1} \sum_{1\le i_1\le \cdots\le i_k \le n} P(A_{i_1}\cdots A_{i_k}) \right].$$
    \end{remark}


\end{asidebox}

\subsection{古典概型模型}
\begin{definition}
    如果一个随机试验具有以下特点:
    \begin{itemize}%\setcounter{enumi}{2}
        \item 样本空间只含有限多个样本点;
        \item 各样本点出现的可能性相等,
    \end{itemize}
    则称此随机试验是古典型的.此时对每个事件$A\subset \Omega$,
    \begin{align*}
        P(A)=\frac{\mbox{事件$A$包含的样本点数}}{\mbox{样本点的总数}}=\frac{n(A)}{n(\Omega)}
    \end{align*}
    称为事件$A$的\textbf{古典概率}.
\end{definition}

根据上述的定义, 我们可以立即得出$P(\emptyset)=0$,$P(\Omega)=1$.



\input{aside/2star-counting-techique.tex}

\subsection{几何概型}

\begin{definition}
    设样本空间为有限区域$\Omega$,若样本点落入$\Omega$内的任何区域$G$中的概率与区域$G$的测度成正比,则样本点落入$G$内的概率为:
    $$
        p=\frac{\vert G\vert}{\vert \Omega \vert}
    $$
\end{definition}

并且我们发现, $P(A)=0$, 则$A$不一定是不可能事件.





\begin{shaded}
    \input{aside/2star-counting-techique.tex}
\end{shaded}
\section{条件概率}

\begin{definition}
    设$P(A)>0$,称
    \begin{align*}
        P(B|A):=\frac{P(AB)}{P(A)}
    \end{align*}
    为在\textbf{事件$A$发生条件下,事件$B$的条件概率}.
\end{definition}

比如, 在古典概率模型中,
\begin{align*}
    P(B|A)=\frac{\mbox{事件$AB$包含的样本点数}}{\mbox{事件$A$包含的样本点数}}=\frac{n(AB)}{n(A)}.
\end{align*}

条件概率也是概率, 因此它也具有概率的性质: 

\begin{proposition}
    设$P(A)>0$,则
    \begin{itemize}
        \item 对任意事件$B$,均有$P(B|A)\ge 0$;
        \item $P(\Omega|A)=1$;
        \item 若事件序列$\{B_n\}_{n\ge 1}$两两互斥,则
              $$P\left( \left. \bigcup_{n=1}^{\infty} B_n \right| A\right)=\sum_{n=1}^{\infty} P(B_n|A)$$
    \end{itemize}
\end{proposition}

由于条件概率, 我们有概率的乘法公式: 

\begin{theorem}[乘法公式]
    由条件概率的定义,得到
    \begin{itemize}
        \item 如果$P(A)>0$,则有$P(AB)=P(A)P(B|A)$.
        \item 如果$P(B)>0$,则有$P(AB)=P(B)P(A|B)$.
    \end{itemize}
\end{theorem}

我们同样可以把这个性质推广到$n$个物品的时候. 

\begin{corollary}
    如果$P(A_1 A_2\cdots A_{n-1})>0$,则有\textbf{乘法公式}
    \begin{align*}
          & P(A_1A_2\cdots A_n)                                                \\
        = & P(A_1)P(A_2|A_1)P(A_3|A_1 A_2)\cdots P(A_n|A_1 A_2\cdots A_{n-1}).
    \end{align*}
\end{corollary}

我们希望把样本空间分类讨论. 

\begin{definition}
    设$\Omega$为某试验的样本空间,$A_1, A_2, \cdots, A_n$为一组事件.如果以下条件成立:
    \begin{itemize}
        \item $A_1, A_2, \cdots, A_n$两两互斥,
        \item $A_1 \cup A_2 \cup \cdots \cup A_n=\Omega$,
    \end{itemize}
    则称$A_1, A_2, \cdots , A_n$为样本空间$\Omega$的一个\textbf{划分}.
    %或称$A_1, A_2, \cdots, A_n$为一个完备事件组.
\end{definition}

有了这样的分类之后, 我们就可以给出全概率了. 

\begin{theorem}[全概率公式]
    如果$A_1, A_2, \cdots, A_n$ 是样本空间的划分,且都有正概率,则对任意事件$B$有
    \begin{align*}
        P(B)=\sum_{i=1}^n P(A_i) P(B|A_i).
    \end{align*}
\end{theorem}

\begin{example}
    假设要对研究生论文抄袭现象进行社会调查,我们设计两个具有\textbf{相同答案}的问题:
        \begin{itemize}
            \item 你的生日是否在7月1日以前?
            \item 你做论文时是否有过抄袭行为?
        \end{itemize}
        同时提供给受访者一个放有等量红球和白球的袋子,
        受访者在不被观察的情况下从袋子中随机取一个球观察颜色后放回.
        如果是红球回答第一个问题,白球回答第二个问题.

        假定受访者有150人,统计出共有60个回答“是”.问:有抄袭行为的比率是多少?
\end{example}

\begin{solution}
    事件$A$表示抽到白球,事件$B$表示回答是,则有
    $$P(B)=P(A)P(B|A)+P(\overline{A})P(B|\overline{A}).$$
    代入已知的概率,得到
    $$\frac{60}{150}=\frac12\cdot P(B|A)+\frac12\cdot\frac12 $$
    求得
    $$P(B|A)=\frac{3}{10}$$
\end{solution}

有时候我们希望把上面的和下面的反过来. 

\begin{theorem}
    设$0<P(A)<1$,$P(B)>0$,则有
    $$P(A|B)=\frac{P(AB)}{P(B)}
        =\frac{P(A)P(B|A)}{P(A)P(B|A)+P(\overline{A})P(B|\overline{A})}$$
\end{theorem}

同样的我们有推论: 

\begin{corollary}
    如果$A_1, A_2, \cdots A_n$ 是样本空间的一个划分,且都有正概率,则对任意正概率的事件$B$有
    \[
        P(A_i|B)=\frac{P(A_i)P(B|A_i)}{P(A_1)P(B|A_1)+\cdots+P(A_n)P(B|A_n)}.%,\ i=1,\cdots,n
    \]
\end{corollary}
\section{事件的独立性}

\begin{definition}
    若两事件$A$、$B$满足
    \begin{align*}
        P(AB)= P(A) P(B),
    \end{align*}
    则称\textbf{事件$A$、$B$相互独立}. %independent
\end{definition}

实际意义:若$P(B)>0$, 则上式等价于
\begin{align*}
    P(A|B)= P(A),
\end{align*}
即\textbf{事件$A$的概率不受事件$B$发生与否的影响}.  也就是事件$B$没有给我们任何的信息.

\begin{remark}
    ``两个事件互斥''和``两个事件相互独立''是不同的概念:
    \begin{itemize}
        \item 互斥 $\Rightarrow$ $P(A\cup B)=P(A)+P(B)$; 
        \item 独立 $\Rightarrow$ $P(AB)=P(A)P(B)$. 
    \end{itemize}
    但两者也有关系:如果$P(A)>0$且$P(B)>0$, 则两者不可能既是互斥的又是独立的. 
\end{remark}

我们接下来看多个事件的独立性:

\begin{definition}
    称$n(n\ge 2)$个事件$A_1, A_2, \cdots, A_n$相互独立, 如果对任意一组指标
    \begin{align*}
        1\le i_1<i_2< \cdots <i_k\le n\quad (k\ge 2)
    \end{align*}
    都有
    \begin{align*}
        P(A_{i_1}A_{i_2}\cdots A_{i_k})=P(A_{i_1})P(A_{i_2})\cdots  P(A_{i_k}).
    \end{align*}
\end{definition}

发现若$A$与$B$相互独立, 且$B$与$C$相互独立, 则$A$与$C$ \textbf{未必}相互独立. 
\begin{example}
    从全体有两个孩子的家庭中随机选择一个家庭, 并考虑下面三个事件:
    \begin{itemize}
        \item $A$为``第一个孩子是男孩'', 
        \item $B$为``两个孩子不同性别'', 
        \item $C$为``第一个孩子是女孩''. 
    \end{itemize}
    容易验证$A$与$B$相互独立, $B$与$C$相互独立, 但是$A$与$C$ \textbf{不独立}. 

    同样, 三个两两独立的也不一定都独立.

    伯恩斯坦四面体问题:一个正四面体有三面各涂上红, 白, 黑三种颜色. 第四面同时涂上三种颜色. 这四个面等概率出现在底面. 以A, B, C分别表示四面体底面出现红, 白, 黑三种颜色的事件. 问A, B, C是否相互独立?

    $$
        P(A)=P(B)=P(C)=2/4=1/2,
    $$
    $$
        P(AB)=P(BC)=P(AC)=1/4,
    $$
    $$
        P(ABC)=1/4\neq P(A)P(B)P(C).
    $$
\end{example}


\subsection{多知道一点: 朴素的Bayes分类器}
我们希望对事情做分类. 

我们有 $n$ 个训练示例的集合, 每个对象具有一系列特征: $\left(X_1, X_2, \cdots, X_m\right)$ , 对于每个 $X_i$ 都在一组特定区域取值. 
每个 $D_i$ 由表达其特征的一组值(方便起见写作向量形式)
$$
x_i=\left(x_{i 1}, x_{i 2}, \cdots, x_{i m}\right)
$$
对象可能的分类集合 $C=\left\{C_1, C_2, \cdots, C_t\right\} . C\left(D_i\right)$ 是 $D_i$ 的分类. 
形式化的说, 我们就有了下面的一组集合供我们训练. 
$$\left\{\left(D_1, C\left(D_1\right)\right),\left(D_2, C\left(D_2\right), \cdots,\left(D_n, C\left(D_n\right)\right)\right\}\right.$$

假设我们有一个足够大的训练集, 然后, 对于每个向量\( y=\left(y_1, \cdots, y_m\right) \)和每一个$c_j$, 我们完全可以使用训练集计算一个这样的经验条件概率: 具有特征向量$y$的对象被分类为$C_j$.

根据条件概率的定义, 我们知道$p_{y,j}$的计算公式
$$
P_{y, j}=\frac{P\left\{\left(\forall i, x_i=y_i\right) \wedge c\left(D_i\right)=c_i\right\}}{P\left\{\forall i, x_i=y_i\right\}}
$$

当一个具有特征$x^*$的对象过来的时候, 我们计算$P(c(D^*)=c_j | x^* =(x_1^*,\cdots, x_n^*))$的估计. 最后, 我们返回一个向量$(p_{x^*, 1},p_{x^*, 2},\cdots, p_{x^*, n})$,表示它被划分到各个类的概率大小.

这种方法的难处在于我们需要大量的准确的样本. 即使我们的特征只能取得0和1两个值, 如果有$m$个特征, 我们也要获得$2^m$个特征值与之对应. 

如果我们的这些特征相互独立, 我们就可以加快这个进程: 
$$
\begin{aligned}
P\left(c\left(D^*\right)=c_j \mid x^*\right) & =\frac{P\left(x^* \mid c\left(D^*\right)=c_j\right) \cdot P\left(c\left(D^*\right)=c_j\right)}{P\left(x^*\right)} \\
& =\frac{\prod_{k=1}^m P\left(x_k^*=x_i \mid c\left(D^*\right)=c_j\right) \cdot P\left(c\left(D^*\right)=c_j\right)}{P\left(x^*\right)} .
\end{aligned}
$$
其中$x_k^*$是$x^*$的第$k$个分量. 由于每个特征的可能数量一定, 在$m$个特征的时候, 我们只需要学习$O(m|C|)$的概率估计. 

训练过程很简单: 
\begin{itemize}
    \item 对于每一类$c_j$, 看一看被分类为$c_j$与总共的占比为多少, 并用此计算$$
    \hat{P}\left(c\left(D^*\right)=c_j\right)=\frac{\left|\left\{i \mid c\left(D_i\right)=c_j\right\}\right|}{|D|}
    $$这里$\hat P$表示我们算的是经验概率.
    \item 对于每一个特征$X_k$和对应的特征值$x_k$, 我们注意这个特征值$x_k$被分到$c_j$那一类的占比. 也就是$$
    \hat{P}\left(x_k^*=x_k \mid c\left(D^*\right)=c_j\right)=\frac{\left|\left\{i: x_k^i=x_k, c\left(D_i\right)=c_j\right\}\right|}{\left\{i \mid c\left(D_i\right)=c_j\right\} \mid} .
    $$
\end{itemize}

一旦我们训练好了分类器, 当一个具有特征向量$x^*=(x_1^*, \cdots, x_m^*)$新的对象$D_i^*$来了的时候, 对于每一个$c_j$, 我们就可以计算$$
\left(\prod_{k=1}^m \hat{P}\left(x_k^*=x_k \mid c\left(D^*\right)=c_j\right)\right) \hat{P}\left(c\left(D^*\right)=c_j\right)
$$
并且取最大值, 得到它最可能在的分类. 

总结一下, 我们的算法如\cref{algo:Bayes-class}所示. 

\begin{algorithm}
    \caption{朴素Bayes分类器算法}
    \label{algo:Bayes-class}
    \KwData{所有分类的集合$C$, 特征以及对应的特征值$F_1, F_2, \cdots, F_m$, 用于分类项目的训练集$D$}

    训练阶段
    \begin{itemize}
        \item [1] 对于每一类$c\in C$, 特征$k=1,2,\cdots, m$, 以及特征值$x_k\in F_k$, 计算$$
        \hat{P}\left(x_k^*=x_k \mid c\left(D^*\right)=c\right)=\frac{\left|\left\{i: x_k^i=x_k, c\left(D_i\right)=c\right\}\right|}{\left\{i : c\left(D_i\right)=c\right\} \mid} .
        $$
        \item [2] 对于每一类$c\in C$, 计算$$
        \hat{P}\left(c\left(D^*\right)=c\right)=\frac{\left|\left\{i : c\left(D_i\right)=c\right\}\right|}{|D|} .
        $$
    \end{itemize}

    给具有特征向量$x^*=x=\left(x_1, \ldots, x_m\right)$的新的对象$D^*$归一个类: 
    \begin{itemize}
        \item [1] 最有可能的分类: $$c\left(D^*\right)=\arg \max _{c_j \in C}\left(\prod_{k=1}^m \hat{P}\left(x_k^*=x_k \mid c\left(D^*\right)=c_j\right)\right) \hat{P}\left(c\left(D^*\right)=c_j\right) .$$
        \item [2] 计算可能的分类分布:$$\hat{P}\left(c\left(D^*\right)=c_j\right)=\frac{\left(\prod_{k=1}^m \hat{P}\left(x_k^*=x_k \mid c\left(D^*\right)=c_j\right)\right) \hat{P}\left(c\left(D^*\right)=c_j\right)}{\hat{P}\left(x^*=x\right)}$$
    \end{itemize}
\end{algorithm}

\begin{shaded}
    \subsection*{多知道一点: 事件独立性的表述}
在概率论中, 往往不但需要考虑事件 (集合) 的独立性, 而且需要研究事件 (集合) 组的独立性. 下面引进相应的定义.
\begin{definition*}
    称 $\Omega$ 子集系的代数 $\mathscr{A}_1$ 和 $\mathscr{A}_2$ (关于概率 $\mathbf{P}$ ) 为独立的或统计独立的,如果对于相应地属于 $\mathscr{A}_1$ 和 $\mathscr{A}_2$ 的两个任意子集 $A_1$ 和 $A_2$ 独立.
\end{definition*}


\end{shaded}

\part{一维随机变量}
\section{随机变量}

我们已经知道了一个实值函数是什么: 如果对于每一个实数 $x$ 又有唯一的 $y$ 与之对应,
那么成为$y$ 为实变量$x$的函数. 这个定义可以推广到$x$不是实数的情形. 具体的, 如果
我们把这个想法定义在样本空间上, 这样的函数就是随机变量. 意味着

\begin{definition*}
    定义在样本空间上的函数就称为随机变量.
\end{definition*}

因为他们会为每一个样本点对应一个值, 在我们进行随机试验的时候, 这个值就好像是随机的一样.

\begin{definition}[随机变量]
    设$\Omega$是某随机试验的样本空间.如果对于每个$\omega\in\Omega$, 都有一个实数$X(\omega)$与其对应, 这样就得到一个定义在$\Omega$上的函数
    \begin{align*}
        X=X(\omega),
    \end{align*}
    称该函数为随机变量(random variable).
\end{definition}

随机变量一般用大写英文字母$X$、$Y$、$Z$或小写希腊字母$\xi$、$\eta$、$\gamma$来表示.
\begin{itemize}
    \item 大写字母: 一个实验中的值
    \item 小写字母: 某个具体实验中的取值
\end{itemize}

按照研究的顺序, 我们现在先研究离散型(只能取有限个或者可列个值), 然后使用微积分的技巧
研究连续型(取得某一区间内的任何数值). 在这期间, 我们借助一些手段(如定义概率分布函数等)
从而可以研究混合型(一部分连续, 一部分离散)的随机变量.

\subsection{*独立的随机变量}

仿照事件的独立性的定义, 我们给出随机变量的独立性的定义. 同样, 这表示他们不彼此依赖. 

\begin{definition}
    设$\xi_1, \cdots, \xi_r$ 在 $\mathbb{R}^1$是一组  中 (有限) 集合 $X$ 上取值的随机称随机变量 $\xi_1, \cdots, \xi_r$ 为 (全体) 独立的, 如果对于任意 $x_1, \cdots, x_r \in X$,
$$
\mathbf{P}\left\{\xi_1=x_1, \cdots, \xi_r=x_r\right\}=\mathbf{P}\left\{\xi_1=x_1\right\} \cdots \mathbf{P}\left\{\xi_r=x_r\right\}
$$
\end{definition}

\begin{shaded}
    记 $\mathscr{X}$ 是 $X$ 中所有子集的代数, 上述的内容可以等价的写成: 对于任意 $B_1, \cdots, B_r \in \mathscr{X}$,
    $$
    \mathbf{P}\left\{\xi_1 \in B_1, \cdots, \xi_r \in B_r\right\}=\mathbf{P}\left\{\xi_1 \in B_1\right\} \cdots \mathbf{P}\left\{\xi_r \in B_r\right\}
    $$
\end{shaded}
\section{离散型随机变量的概率分布}
当实验者遇到描绘某些记载或读数的随机变量时, 则他关心的基本问题是, 该随机变量取各个数值的概率如何. 由此我们给出概率分布的定义:
\begin{definition}[概率分布]
    若离散型随机变量$X$的所有可能值为$\{x_k\}$,分别对应概率$\{p_k\}$,则称
    $$P\{X=x_k\}=p_k, \quad k=1,2,\cdots$$
    为$X$的概率分布(分布律).
\end{definition}

为了方便起见, 可以把概率列表以得到直观描述:
\[\begin{array}{|c||c|c|c|c|c|}\hline
        X & x_1 & x_2 & \cdots & x_k & \cdots \\ \hline
        P & p_1 & p_2 & \cdots & p_k & \cdots \\ \hline
    \end{array}\]
这张表我们称作随机变量$X$的分布列.

由于每一个概率是非负的, 并且我们遍历的整个概率空间, 因此概率分布具有如下的性质:
$\displaystyle \left\{ \begin{array}{l} p_k\ge 0, \quad k=1,2,\cdots \\
        \sum\limits_{k}{p_k}=1\end{array} \right.$

有时候我们对于小于某一个特定值的变量的取值内容感兴趣.概率分布函数就是统计小于某一个
特定值发生的概率.
同时这种定义也可以让我们从一个具体的
点之中脱身. 我们的视野就可以看得更广了, 这种技巧在连续的时候很有用处.

\begin{definition}[概率分布函数]
    设离散型随机变量$X$的概率分布为
    \begin{align*}
        P\{X=x_k\}=p_k,\qquad k=1,2,\cdots.
    \end{align*}
    则定义$X$的分布函数为
    \begin{align*}
        F_X(x)=P\{X\le x\}=\sum_{x_k\le x} p_k,
    \end{align*}
    这里的和式是对所有满足$x_k\le x$的$p_k$求和.
\end{definition}

注意, 这样的定义下, 函数是右连续的. 也就是在一个间断点$a$的时候, 
$\lim_{x\to a^+}f(x)=f(a)$.

有了这样的一番定义之后, 我们来看几个离散型随机变量.

\begin{takeaway}
    
    我们可以考察每个情况出现的概率. 

    概率分布 的前缀和是 概率分布函数, 概率分布函数的差分是概率分布. 
\end{takeaway}

\subsection{经典的离散型随机变量}

\paragraph{(一) 0-1分布(两点分布)}

\begin{example}
    $100$件产品中,有$98$件是正品,$2$件是次品,今从中随机地抽取一件,若规定
    \[X=\left\{ \begin{array}{ll} 1, & \text{取到正品;} \\
             0,      & \text{取到次品;}\end{array} \right.\]
    则随机变量$X$的概率分布表为
    $\begin{array}{|c|c|c|}\hline
            X & 0    & 1    \\ \hline
            P & 0.02 & 0.98 \\ \hline
        \end{array}$.
\end{example}

\begin{definition}[两点分布(0-1分布)]
    若随机变量$X$的概率分布为
    $\begin{array}{|c||c|c|}\hline
            X & 0   & 1 \\ \hline
            P & 1-p & p \\ \hline
        \end{array}$,
    则称$X$服从参数为$p$的两点分布(或 0-1分布).记为$$X\sim B(1,p)$$
\end{definition}

\paragraph{(二-$\varepsilon$) 几何分布}

\begin{example}
    抛一枚硬币, 硬币出现正面的概率为$p$, 请问前$k-1$次抛出反面, 第$k$次出现正面的概率.
\end{example}

\begin{definition}
    若随机变量$X$的概率分布为
    \[P\{X=k\}=p(1-p)^{k-1},\quad k=1,2,3\cdots,\]
    则称$X$服从参数为$p$的几何分布,记为$X\sim G(p)$.
\end{definition}

几何分布是接下来的例子要说的二项分布, 当次数恰好等于1时的特例.

\paragraph{(二) 二项分布}

\begin{example}
    若某射手每次射击命中的概率均为$p$,现进行$n$次独立射击,求恰有$k$次命中的概率.
\end{example}
% 需要 pifont 包
\def\1{\ding{51}} % 勾
\def\0{\ding{55}} % 叉
先研究射击次数$n=4$的特殊情形.此时有


{$$
    \begin{tabular}{|c|l|}
        \hline
        $k=0$ & \0\0\0\0                                                   \\
        \hline
        $k=1$ & \1\0\0\0, \0\1\0\0, \0\0\1\0, \0\0\0\1                     \\
        \hline
        $k=2$ & \1\1\0\0, \1\0\1\0, \1\0\0\1, \0\1\1\0, \0\1\0\1, \0\0\1\1 \\
        \hline
        $k=3$ & \1\1\1\0, \1\1\0\1, \1\0\1\1, \0\1\1\1                     \\
        \hline
        $k=4$ & \1\1\1\1                                                   \\
        \hline
    \end{tabular}
$$}

\begin{definition}[$n$重Bernoulli实验]
    只有两种可能结果的试验称为Bernoulli试验.将一Bernoulli试验独立重复$n$次称为$n$重Bernoulli试验.
\end{definition}
        
我们发现, 上面的问题就是$k$重Bernoulli实验恰好成功$k$次的概率. 根据计数的技巧, 我们发上发现:

\begin{theorem}[Bernoulli定理]
    设一次试验中事件$A$发生的概率为$p(0<p<1)$,则$n$重Bernoulli试验中,事件$A$恰好发生$k(0\le k\le n)$ 次的概率为
    \begin{align*}
        b(k;n,p)={n\choose k} p^k (1-p)^{n-k}.
    \end{align*}
\end{theorem}

对二项分布$B(n,p)$,当$n$充分大、$p$很小时,(但是保证$np_n=\lambda$不变)形成的函数曲线是什么?
\begin{align*}
                   & {n \choose k}p_n^k(1-p_n)^{n-k}                                                                                                                                                    \\
    ={}            & \frac{n(n-1)\cdots (n-k+1)}{k!}(\frac{\lambda}{n})^k(1-\frac{\lambda}{n})^{n-k}                                                                                                    \\
    ={}            & \frac{\lambda^k}{k!}\textcolor{teal}{\Big(1\cdot (1-\frac{1}{n})\cdots (1-\frac{k-1}{n})\Big)}\textcolor{red}{(1-\frac{\lambda}{n})^n}\textcolor{blue}{(1-\frac{\lambda}{n})^{-k}} \\
    \rightarrow {} & \frac{\lambda^k}{k!}\cdot \textcolor{teal}{1}\cdot \textcolor{red}{e^{-\lambda}}\cdot \textcolor{blue}{1}
\end{align*}

因此我们发现, 当$n\to\infty$的时候, 有
\begin{theorem}
    设$\lambda>0$为一个常数,$n$为任意正整数,且$np_n=\lambda$,则对任意一个\textcolor{blue}{固定的}非负整数$k$,
    $$
        \lim_{n\rightarrow \infty}{n\choose k}p_n^k(1-p_n)^{n-k}=\frac{\lambda^k}{k!}e^{-\lambda}.
    $$
\end{theorem}
这就是Poisson分布.

\paragraph{(三) Poisson分布}

Poisson分布的定义如下.
\begin{definition}[Poisson分布]
    如果随机变量$X$服从以下分布律
    \[ P\{X=k\}=\frac{\lambda^k}{k!}e^{-\lambda},\quad k=0,1,\cdots \]
    其中$\lambda>0$,则称$X$服从参数为$\lambda$的Poisson分布,记为
    $X\sim P(\lambda).$
\end{definition}%
Poisson分布常与单位时间(或单位面积、单位产品等)上的计数过程相联系.

\begin{takeaway}
    三个重要离散型随机变量的分布:
    \begin{itemize}
        \item 几何分布$$X\sim G(p): P\{X=k\}=p(1-p)^{k-1}, \quad k=1,2,3 \cdots$$
        \item 二项分布$$X\sim b(k ; n): P\{X=p\}=\binom nk p^k(1-p)^{n-k}$$
        \item Poisson分布: $$X\sim P(\lambda): P\{X=k\}=\frac{\lambda^k}{k !} e^{-\lambda}, \quad k=0,1, \cdots$$
    \end{itemize}
\end{takeaway}
\section{连续型随机变量}
连续型随机变量 $X$ 的取值范围为一个区间.此时$X$取某个确定值的概率总是等于零.我们发现
事件$X=a$可能发生,但$P\{X=a\}=0$.

这就启发我们使用极限的想法来考虑之. $X$在某个小区间内取值的概率可以大于零.
将区间分为若干段,研究$X$在各小区间取值的概率.分组的频率直方图刻画随机变量的概率分布,
分组越细,频率直方图就越接近一条连续曲线.取极限, 就有:

\begin{definition}{概率密度函数}
    如果存在非负函数$f(x)$,满足
    $$
        P(X\leq x)=\int_{-\infty}^{x}f(t)dt,
    $$
    则称$X$为连续型随机变量,称$f(x)$为$X$的概率密度函数(pdf, probability density function).
\end{definition}

和概率分布函数一样, 具有如下的性质:
$$
    (1)\,f(x)\ge 0,\quad (2)\ \int_{-\infty}^{+\infty}f(x)dx=1.
$$

这样确实可以来表述问题. 但是对于一半离散, 一半连续的随机变量, 该怎么统一起来?
实际上, 我们希望``忽略''在转折点上面带来的影响. 这样, 我们可以使用上面的分布函数的
思想进行.

\begin{definition}
    对任意随机变量$X$(离散或连续),称函数
    $$F_X(x):=P\{X\le x\},\quad x\in \mathbb{R}$$
    为$X$的分布函数.
\end{definition}
这是一个对于概率密度函数(或者概率分布)的累计, 自然, 它应该满足如下的性质:

\begin{itemize}
    \item 对任意实数$a<b$,总有$F(a)\le F(b)$;
    \item $0 \le F(x) \le 1$;
    \item $F(-\infty)=0$,$F(+\infty)=1$.
\end{itemize}
分别对应着: 概率不可能为负数; 所有的概率必须在0和1之间; 并且所有的可能性加起来必须等于1(整个$P(\Omega)$的).
\begin{takeaway}
    我们可以考察每个小区间出现的概率.

    概率密度函数 的积分是 概率分布函数, 概率分布函数求导得到概率密度函数.

    不论是离散型的或非离散型的随机变量 $X$, 都可以借助分布函数$F(x)=P\{X \leqslant x\}, \quad-\infty<x<\infty$
    来描述.
\end{takeaway}
\subsection{常见的连续性随机变量}

\paragraph{(一) 均匀分布}

这可能是最基本的一个分布, 这个分布在一个区间内每个点的取值相同:

\begin{definition}
    若随机变量$X$有概率密度
    \begin{align*}
        f(x)=\left\{\begin{array}{lcl}
                        \frac1{b-a} &  & x\in [a,b]       \\
                        0           &  & \mbox{otherwise}
                    \end{array}\right. ,\qquad (a<b)
    \end{align*}
    则称$X$服从区间$[a,b]$上的均匀分布:$X\sim U(a,b)$.
\end{definition}%


\paragraph{(二) 指数分布}

假设某医院平均每小时出生$\lambda$个婴儿, 也就是说
\begin{itemize}
    \item $1$小时内出生婴儿个数$N(1)$服从Poisson分布$P(\lambda)$.
    \item $t$小时内出生婴儿个数$N(t)$服从Poisson分布$P(\lambda t)$.
\end{itemize}
我们希望研究婴儿出生的时间间隔$X$的概率分布.


当$t<0$时,有$F(t)=P\{X\le t\}=0$. 当$t\ge0$时,
\begin{align*}
    F(t) & =P\{X\le t\}=1-P\{X>t\}          \\
         & =1-P\{N(t)=0\}=1-e^{-\lambda t}.
\end{align*}

因此,当$t<0$时,$f(t)=0$; 当$t\ge0$时,$f(t)=\lambda e^{-\lambda t}$.

基于此, 我们给出指数分布的描述:
\begin{definition}
    如果随机变量$X$有以下概率密度
    \begin{align*}
        f(x)=\left\{\begin{array}{lcl}
                        \lambda e^{-\lambda x} &  & x\ge 0           \\
                        0                      &  & \mbox{otherwise}
                    \end{array}\right. ,
    \end{align*}
    其中$\lambda>0$,则称$X$服从参数为$\lambda$的指数分布,记为
    \[\pmb X\sim \text{Exp}(\lambda). \]
    其分布函数为
    \begin{align*}
        F(x)=\left\{\begin{array}{lcl}
                        1- e^{-\lambda x} &  & x\ge 0 \\
                        0                 &  & x<0
                    \end{array}\right. .
    \end{align*}
\end{definition}

指数分布经常作为时间间隔或等待时间的分布. 其有一个非常重要的特性: 无记忆性.

\begin{proposition}[指数分布的无记忆性]
    随机变量$X$服从参数为$\lambda$的指数分布.设$s,t>0$,即有
    \[ P\{X>s+t|X>s\}=P\{X>t\}.\]
\end{proposition}

\begin{proof}
    证明无记忆性, 只要证明 $\forall s, t \geq 0$, 有 $P\{X \leq s\}=P\{X \leq s+t \mid X \geq t\}$容易计算, $P\{X \leq s+t \mid X \geq t\}=\frac{P\{t<X \leq s+t\}}{P\{X \geq t\}}=\frac{\int_1^{s+t} \lambda e^{-\lambda x} d x}{\int_t^{+\infty} \lambda e^{-\lambda x} d x}=\frac{e^{-\lambda t}-e^{-\lambda(s+t)}}{e^{-\lambda t}}=1-e^{-\lambda s}$ $P\{X \leq s\}=\int_0^s \lambda e^{-\lambda x} d x=1-e^{-\lambda s}$
\end{proof}

\paragraph{(三) 正态分布}

在《高等数学 II》中我们使用极坐标的换元法了解到了如下的事实:
$$
    \int_{-\infty}^{+\infty}e^{-x^2} dx =\sqrt \pi
$$

实际上, 这个性质比较好的函数稍加改造便可得到一个非常重要的分布函数 --
正态分布函数. 我们在后面会发现, 当样本总体是大量的随机变量求和的时候,
分布将不可避免地趋向于正态分布.

\begin{definition}
    如果随机变量$X$有以下概率密度
    \begin{align*}
        f(x)=\frac1{\sqrt{2\pi}\cdot\sigma}e^{-\frac{(x-\mu)^2}{2\sigma^2}},
    \end{align*}
    其中$\mu,\sigma$为常数且$\sigma>0$,则称$X$服从正态分布,简记为%(normal distribution)
    \begin{align*}
        X \sim N(\mu,\sigma^2).
    \end{align*}
    称$N(0,1)$为标准正态分布.
\end{definition}

带入数据, 我们发现标准正态分布的概率密度函数(PDF)为
\begin{align*}
    \varphi(x)=\frac1{\sqrt{2\pi}}e^{-\frac{x^2}{2}},
\end{align*}
前面奇怪的系数是保证它在做积分的时候满足规范化的条件.

在这个问题中, 其分布函数
\begin{align*}
    \Phi(x)=\frac1{\sqrt{2\pi}}\int_{-\infty}^xe^{-\frac{t^2}{2}}dt.
\end{align*}
是没有显式的积分结果的. 通常这个结果需要查表得到.

如果发现一个分布不是标准正态分布, 可以用线性变换把它变成标准正态分布.

\begin{proposition}
    随机变量$X\sim N(\mu,\sigma^2)$, 则 $Z=\frac{X-\mu}{\sigma}\sim N(0,1)$.
\end{proposition}


\begin{proof}
    { $Z=\frac{X-\mu}{\sigma}$ 的分布函数为$$
            \begin{aligned}
                P\{Z \leqslant x\} & =P\left\{{(X-\mu)}/{\sigma} \leqslant x\right\}=P\{X \leqslant \mu+\sigma x\}                                         \\
                                   & =\frac{1}{\sqrt{2 \pi} \sigma} \int_{-\infty}^{\mu+\alpha t} \mathrm{e}^{-\frac{(t-\mu)^2}{2 \sigma^2}} \mathrm{~d} t
            \end{aligned}
        $$
        做变量代换, 令 $\frac{t-\mu}{\sigma}:=u$, 得$$
            \small P\{Z \leqslant x\}=\frac{1}{\sqrt{2 \pi}} \int_{-\infty}^x \mathrm{e}^{-u^2 / 2} \mathrm{~d} u=\Phi(x),
        $$
        由此知 $Z=\frac{X-\mu}{\sigma} \sim N(0,1)$.}
\end{proof}

\begin{takeaway}
    常见的连续性概率分布: 
    \begin{itemize}
        \item 均匀分布: $X\sim U(a, b): $的PDF是$$
        f(x)=\left\{\begin{array}{ll}
        \frac{1}{b-a} & x \in[a, b] \\
        0 & \text { otherwise }
        \end{array}, \quad(a<b)\right.
        $$
        \item 指数分布${X} \sim \operatorname{Exp}(\lambda)$的PDF是$$
        f(x)= \begin{cases}\lambda e^{-\lambda x} & x \geq 0 \\ 0 & \text { otherwise }\end{cases}
        $$
        \item 正态分布$X\sim N(\mu,\sigma)$的PDF是$$
        f(x)=\frac{1}{\sqrt{2 \pi} \cdot \sigma} e^{-\frac{(x-\mu)^2}{2 \sigma^2}}
        $$
    \end{itemize}
\end{takeaway}
\section{随机变量的函数的分布}

在物理实验中, 我们希望测量直径$d$, 从而得到圆的面积$S={1\over4} \pi d^2$. 
但是测量是有误差的. 这时候我们发现, 随机变量$S$是随机变量$d$的函数. 记得
随机变量是定义在样本空间上的函数; 而这样的一个样本空间上面的函数又可以定义
另一个函数. 这就是我们做抽象的好处. 

更一般地, 我们希望找到如下问题的一类解法: 
\begin{itemize}
    \item 已知的随机变量 $X$ 的概率分布
    \item 去求得它的函数 $Y=g(X)$ ($g$连续, 且已知)
\end{itemize}

首先来看一个离散情形下的例子. 这里我们主要演示这件事情主要在做什么事情. 

\begin{example}
  设随机变量 $X$ 具有以下的分布律, 试求 $Y=(X-1)^2$ 的分布律.
  $$
\begin{array}{c|cccc}
X & -1 & 0 & 1 & 2 \\
\hline p_k & 0.2 & 0.3 & 0.1 & 0.4
\end{array}
$$
\end{example}

\begin{solution}
  $Y$ 所有可能取的值为 $0,1,4$. 由
$$
\begin{aligned}
& P\{Y=0\}=P\left\{(X-1)^2=0\right\}=P\{X=1\}=0.1, \\
& P\{Y=1\}=P\{X=0\}+P\{X=2\}=0.7, \\
& P\{Y=4\}=P\{X=-1\}=0.2,
\end{aligned}
$$
所以$Y$的分布律为: 
$$
\begin{aligned}
&\begin{array}{c|ccc}
Y & 0 & 1 & 4 \\
\hline p_k & 0.1 & 0.7 & 0.2
\end{array}
\end{aligned}
$$
\end{solution}

再一个简单的例子, 有别于离散的情况可以一个点一个点的考虑, 我们可以使用分布函数
这样的累计量来求积分得到.  

\begin{example}
  设随机变量为 $X$ $\sim$ $N (0, 1)$, 求$Y = e^{X}$的概率密度函数. 
  ($N(0,1)$意味着$X\sim \frac1{\sqrt{2\pi}}e^{-x^2/2}$)
\end{example}
\begin{solution}
  当$y \leq 0$时,
  \[F_Y(y) = P\{Y \leq y\} = 0\]
  \quad 故,
  \[f_Y(y) = 0\]
  
  
  \quad 当$y > 0$时,
  \[F_Y(y) = P\{Y \leq y\} = P\{ e^{X} \leq y\} = P\{X \leq \ln^{y}\}= F_X(\ln^{y})\]
  
  \quad 对两端对$x$求导, 得到
  \[f_Y(y) = f_X(\ln^{y})(\ln^{y})' = \frac{1}{y}\frac{1}{\sqrt{2\pi}}e^{-\frac{1}{2}(\ln^y)^2}\]

\end{solution}

我们发现在这种情形下, 有我们可以用如下的步骤清楚地表述我们所做的事情: 

\begin{theorem}[两个随机变量的联系(关系是严格单调函数)]
  设随机变量X具有概率密度$f_X(x),-\infty < x < \infty$,又设函数g(x)处处可导且恒有$g'(x)>0$(或恒有$g'(x)<0$),则$Y = g(X)$是连续型随机变量, 其概率密度为
  \begin{equation*}
      f_Y(y)=
      \begin{cases}
          \displaystyle f_X[h(x)]\mid h'(x)\mid,\,\, &\alpha<y<\beta,\\
          \displaystyle 0,\,\, &\text{其他},
      \end{cases}
  \end{equation*}
  其中$\alpha = \min\{g(-\infty),g(+\infty)\},\beta = \max\{g(-\infty),g(+\infty)\}$,$h(y)$是$g(x)$的反函数.
\end{theorem}

\begin{proof}
  我们只证$g'(x)>0$的情况, 此时$g(x)$在$(-\infty,\infty)$内严格单调增加, 它的反函数$h(y)$存在, 且在$(\alpha,\beta)$内严格单调增加、可导.
    
分别记X,Y的分布函数为$F_X(x),F_Y(y)$.现在先来求$Y$的分布函数$F_Y(y)$.
 
因为$Y = g(X)$在$(\alpha,\beta)$内取值, 故当$y \leq \alpha$时, $F_Y(y) = P\{Y \leq y\} = 0$;当$y \geq \beta$时, $F_Y(y) = P\{Y \leq y\} = 1$.
 
 当$\alpha < y < \beta$时, 
 $$
 \begin{aligned}
     \small F_Y(y) &= P\{Y \leq y\} = P\{g(X) \leq y\}\\
     \small &= P\{X \leq h(y)\} = F_X[h(y)]
 \end{aligned}
$$
将$F_Y(y)$关于$y$求导数, 即得$Y$的概率密度.

\begin{align}
  f_Y(y) =
  \begin{cases}
      \displaystyle f_X[h(y)]h'(y),\,\,  &\alpha < y < \beta,\\
      \displaystyle 0,\,\,  &\text{其他}.
  \end{cases}
\end{align}
对于$g'(x)<0$的情况可以同样地证明, 此时有
\begin{align}
  \begin{cases}
      f_Y(y)=
      \displaystyle f_X[h(y)][-h'(y)],\,\, &\alpha < y < \beta,\\
      \displaystyle 0,\,\, &\text{其他}.
  \end{cases}
\end{align}
合并(1)(2)式可证定理成立.
\end{proof}

那么, 在不单调的情况下, 我们应该怎么办呢? 实际上, 我们应该分类讨论, 一个一个的来看. 如下
所示. 

\begin{example}
  设随机变量为 $X$ $\sim$ $N (0, 1)$,$Y = 2X^2+1$的概率密度函数.
\end{example}

\begin{solution}
  当$y < 1$时,
    \[ F_Y(y) = P\{Y \leq y \} = 0,f_Y(y) = 0\]

    \quad 当$y \geq 1$时, 
    \begin{align*}
    F_Y(y) & =  P\{Y \leq y\}=P\left\{2 X^2+1 \leq y\right\} \\
    & =P\left\{-\sqrt{\frac{y-1}{2}} \leq X\right.  \left.\leq \sqrt{\frac{y-1}{2}}\right\}\\
    &=\Phi\left(\sqrt{\frac{y-1}{2}}\right)-\Phi\left(-\sqrt{\frac{y-1}{2}}\right) \\
    & =2 \Phi\left(\sqrt{\frac{y-1}{2}}\right)-1\\
    f_Y(y) &= \frac{\mathrm{d}[2\Phi(\sqrt{\frac{y-1}{2}}) - 1]}{\mathrm{d}y} = \frac{1}{2\sqrt{\pi(y-1)}}e^{\frac{-(y-1)}{4}}
    \end{align*}
\end{solution}

\begin{example}
  设随机变量 $X$ 在区间 $(0,1)$ 上服从均匀分布, 求 $Y=\frac{1}{1+X}$ 的概率密度.
\end{example}
\begin{solution}
  $$
\begin{aligned}
F_Y(y) & =P\{Y \leq y\}=P\left\{\frac{1}{X+1} \leq y\right\} \\
& =P\left\{X \geq \frac{1-y}{y}\right\}=1-P\left\{X \leq \frac{1-y}{y}\right\} \\
& =1-F_X\left(\frac{1-y}{y}\right)
\end{aligned}
$$

对上式两边求导得:
$$
f_Y(y)=-f_X\left(\frac{1-y}{y}\right)\left(\frac{1-y}{y}\right)^{\prime}= \begin{cases}\frac{1}{y^2}, & \frac{1}{2}<y<1, \\ 0, & \text { 其他 }\end{cases}
$$
\end{solution}


\part{多维随机变量及其分布}
\section{二维随机变量及其分布函数}

有时候样本空间不一定受一维量的影响. 例如研究学龄前的儿童发育情况, 对这一地区的儿童进行抽查.
比如观察到: 身高$H$, 体重$W$.样本空间: $S=\{e\}=\{$ 某地区的全部学龄前儿童 $\}$, 那么$H(e)$ 和 $W(e)$ 是 定义在 $S$ 上的两个随机变量.
又如观察炮弹着陆点, 是由横坐标$x$, 纵坐标$y$确定的. 那么, 对于二维的随机变量, 我们应该如何分析? 

\begin{definition}[二维随机变量]
  一般地, 设 $E$ 是一个随机试验, 它的样本空间是 $S=\{e\}$, 设 $X=X(e)$ 和 $Y=Y(e)$ 是定义在 $S$上的随机变量, 由它们构成的一个向量 $(X, Y)$,叫做\emph{二维随机向量}或\emph{二维随机变量}. 第二章讨论的随机变量也叫一维随机变量.
\end{definition}

最终的二维随机变量$(X,Y)$:
    \begin{itemize}
        \item 与$X$有关, 与$Y$有关
        \item 与$X,Y$\emph{之间的关系}有关
    \end{itemize}
  因此通常将$(X,Y)$作为一个整体来研究.

  \begin{remark}
    这个实际上可以推广. 对于$n$维的随机变量也是有类似的记号.  
  \end{remark}

  

  像上面的例子一样, 我们同样可以定义随机变量的分布函数, 表示累积量的关系, 便于后面的问题分析. 
  \begin{definition}[多维变量的分布函数]
    \label{def:cumdist}
    设 $(X, Y)$ 是二维随机变量, 对于任意实数 $x, y$, 二元函数:
    $$
        F(x, y)=P\{(X \leqslant x) \cap(Y \leqslant y)\} := P\{X \leqslant x, Y \leqslant y\}
    $$
    称为二维随机变量 $(X, Y)$ 的分布函数, 或称为随机变量 $X$ 和 $Y$ 的\emph{联合分布函数}.
\end{definition}

二维的情形下, 如何求出落入小区间的概率? 即: 随机点 $(X, Y)$ 落在矩形域 $\{(x, y)\mid \left.x_1<x \leqslant x_2, y_1<y \leqslant y_2\right\}$ 的概率是什么?
  根据图像: 
  $$
        \begin{aligned}
             & P\left\{x_1<X \leqslant x_2, y_1<Y \leqslant y_2\right\}                                            \\
             & \quad=F\left(x_2, y_2\right)-F\left(x_2, y_1\right)+F\left(x_1, y_1\right)-F\left(x_1, y_2\right) .
        \end{aligned}
    $$

  \subsection{分布函数的性质}
  像一维的时候那样, 同样发现分布函数要满足这些基本的性质:

    \begin{proposition}[分布函数的性质]
      \label{prop:distfunc}
      分布函数满足如下的性质: 
      \begin{itemize}
        \item \emph{单调.}$F(x, y)$ 是变量 $x$ 和 $y$ 的不减函数
              \begin{itemize}
                  \item 对于任意固定的 $y$, 当 $x_2>x_1$ 时 $F\left(x_2, y\right) \geqslant F\left(x_1, y\right)$;
                  \item 对于任意固定的 $x$, 当 $y_2>y_1$ 时 $F\left(x, y_2\right) \geqslant F\left(x, y_1\right)$.
              \end{itemize}
        \item $0 \leqslant F(x, y) \leqslant 1 $
              \begin{itemize}
                  \item $\forall$固定的 $y, F(-\infty, y)=0$,
                  \item $\forall$固定的 $x, F(x,-\infty)=0$,
                  \item $F(-\infty,-\infty)=0, F(\infty, \infty)=1$.
              \end{itemize}
        \item \emph{右连续.}$F(x+0, y)=F(x, y), F(x, y+0)=F(x, y)$, 即 $F(x, y)$ 关于 $x$ 右连续,关于 $y$ 也右连续.
        \item \emph{非负.}对于任意 $\left(x_1, y_1\right),\left(x_2, y_2\right), x_1<x_2, y_1<y_2$,$F\left(x_2, y_2\right)-F\left(x_2, y_1\right)+F\left(x_1, y_1\right)-F\left(x_1, y_2\right) \geqslant 0$.
    \end{itemize}
    \end{proposition}

    \subsection{离散型随机变量}
    我们先从离散的情形看到一些概念:

    \begin{definition}[离散型随机变量]
        如果二维随机变量 $(X, Y)$ 全部可能取到的值是有限对或可列无限多对, 则称 $(X, Y)$ 是\emph{离散型的随机变量}.
    \end{definition}

    $$
        \begin{array}{|c|ccccc|}
            \hline Y \ddots X & x_1     & x_2     & \cdots & x_i     & \cdots \\
            \hline y_1 & p_{11}  & p_{21}  & \cdots & p_{i 1} & \cdots \\
            y_2        & p_{12}  & p_{22}  & \cdots & p_{i 2} & \cdots \\
            \vdots     & \vdots  & \vdots  &        & \vdots  &        \\
            y_j        & p_{1 j} & p_{2 j} & \cdots & p_{i j} & \cdots \\
            \vdots     & \vdots  & \vdots  &        & \vdots  &        \\
            \hline
        \end{array}
    $$
    
    我们同样希望使用表格来``枚举''每一种情况的概率. 这种表格我们称为``联合分布律''. 

    \begin{definition}
      设二维离散型随机变量 $(X, Y)$ 所有可能取的值为 $\left(x_i, y_j\right), i, j=1,2, \cdots$, 记 $P\left\{X=x_i, Y=y_j\right\}=p_{i j}, i, j=1,2, \cdots$, 由概率的定义有
    $$
        p_{i j} \geqslant 0, \quad \sum_{i=1}^{\infty} \sum_{j=1}^{\infty} p_{i j}=1 .
    $$

    称 $P\left\{X=x_i, Y=y_j\right\}=p_{i j}, i, j=1,2, \cdots$ 为二维离散型随机变量 $(X, Y)$的\emph{分布律},或随机变量 $X$ 和 $Y$ 的\emph{联合分布律}.
    \end{definition}

    实际上, 这种问题我们可以在未来处理连续性问题的时候带来一个平滑的转化. 

    \begin{example}
      设随机变量 $X$ 在 $1,2,3,4$ 四个整数中等可能地取一个值, 另一个随机变量 $Y$ 在 $1 \sim X$ 中等可能地取一整数值. 试求 $(X, Y)$ 的分布律.
    \end{example}
    \begin{solution}
      {可由乘法公式求得 $(X, Y)$ 的分布律.}
        { $$
            \begin{gathered}
                P\{X=i, Y=j\}=P\{Y=j \mid X=i\} P\{X=i\}=\frac{1}{i} \cdot \frac{1}{4}, \\
                i=1,2,3,4, j \leqslant i .
            \end{gathered}
        $$}
        {于是 $(X, Y)$ 的分布律为}
        $$
            \begin{array}{l|cccc}
                \hline Y \backslash X   & 1           & 2           & 3            & 4            \\
                \hline 1 & \frac{1}{4} & \frac{1}{8} & \frac{1}{12} & \frac{1}{16} \\
                2        & 0           &\frac{1}{8}   &\frac{1}{12} &
                \frac{1}{16}\\
                3        & 0           & 0            &\frac{1}{12} & \frac{1}{16} \\
                4        & 0           & 0            & 0            & \frac{1}{16} \\
                \hline
            \end{array}
        $$
    \end{solution}
    有了联合分布律, 自然可以得到联合分布函数. 按照\cref{def:cumdist}写一下就会发现, 
    $$
        F(x, y)=\sum_{x_i \leqslant x} \sum_{y_j \leqslant y} p_{i j},
    $$
    
    下面我们把上面的\cref{def:cumdist}推广到连续情形.

    \subsection{二维连续型随机变量}
    \begin{definition}
      对于二维随机变量 $(X, Y)$ 的分布函数 $F(x, y)$, 如果存在非负的函数 $f(x, y)$ 使对于任意 $x, y$ 有
      $$
          F(x, y)=\int_{-\infty}^y \int_{-\infty}^x f(u, v) \mathrm{d} u \mathrm{~d} v,
      $$
      则称 $(X, Y)$ 是连续型的二维随机变量, 函数 $f(x, y)$ 称为二维随机变量 $(X, Y)$ 的\emph{概率密度},或称为随机变量 $X$ 和 $Y$ 的\emph{联合概率密度}. $F$称为随机变量$X, Y$的联合分布函数. 
  \end{definition}
  观察到, 只是上述的$\sum$换成了$\int$. 而且它同样遵循一些换元规则. 我们可以使用矩阵的记号描述之. 



  和离散的情形一样, 联合分布函数也满足一些基本性质 -- 和\cref{prop:distfunc}所指示的基本类似. 
    \begin{itemize}
      \item \emph{非负.}  $f(x, y) \geqslant 0$.
      \item $\int_{-\infty}^{\infty} \int_{-\infty}^{\infty} f(u, v) \mathrm{d} u \mathrm{~d} v=F(\infty, \infty)=1$.
      \item 设 $G$ 是 $x O y$ 平面上的区域, 点 $(X, Y)$ 落在 $G$ 内的概率为
      $$
          P\{(X, Y) \in G\}=\iint_G f(u, v) \mathrm{d} u \mathrm{~d} v .
      $$
      \item 若 $f(x, y)$ 在点 $(x, y)$ 连续, 则有
      $$
          \frac{\partial^2 F(x, y)}{\partial x \partial y}=f(x, y) .
      $$
    \end{itemize}

    实际上, 上述的第(4)条性质有一个直观的解释: 若 $f(x, y)$ 在点 $(x, y)$ 处连续, 则当 $\Delta x, \Delta y$ 很小时$P\{x<X \leqslant x+\Delta x, y<Y \leqslant y+\Delta y\} \approx f(x, y) \Delta x \Delta y$. 对于一个联合密度函数, 如果其表达式为$z=f(x, y)$, 直观来看就是表示空间中的一个曲面. 这个曲面有一些特别之处:介于它和 $x O y$ 平面的空间区域的体积为 1. 并且概率分布的值就是$P\{(X, Y) \in G\}$ 的值等于以 $G$ 为底, 以曲面 $z=f(x, y)$ 为顶面的柱体体积.

    使用极限的语言也容易证明上述的第四条性质: 
    $$
            \begin{aligned}
                 & \lim _{\substack{\Delta x \rightarrow 0^{+}                                                                 \\
                \Delta y \rightarrow 0^{+}}} \frac{P\{x<X \leqslant x+\Delta x, y<Y \leqslant y+\Delta y\}}{\Delta x \Delta y} \\
                 & = \lim _{\substack{\Delta x \rightarrow 0^{+}                                                               \\
                \Delta y \rightarrow 0^{+}}} \frac{1}{\Delta x \Delta y}[ F(x+\Delta x, y+\Delta y)-F(x+\Delta x, y)           \\
                 & \quad \quad  -F(x, y+\Delta y)+F(x, y)]                                                                     \\
                 & =\frac{\partial^2 F(x, y)}{\partial x \partial y}=f(x, y)
            \end{aligned}
        $$
  \begin{remark}
    * 在微积分中, 我们了解了换元法. 并且很多时候它可以带给我们方便. 这里我们同样介绍类似的方法. 例如${P}\left\{a_1<X_1 \leqslant b_1, a_2<X_2 \leqslant b_2\right\}=\int_{a_1}^{b_1} \int_{a_2}^{b_2} f\left(x_1, x_2\right) \mathrm{d} x_1 \mathrm{~d} x_2$. 有时候为了方便起见, 我们会进行变量替换: $X_1=a_{11} Y_1+a_{12} Y_2, \quad X_2=a_{21} Y_1+a_{22} Y_2$. 这个变量替换是可逆的, 因为其行列式$\Delta=a_{11} a_{22}-a_{12} a_{21}\neq 0$. 

    带入这样的变量代换, 就将原来的区域变换为了$\mathrm{P}\{\Omega\}=\iint_{\Omega_*} f\left(a_{11} y_1+a_{12} y_2, a_{21} y_1+a_{22} y_2\right) \cdot \Delta \mathrm{d} y_1 \mathrm{~d} y_2$

    区域 $\Omega_*$ 由所有这样的点 $\left(y_1, y_2\right)$ 组成: 它的像 $\left(x_1, x_2\right)$ 在 $\Omega$ 中. 因为事件 $\left(X_1, X_2\right) \in$ $\Omega$ 和 $\left(Y_1, Y_2\right) \in \Omega_*$ 是相等的, 由此 $\left(Y_1, Y_2\right)$ 的联合密度由下式给出:
$$
g\left(y_1, y_2\right)=f\left(a_{11} y_1+a_{12} y_2, a_{21} y_1+a_{22} y_2\right) \cdot \Delta
$$

实际上, 像上面的内容进行的变量代换可以写作线性方程组的形式. 
行向量的利用需要把由 $\mathbf{R}^r$ 到 $\mathbf{R}^m$ 的线性变换写成形式
$$
\boldsymbol{Y}=\boldsymbol{X} \boldsymbol{A}
$$

即
$$
y_k=\sum_{j=1}^r a_{j_k} x_j \quad k=1, \cdots, m
$$
因此, 引入矩阵记号对我们有很大的帮助. 
  \end{remark}

  \begin{asidebox}
    回顾矩阵的一些基本内容:   
    \begin{itemize}
      \item 访问:一个矩阵在访问元素的时候, 首先是行, 其次是列. 因此, $\alpha \times \beta$ 矩阵 $\boldsymbol{A}$ 含有 $\alpha$ 行和 $\beta$ 列, 它的元素记为 $a_{j_k}$, 第 1 个下标表示行, 第 2 个下标表示列. 可以认为表现为一个线性方程组.
      \item 矩阵乘法: 类似于线性方程组的代换. 一般的规则是: 如果有$\boldsymbol{A}$ 含有 $\alpha$ 行和 $\beta$ 列, 其元素为$a_{j_k}$; 如果 $\boldsymbol{B}$ 是含有元素 $b_{j_k}$ 的 $\beta \times \gamma$ 矩阵乘积 $A B$ 是含有元素 $a_{j_1} b_{1 k}+a_{j_2} b_{2 k}+\cdots+a_{j_\beta} b_{\beta k}$ 的 $\alpha \times \gamma$ 矩阵.
      \begin{itemize}
        \item 不一定满足交换律, 但是满足结合律. 
      \end{itemize}
      \item 只含有一行的 $1 \times \alpha$ 矩阵称为行向量, 只含有一列的矩阵称为列向量.
    \end{itemize}
    矩阵的内积: 两个行向量 $\boldsymbol{x}=\left(x_1, \cdots, x_\alpha\right)$ 和 $\boldsymbol{y}=\left(y_1, \cdots, y_\alpha\right)$ 的内积定义为
    $$\boldsymbol{x} \boldsymbol{y}^{\mathrm{T}}=\boldsymbol{y} \boldsymbol{x}^{\mathrm{T}}=\sum_{j=1}^\alpha x_j y_j$$

    二次型: 如果 $a_{j k}=a_{k j}$, 即 $\boldsymbol{A}^{\mathrm{T}}=\boldsymbol{A}$, 则方阵 $\boldsymbol{A}$ 是对称的, 与对称的 $r \times r$ 矩阵 $\boldsymbol{A}$ 有关的二次型定义为
$$
\boldsymbol{x} \boldsymbol{A} \boldsymbol{x}^{\mathrm{T}}=\sum_{j, k=1}^r a_{j k} x_j x_k,
$$

其中 $x_1, \cdots, x_r$ 是未定的. 如果对于所有的非零向量 $\boldsymbol{x}$ 有 $\boldsymbol{x} \boldsymbol{A} \boldsymbol{x}^{\mathrm{T}}>0$, 是称矩阵 $\boldsymbol{A}$ 是正定的. 由上述准则推出, 正定矩阵是可逆的.
    
  \end{asidebox}

  \begin{example}
    设二维随机变量 $(X, Y)$ 具有概率密度
    $$
        f(x, y)= \begin{cases}2 \mathrm{e}^{-(2 x+y)}, & x>0, y>0, \\ 0, & \text { 其他. }\end{cases}
    $$

    (1) 求分布函数 $F(x, y)$; (2) 求概率 $P\{Y \leqslant X\}$.
  \end{example}
  \begin{solution}
    (1) $\begin{aligned} F(x, y) & =\int_{-\infty}^y \int_{-\infty}^x f(u, v) \mathrm{d} u \mathrm{~d} v \\ & = \begin{cases}\int_0^y \int_0^x 2 \mathrm{e}^{-(2 u+v)} \mathrm{d} u \mathrm{~d} v, & x>0, y>0, \\ 0, & \text { 其他. }\end{cases} \end{aligned}$

            即有 $F(x, y)= \begin{cases}\left(1-\mathrm{e}^{-2 x}\right)\left(1-\mathrm{e}^{-y}\right), & x>0, y>0, \\ 0, & \text { 其他. }\end{cases}$

            (2) 将 $(X, Y)$ 看作是平面上随机点的坐标. 即有$\{Y \leqslant X\}=\{(X, Y) \in G\},$其中 $G$ 为 $x O y$ 平面上直线 $y=x$ 及其下方的部分. 于是
                    $$\begin{aligned} P\{Y \leqslant X\} & =P\{(X, Y) \in G\}=\iint_G f(u, v) \mathrm{d} u \mathrm{~d} v \\ & =\int_0^{\infty} \int_y^{\infty} 2 \mathrm{e}^{-(2 u+v)} \mathrm{d} u \mathrm{~d} v=\frac{1}{3} . \quad \end{aligned}.$$
  \end{solution}
  \subsection{$n$维随机变量及其分布函数}
  可以推广到$n>2$的情形.
    \begin{definition}
        设 $E$ 是一个随机试验, 它的样本空间是 $S=\{e\}$, 设 $X_1=$ $X_1(e), X_2=X_2(e), \cdots, X_n=X_n(e)$ 是定义在 $S$上的随机变量, 由它们构成的一个 $n$ 维向量 $\left(X_1, X_2, \cdots, X_n\right)$ 叫\emph{做 $n$ 维随机向量}或 \emph{$n$ 维随机变量}.
    \end{definition}

    \begin{definition}
        对于任意 $n$ 个实数 $x_1, x_2, \cdots, x_n, n$ 元函数
        $$
            F\left(x_1, x_2, \cdots, x_n\right)=P\left\{X_1 \leqslant x_1, X_2 \leqslant x_2, \cdots, X_n \leqslant x_n\right\}
        $$
        称为 $n$ 维随机变量 $\left(X_1, X_2, \cdots, X_n\right)$ 的分布函数或随机变量 $X_1, X_2, \cdots, X_n$ 的联合分布函数.
    \end{definition}
    其具有类似于二维随机变量的分布函数的性质.
    
    把分布函数的定义推广到$n$维空间, 就给出了如下的定义. 

\begin{definition}
    n维随机变量的$(X_1,X_2,\dots,X_n)$的分布函数定义为
     \[F(x_1,x_2,\dots,x_n) = P\{X_1 \leq x_1,X_2 \leq x_2,\dots,X_n \leq x_n\},\]
    其中\(x_1,x_2,\dots,x_n\)为任意实数.\\
    若存在非负可积函数$f(x_1,x_2,\dots,x_n)$, 使对于任意实数$x_1,x_2,\dots,x_n$有
    \[F(x_1,x_2,\dots,x_n) = \int\limits_{-\infty}^{x_n}\int\limits_{-\infty}^{x_{n-1}} \dots \int\limits_{-\infty}^{x_1}f(x_1,x_2,\dots,x_n)d{x_1}d{x_2} \dots d{x_n},\]
    则称\(f(x_1,x_2,\dots,x_n)\)为\((X_1,X_2,\dots,X_n)\)的概率密度函数.
\end{definition}
\section{边缘分布}

我们可以用多种角度来看二维随机变量及其分布函数: 
\begin{itemize}
  \item 作为整体: $F(x, y)$, 其中$X,Y$都是随机变量.
  \item 关注局部: $X$ 和 $Y$ 也有他们自己的分布函数. 记为 $F_X(x), F_Y(y)$.
\end{itemize}

我们``选择性地''忽略一个变量, 就得到了如下的定义:
\begin{definition}[边缘分布]
  \label{def:marginaldist}
  二维随机变量 $(X, Y)$ 作为一个整体, 具有分布函数 $F(x, y)$. 而 $X$ 和 $Y$ 都是随机变量, 各自也有分布函数, 将它们分别记为 $F_X(x), F_Y(y)$, 依次称为二维随机变量 $(X, Y)$ 关于 $X$ 和关于 $Y$ 的\emph{边缘分布函数}. 边缘分布函数可以由 $(X, Y)$ 的分布函数 $F(x, y)$ 所确定, 并且$F_X(x)=P\{X \leqslant x\}=P\{X \leqslant x, Y<\infty\}=F(x, \infty)$.
\end{definition}

\subsection{离散情形的边缘分布}
根据\cref{def:marginaldist}和离散型变量的特点, 我们可以考虑离散状态下二维随机变量的边缘分布: 

假设有二维离散型随机变量 $(X, Y)$ ,其所有可能取的值为 $\left(x_i, y_j\right), i, j=1,2, \cdots$, 记 $P\left\{X=x_i, Y=y_j\right\}=p_{i j}$. 
\begin{definition*}
  考察$X$的边缘分布:
  $$
        F_X(x)=F(x, \infty)=\sum_{x_i \leqslant x} \sum_{j=1}^{\infty} p_{i j}
    $$

    \begin{itemize}
        \item $X$的分布律: $P\left\{X=x_i\right\}=\sum_{j=1}^{\infty} p_{i j}, \quad i=1,2, \cdots .$
        \item $Y$的分布律: $P\left\{Y=y_j\right\}=\sum_{i=1}^{\infty} p_{i j}, \quad j=1,2, \cdots .$
    \end{itemize}

    为了方便, 可以引入如下的记号
    $$
        \begin{aligned}
             & p_{i \red{\bullet}}:=\sum_{j=1}^{\infty} p_{i j}=P\left\{X=x_i\right\}, \quad i=1,2, \cdots,   \\
             & p_{{\red{\bullet}} j}:=\sum_{i=1}^{\infty} p_{i j}=P\left\{Y=y_j\right\}, \quad j=1,2, \cdots,
        \end{aligned}
    $$
    分别称 $p_{i \bullet} (i=1,2, \cdots)$ 和 $p_{\bullet j}(j=1,2, \cdots)$ 为 $(X, Y)$ 关于 $X$ 和关于 $Y$ 的\emph{边缘分布律}.
\end{definition*}
    
    其中, 下标中的$\red \bullet$类似通配符: $p_{i\bullet}$ 是由 $p_{i j}$ 关于 $j$ 求和后得到的, 反之亦然. 这样的记号将有助于理解连续的情形.

    \subsection{连续情形}

    \begin{definition*}
      对于连续型随机变量 $(X, Y)$, 设它的概率密度为 $f(x, y)$, 由于
        $$
            F_X(x)=F(x, \infty)=\int_{-\infty}^x\left[\int_{-\infty}^{\infty} f(u, v) \mathrm{d} v\right] \mathrm{d} u
        $$

        \begin{itemize}
            \item $X$是一个连续型的随机变量, 其概率密度为$f_X(x)=\int_{-\infty}^{\infty} f(x, v) \mathrm{d} v$.
            \item $Y$是一个连续型的随机变量, 其概率密度为$f_Y(y)=\int_{-\infty}^{\infty} f(u, y) \mathrm{d} u$.
        \end{itemize}

        分别称 $f_X(x), f_Y(y)$ 为 $(X, Y)$ 关于 $X$ 和关于 $Y$ 的\emph{边缘概率密度.}
    \end{definition*}

    \begin{example}
      设随机变量 $X$ 和 $Y$ 具有联合概率密度
            $$
                f(x, y)= \begin{cases}6, & x^2 \leqslant y \leqslant x, \\ 0, & \text { 其他. }\end{cases}
            $$
            求边缘概率密度 $f_X(x), f_Y(y)$.
    \end{example}

    \begin{solution}根据定义:
      $$\begin{aligned}
              f_X(x) & =\int_{-\infty}^{\infty} f(u, v) \mathrm{d} v= \begin{cases}\int_{x^2}^x 6 \mathrm{~d} v=6\left(x-x^2\right), 0 \leqslant & x \leqslant 1, \\
           0,                                                            & \text { 其他. }\end{cases}               \\
              f_Y(y) & =\int_{-\infty}^{\infty} f(u, v) \mathrm{d} u = \begin{cases}\int_y^{\sqrt{y}} 6 \mathrm{~d} u=6(\sqrt{y}-y), & 0 \leqslant y \leqslant 1, \\
           0,                                               & \text { 其他. }\end{cases}
          \end{aligned}
      $$
  \end{solution}

  最后一个问题是有边缘分布, 能推出联合分布吗? 事实上是不行的. 我们用正态分布做一个例子: 

  \begin{example}
    设二维随机变量 $(X, Y)$ 的概率密度为
    $$
        \begin{aligned}
            f(x, y)= &
            \frac{1}{2 \pi \sigma_1 \sigma_2 \sqrt{1-\rho^2}}
            \exp \left\{
            \frac { - 1 } { 2 ( 1 - \rho ^ { 2 } ) }
            \left[{\color{red}\frac{\left(x-\mu_1\right)^2}{\sigma_1^2}}\right.\right. \\
                     & \left.\left.-2 \rho
            \frac{\left({\color{red}x-\mu_1}\right)\left({\color{teal}y-\mu_2}\right)}{{\color{red}\sigma_1} {\color{teal}\sigma_2}}
            +{\color{teal}\frac{\left(y-\mu_2\right)^2}{\sigma_2^2}}\right]\right\},
        \end{aligned}
    $$
    其中 $\mu_1, \mu_2, \sigma_1, \sigma_2, \rho$ 都是常数, 且 $\sigma_1>0, \sigma_2>0,-1<\rho<1$. 我们称 $(X, Y)$ 为服从参数为 $\mu_1, \mu_2, \sigma_1, \sigma_2, \rho$ 的二维正态分布 (这五个参数的意义将在下一章说明), 记为 $(X, Y) \sim N\left(\mu_1, \mu_2, \sigma_1^2, \sigma_2^2, \rho\right)$. 试求二维正态随机变量的边缘概率密度.
\end{example}

\begin{solution}
  要计算$f_X(x)=\int_{-\infty}^{\infty} f(x, y) \mathrm{d} y$, 由于
            $$\frac{\left(y-\mu_2\right)^2}{\sigma_2^2}-2 \rho \frac{\left(x-\mu_1\right)\left(y-\mu_2\right)}{\sigma_1 \sigma_2}=\left(\frac{y-\mu_2}{\sigma_2}-\rho \frac{x-\mu_1}{\sigma_1}\right)^2-\rho^2 \frac{\left(x-\mu_1\right)^2}{\sigma_1^2}$$,

            于是$$f_X(x)=\frac{1}{2 \pi \sigma_1 \sigma_2 {\color{red}\sqrt{1-\rho^2}}} \mathrm{e}^{-\frac{\left(x-\mu_1\right)^2}{2 \sigma_1^2}} \int_{-\infty}^{\infty} \mathrm{e}^{-\frac{1}{2\left(1-\rho^2\right)}\left({\color{red}\frac{y-\mu_2}{\sigma_2}-\rho \frac{x-\mu_1}{\sigma_1}}\right)^2} \mathrm{~d} y$$
        令 $$t:=\frac{1}{\sqrt{1-\rho^2}}\left(\frac{y-\mu_2}{\sigma_2}-\rho \frac{x-\mu_1}{\sigma_1}\right)$$

        则有$$f_X(x)=\frac{1}{2 \pi \sigma_1} \mathrm{e}^{-\frac{\left(x-\mu_1\right)^2}{2 \sigma_1^2}} \int_{-\infty}^{\infty} \mathrm{e}^{-t^2 / 2} \mathrm{~d} t$$
        即 $$f_X(x)=\frac{1}{\sqrt{2 \pi} \sigma_1} \mathrm{e}^{-\frac{\left(x-\mu_1\right)^2}{2 \sigma_1^2}}, \quad-\infty<x<\infty,$$
        同理有
        \small $$f_Y(y)=\frac{1}{\sqrt{2 \pi} \sigma_2} \mathrm{e}^{-\frac{\left(y-\mu_2\right)^2}{2 \sigma_2^2}}, \quad-\infty<y<\infty.$$

        我们发现二维正态分布的两个边缘分布都是一维正态分布, 而且并不依赖$\rho$. 
        因此, 边缘分布一般不能决定联合分布. 
\end{solution}

\subsection{推广到$n$维}

当然, 高维空间里面的边缘分布也是通过``选择性忽略''有些值定义的. 

\begin{definition}
  设$(X_1,X_2,\dots,X_n)$的分布函数$F(x_1,x_2,\dots,x_n)$为已知, 则$(X_1,X_2,\dots,X_n)$的$k(1 \leq k < n )$维边缘分布函数就随之确定, 例如$(X_1,X_2,\dots,X_n)$关于$X_1$、关于$(X_1,X_2)$的边缘分布函数分别为
    \[F_{X_1}(x_1) = F(x_1,\infty,\infty,\dots,\infty)\],
    \[F_{X_1,X_2}(x_1,x_2) = F(x_1,x_2,\infty,\infty,\dots,\infty)\].
又若$f(x_1,x_2,\dots,x_n)$是$(X_1,X_2,\dots,X_n)$的概率密度, 则$(X_1,X_2,\dots,X_n)$关于$X_1$、关于$(X_1,X_2)$的边缘概率密度分别为
    \[f_{X_1}(x_1) = \int_{-\infty}^{\infty}\int_{-\infty}^{\infty}
    \dots \int_{-\infty}^{\infty}f(x_1,x_2,\dots,x_n)d{x_2}d{x_3}\dots d{x_n}\].
\end{definition}
\section{相互独立的随机变量}

在离散的情形, 我们定义两个事件是独立的当且仅当$P(AB)=P(A)P(B)$. 那么对于随机变量, 我们也给出同样的定义: 

\begin{definition}[相互独立的随机变量]
    设 $\red{F_{X,Y}(x, y)}$ 及 $\teal{F_X(x), F_Y(y)}$ 分别是二维随机变量 $(X, Y)$ 的分布函数及边缘分布函数. 若对于所有 $x, y$ 有
    $$
\begin{gathered}
    P\{X \leqslant x, Y \leqslant y\}=P\{X \leqslant x\} P\{Y \leqslant y\}, \\
\red{F_{X,Y}(x, y)}=\teal{F_X(x) F_Y(y)},
\end{gathered}
$$
则称随机变量 $X$ 和 $Y$ 是相互独立的.
\end{definition}

实际上, 刚刚的定义给了我们一点提示: 如果$(X, Y)$ 是连续型随机变量,$f_{X,Y}(x, y), f_X(x), f_Y(y)$是概率密度以及边缘概率密度, 那么: 
\begin{itemize}
    \item $X$ 和 $Y$ 相互独立 $\iff$ $f_{X,Y}(x, y)=f_X(x) f_Y(y)$ 几乎处处成立. 
    \item (除了在平面上面积为0的集合)
\end{itemize}

也就是说, 如果 $(X, Y)$ 是离散型随机变量:
    \begin{itemize}
        \item $P\left\{X=x_i, Y=y_j\right\}=P\left\{X=x_i\right\} P\left\{Y=y_j\right\}$.
    \end{itemize}

\subsection{推广到$n$维}

对于多个事件的独立性, 我们知道任意的一个事件的子集都必须满足$P(A_1\cdots A_k)=P(A_1)\cdots P(A_n)$. 高维的情形也是类似的: 

\begin{definition*}

    设$(X_1,X_2,\dots,X_n)$的分布函数$F(x_1,x_2,\dots,x_n)$为已知, 则$(X_1,X_2,\dots,X_n)$的$k(1 \leq k < n )$维边缘分布函数就随之确定, 例如$(X_1,X_2,\dots,X_n)$关于$X_1$、关于$(X_1,X_2)$的边缘分布函数分别为
    \[F_{X_1}(x_1) = F(x_1,\infty,\infty,\dots,\infty),\]
    \[F_{X_1,X_2}(x_1,x_2) = F(x_1,x_2,\infty,\infty,\dots,\infty)\].
又若$f(x_1,x_2,\dots,x_n)$是$(X_1,X_2,\dots,X_n)$的概率密度, 则$(X_1,X_2,\dots,X_n)$关于$X_1$、关于$(X_1,X_2)$的边缘概率密度分别为
    \[f_{X_1}(x_1) = \int_{-\infty}^{\infty}\int_{-\infty}^{\infty}
    \dots \int_{-\infty}^{\infty}f(x_1,x_2,\dots,x_n)d{x_2}d{x_3}\dots d{x_n}\].

    \[f_{X_1,X_2}\left(x_1,x_2\right) = \int_{-\infty}^{\infty} \int_{-\infty}^{\infty} \dots \int_{-\infty}^{\infty}f\left(x_1,x_2,\dots,x_n\right)d{x_3}d{x_4} \dots {x_n},\]
    \quad 若对于所有的$x_1,x_2,\dots,x_n$有
    \[F\left(x_1,x_2,\dots,x_n\right) = F_{X_1}\left(x_1\right)F_{X_2}\left(x_2\right)\dots F_{X_n}\left(x_n\right),\]
    则称$X_1,X_2,\dots,X_n$是相互独立的.\\
    若对于所有的$x_1,x_2,\dots,x_m$;$y_1,y_2,\dots,y_n$有
    \[F\left(x_1,x_2,\dots,x_m,y_1,y_2,\dots,y_n\right) = F_1\left(x_1,x_2,\dots,x_m\right)F_2\left(y_1,y_2,\dots,y_n\right),\]
    其中$F_1$,$F_2$,F依次为随机变量$\left(X_1,X_2,\dots,X_m\right)$,$\left(Y_1,Y_2,\dots,Y_n\right)$和$\left(X_1,X_2,\dots,X_m,Y_1,Y_2,\dots,Y_n\right)$的分布函数, 则称随机变量$\left(X_1,X_2,\dots,X_m\right)$和$\left(Y_1,Y_2,\dots,Y_n\right)$是相互独立的.
\end{definition*}


\begin{theorem}
    设$\left(X_1,X_2,\dots,X_m\right)$和$\left(Y_1,Y_2,\dots,Y_n\right)$相互独立, 则$X_i\left(i = 1,2,\dots,m\right)$和$Y_j\left(j = 1,2,\dots,n\right)$相互独立, 又若$h,g$是连续函数,则$h\left(X_1,X_2,\dots,X_m\right)$和$g\left(Y_1,Y_2,\dots,Y_n\right)$相互独立.
\end{theorem}

\section{两个随机变量的函数的分布}

我们先来看一个问题: $X,Y$是相互独立的离散型随机变量, 等概率地取$[0,3]$区间的整数. 问$X+Y=3$的概率是多少? 

解答也不难. 考虑一共有$(0, 3), (1, 2), (2, 1), (3, 0)$四种情况, 相加即可. 那么我们把这个$X+Y$看做新的随机变量, 应该如何求? 实际上我们只要枚举就好了.
$$
\begin{aligned}
P_W(w) & =P(X+Y=w) \\
& =\sum_x P(X=x)  P(Y=w-x) \\
& =\sum_x P_X(x) P_Y(w-x) . 
\end{aligned}
$$

我们现在来系统的考察对于任意的分布函数, 几个常见的两个随机变量参与运算之后的概率密度. 

\paragraph{(一) $Z=X+Y$ 的分布}

设 $(X, Y)$ 是二维连续型随机变量, 它具有概率密度 $f_{X,Y}(x, y)$.那么连续型随机变量$Z=X+Y$的概率密度是多少?

大致思路: 
    \begin{itemize}
        \item 先来求 $Z=X+Y$ 的分布函数 $F_Z(z)$
        \item $F_Z(z)=P(Z \leq z)=\iint_{x+y \leq z} f_{X,Y}(x, y) \mathrm{d} x \mathrm{~d} y$
    \end{itemize}
    $$\begin{aligned}
        F_Z(z)&=\int_{-\infty}^{\infty}\left[\int_{-\infty}^{\purple {z-y}} f({\red{x}}, y) \mathrm{d} {\red{x}}\right] \mathrm{d} y \\ 
        &\stackrel{{\red x}:=\teal{u-y}}{\stackrel{\rule{1.5cm}{0.4pt}}{\rule{1.5cm}{0.4pt}}}
        \int_{-\infty}^{\infty}\left[\int_{-\infty}^z f({\teal {u-y}}, y) \mathrm{d} u\right] \mathrm{d} y\\
        &=\int_{-\infty}^z\left[\int_{-\infty}^{\infty} f(u-y, y) \mathrm{d} y\right] \mathrm{d} u
    \end{aligned}$$

    这里的换元法实际上是使用给定的不等关系$x+y\leq z$为了消去$x$, 减少变量个数. 

    根据定义, 求导得到: 
    $f_{X+Y}(z)=\int_{-\infty}^{\infty} f(u-y, y) \mathrm{d} y=\int_{-\infty}^{\infty} f(z-y, y) \mathrm{d} y.$

我们用定理的形式总结这一事实: 
\begin{theorem}
    设 $(X, Y)$ 是二维连续型随机变量, 它具有概率密度 $f_{X,Y}(x, y)$. 则 $Z=X+Y$ 仍为连续型随机变量, 其概率密度为
    $$f_{X+Y}(z)=\int_{-\infty}^{\infty} f(z-y, y) \mathrm{d} y$$
    或
    $$
f_{X+Y}(z)=\int_{-\infty}^{\infty} f(x, z-x) \mathrm{d} x
$$
\end{theorem}

如果$X,Y$相互独立的话, 上述公式可以进一步地写作
\begin{itemize}
    \item $f_{X+Y}(z)=\int_{-\infty}^{\infty} f_X(z-y) f_Y(y) \mathrm{d} y$;
    \item $f_{X+Y}(z)=\int_{-\infty}^{\infty} f_X(x) f_Y(z-x) \mathrm{d} x$
\end{itemize}

\begin{asidebox}
    \newword{卷积}{convolution}: 刚刚常见的操作其实有一个名字: 卷积. 比如我们在求两个多项式的乘积的时候, $(\sum_{i=0}^{\infty} a_i x^i)(\sum_{j=0}^{\infty} b_j x^j).$

    我们可以用这样的公式计算: 

    $$
    \begin{aligned}
        \sum_{k=0}^{\infty}\left(\sum_{i+j=k} a_i b_j\right) x^k
        =\sum_{k=0}^{\infty}\left(\sum_{i=0}^k a_i b_{k-i}\right) x^k
    \end{aligned}
    $$

    刚刚的那个问题同样和这个问题有类似的性质, 只是求和号变为了积分号. 

\end{asidebox}

为了方便起见, 这两个公式称为 $f_X$ 和 $f_Y$ 的卷积公式, 记为 $f_X * f_Y$, 即
$$
f_X * f_Y=\int_{-\infty}^{\infty} f_X(z-y) f_Y(y) \mathrm{d} y=\int_{-\infty}^{\infty} f_X(x) f_Y(z-x) \mathrm{d} x
$$

\begin{asidebox}
    直观理解卷积: 先把纸片翻一下, 然后平移, 最后对应相乘相加. 如\cref{fig:convolution}.
\end{asidebox}

\begin{figure}
    \center
    % \usepackage[usenames,dvipsnames]{pstricks}
% \usepackage{pstricks-add}
% \usepackage{epsfig}
% \usepackage{pst-grad} % For gradients
% \usepackage{pst-plot} % For axes
% \usepackage[space]{grffile} % For spaces in paths
% \usepackage{etoolbox} % For spaces in paths
% \makeatletter % For spaces in paths
% \patchcmd\Gread@eps{\@inputcheck#1 }{\@inputcheck"#1"\relax}{}{}
% \makeatother
% 
\psscalebox{0.7 0.7} % Change this value to rescale the drawing.
{
\begin{pspicture}(0,11.045)(26.246119,21.935)
\definecolor{colour2}{rgb}{0.0,0.0,0.5019608}
\definecolor{colour3}{rgb}{0.0,0.5019608,0.5019608}
\definecolor{colour5}{rgb}{0.8,0.2,0.2}
\psline[linecolor=colour2, linewidth=0.04, fillstyle=crosshatch, hatchwidth=0.028222222, hatchangle=0.0, hatchsep=0.1411111, arrowsize=0.05291667cm 2.0,arrowlength=1.4,arrowinset=0.0]{->}(6.462786,17.445)(6.462786,20.445)
\psline[linecolor=colour2, linewidth=0.04, fillstyle=crosshatch, hatchwidth=0.028222222, hatchangle=0.0, hatchsep=0.1411111, arrowsize=0.05291667cm 2.0,arrowlength=1.4,arrowinset=0.0]{->}(6.3961196,17.445)(12.29612,17.445)
\psline[linecolor=colour2, linewidth=0.04, fillstyle=crosshatch, hatchwidth=0.028222222, hatchangle=0.0, hatchsep=0.1411111, dotsize=0.07055555cm 2.0]{-*}(7.3961196,17.445)(7.3961196,18.445)(7.3961196,18.445)
\psline[linecolor=colour2, linewidth=0.04, fillstyle=crosshatch, hatchwidth=0.028222222, hatchangle=0.0, hatchsep=0.1411111, dotsize=0.07055555cm 2.0]{-*}(8.396119,17.445)(8.396119,18.445)
\psline[linecolor=colour2, linewidth=0.04, fillstyle=crosshatch, hatchwidth=0.028222222, hatchangle=0.0, hatchsep=0.1411111, dotsize=0.07055555cm 2.0]{-*}(9.396119,17.445)(9.396119,18.445)
\psline[linecolor=colour2, linewidth=0.04, fillstyle=crosshatch, hatchwidth=0.028222222, hatchangle=0.0, hatchsep=0.1411111, dotsize=0.07055555cm 2.0]{-*}(10.396119,17.445)(10.396119,18.445)
\rput[bl](7.0961194,18.745){$1/4$}
\rput[bl](8.096119,18.745){$1/4$}
\rput[bl](9.096119,18.745){$1/4$}
\rput[bl](10.096119,18.745){$1/4$}
\rput[bl](5.1961193,20.445){\textcolor{colour2}{$p_X(x)$
}}
\rput[bl](12.396119,17.345){\textcolor{colour2}{$x$}}
\psline[linecolor=colour3, linewidth=0.04, fillstyle=crosshatch, hatchwidth=0.028222222, hatchangle=0.0, hatchsep=0.1411111, arrowsize=0.05291667cm 2.0,arrowlength=1.4,arrowinset=0.0]{->}(6.462786,12.545)(6.462786,15.545)
\psline[linecolor=colour3, linewidth=0.04, fillstyle=crosshatch, hatchwidth=0.028222222, hatchangle=0.0, hatchsep=0.1411111, arrowsize=0.05291667cm 2.0,arrowlength=1.4,arrowinset=0.0]{->}(6.4961195,12.545)(12.29612,12.545)
\psline[linecolor=colour3, linewidth=0.04, fillstyle=crosshatch, hatchwidth=0.028222222, hatchangle=0.0, hatchsep=0.1411111, dotsize=0.07055555cm 2.0]{-*}(8.396119,12.545)(8.396119,14.645)
\psline[linecolor=colour3, linewidth=0.04, fillstyle=crosshatch, hatchwidth=0.028222222, hatchangle=0.0, hatchsep=0.1411111, dotsize=0.07055555cm 2.0]{-*}(10.396119,12.545)(10.396119,13.345)
\rput[bl](8.396119,14.445){\textcolor{colour3}{$2/3$}}
\rput[bl](9.896119,13.445){\textcolor{colour3}{$1/3$}}
\rput[bl](5.162786,15.545){\textcolor{colour5}{$p_X(x)$
}}
\rput[bl](12.396119,12.445){\textcolor{colour3}{$x$}}
\rput[bl](7.296119,20.245){\textcolor{colour2}{$p_X(x)=1/4(x=1,2,3,4)$}}
\rput[bl](7.1961193,15.045){\textcolor{colour3}{$p_Y(y)=\begin{cases}1/3, y=4;  2/3, y=2; 0, ow\end{cases}$}}
\rput[bl](7.296119,17.045){1}
\rput[bl](8.29612,17.045){2}
\rput[bl](9.29612,17.045){3}
\rput[bl](10.29612,17.045){4}
\rput[bl](7.296119,12.145){\textcolor{colour3}{1}}
\rput[bl](8.29612,12.145){\textcolor{colour3}{2}}
\rput[bl](9.29612,12.145){\textcolor{colour3}{3}}
\rput[bl](10.29612,12.145){\textcolor{colour3}{4}}
\rput[bl](7.296119,21.545){$P(X+Y=3)=?$}
\psline[linecolor=colour2, linewidth=0.04, fillstyle=crosshatch, hatchwidth=0.028222222, hatchangle=0.0, hatchsep=0.1411111, arrowsize=0.05291667cm 2.0,arrowlength=1.4,arrowinset=0.0]{->}(18.89612,17.178333)(18.89612,20.178333)
\psline[linecolor=colour2, linewidth=0.04, fillstyle=crosshatch, hatchwidth=0.028222222, hatchangle=0.0, hatchsep=0.1411111, arrowsize=0.05291667cm 2.0,arrowlength=1.4,arrowinset=0.0]{->}(18.89612,17.178333)(25.79612,17.178333)
\psline[linecolor=colour2, linewidth=0.04, fillstyle=crosshatch, hatchwidth=0.028222222, hatchangle=0.0, hatchsep=0.1411111, dotsize=0.07055555cm 2.0]{-*}(20.89612,17.178333)(20.89612,18.178333)(20.89612,18.178333)
\psline[linecolor=colour2, linewidth=0.04, fillstyle=crosshatch, hatchwidth=0.028222222, hatchangle=0.0, hatchsep=0.1411111, dotsize=0.07055555cm 2.0]{-*}(21.89612,17.178333)(21.89612,18.178333)
\psline[linecolor=colour2, linewidth=0.04, fillstyle=crosshatch, hatchwidth=0.028222222, hatchangle=0.0, hatchsep=0.1411111, dotsize=0.07055555cm 2.0]{-*}(22.89612,17.178333)(22.89612,18.178333)
\psline[linecolor=colour2, linewidth=0.04, fillstyle=crosshatch, hatchwidth=0.028222222, hatchangle=0.0, hatchsep=0.1411111, dotsize=0.07055555cm 2.0]{-*}(23.89612,17.178333)(23.89612,18.178333)
\rput[bl](20.596119,18.478333){$1/4$}
\rput[bl](21.596119,18.478333){$1/4$}
\rput[bl](22.596119,18.478333){$1/4$}
\rput[bl](23.596119,18.478333){$1/4$}
\rput[bl](18.69612,20.178333){\textcolor{colour2}{$p_X(x)$
}}
\rput[bl](25.89612,17.078333){\textcolor{colour2}{$x$}}
\psline[linecolor=colour5, linewidth=0.04, fillstyle=crosshatch, hatchwidth=0.028222222, hatchangle=0.0, hatchsep=0.1411111, arrowsize=0.05291667cm 2.0,arrowlength=1.4,arrowinset=0.0]{->}(6.358333,12.544683)(6.358333,15.544683)
\psline[linecolor=colour5, linewidth=0.04, fillstyle=crosshatch, hatchwidth=0.028222222, hatchangle=0.0, hatchsep=0.1411111, arrowsize=0.05291667cm 2.0,arrowlength=1.4,arrowinset=0.0]{->}(6.3961196,12.545)(0.10416634,12.544683)
\psline[linecolor=colour5, linewidth=0.04, fillstyle=crosshatch, hatchwidth=0.028222222, hatchangle=0.0, hatchsep=0.1411111, dotsize=0.07055555cm 2.0]{-*}(4.466666,12.544683)(4.466666,14.644684)
\psline[linecolor=colour5, linewidth=0.04, fillstyle=crosshatch, hatchwidth=0.028222222, hatchangle=0.0, hatchsep=0.1411111, dotsize=0.07055555cm 2.0]{-*}(2.483333,12.544683)(2.483333,13.344684)
\rput[bl](4.466666,14.444683){\textcolor{colour5}{$2/3$}}
\rput[bl](2.2041664,13.644684){\textcolor{colour5}{$1/3$}}
\rput[bl](5.4999995,15.444683){\textcolor{colour5}{$p_X(x)$
}}
\rput[bl](0.0,12.444683){\textcolor{colour5}{$x$}}
\rput[bl](20.79612,19.978333){\textcolor{colour2}{$p_X(x)=1/4(x=1,2,3,4)$}}
\rput[bl](20.79612,16.778334){1}
\rput[bl](21.79612,16.778334){2}
\rput[bl](22.79612,16.778334){3}
\rput[bl](23.79612,16.778334){4}
\rput[bl](5.3124995,12.144684){\textcolor{colour5}{1}}
\rput[bl](4.270833,12.144684){\textcolor{colour5}{2}}
\rput[bl](3.2291663,12.144684){\textcolor{colour5}{3}}
\rput[bl](2.1874998,12.144684){\textcolor{colour5}{4}}
\psarc[linecolor=colour5, linewidth=0.04, dimen=outer, arrowsize=0.05291667cm 2.0,arrowlength=1.4,arrowinset=0.0]{->}(6.2294526,13.078333){3.3}{60.786674}{118.705956}
\psline[linecolor=red, linewidth=0.04, fillstyle=crosshatch, hatchwidth=0.028222222, hatchangle=0.0, hatchsep=0.1411111, arrowsize=0.05291667cm 2.0,arrowlength=1.4,arrowinset=0.0]{->}(23.658333,12.678017)(23.658333,15.678017)
\psline[linecolor=red, linewidth=0.04, fillstyle=crosshatch, hatchwidth=0.028222222, hatchangle=0.0, hatchsep=0.1411111, arrowsize=0.05291667cm 2.0,arrowlength=1.4,arrowinset=0.0]{->}(23.658333,12.678017)(16.470833,12.678017)
\psline[linecolor=red, linewidth=0.04, fillstyle=crosshatch, hatchwidth=0.028222222, hatchangle=0.0, hatchsep=0.1411111, dotsize=0.07055555cm 2.0]{-*}(20.833332,12.678017)(20.833332,14.778017)
\psline[linecolor=red, linewidth=0.04, fillstyle=crosshatch, hatchwidth=0.028222222, hatchangle=0.0, hatchsep=0.1411111, dotsize=0.07055555cm 2.0]{-*}(18.85,12.678017)(18.85,13.478017)
\rput[bl](20.833332,14.578017){\textcolor{red}{$2/3$}}
\rput[bl](18.570833,13.778017){\textcolor{red}{$1/3$}}
\rput[bl](23.866667,15.678017){\textcolor{red}{$p_X(x)$
}}
\rput[bl](16.366667,12.578017){\textcolor{red}{$x$}}
\rput[bl](21.679167,12.278017){\textcolor{red}{1}}
\rput[bl](20.637499,12.278017){\textcolor{red}{2}}
\rput[bl](19.595833,12.278017){\textcolor{red}{3}}
\rput[bl](18.554167,12.278017){\textcolor{red}{4}}
\psline[linecolor=black, linewidth=0.04, arrowsize=0.05291667cm 2.0,arrowlength=1.4,arrowinset=0.0]{->}(19.19612,16.045)(21.096119,16.045)(21.096119,16.045)
\rput[bl](21.162786,15.911667){3}
\rput[bl](5.4961195,11.045){(a)}
\rput[bl](20.89612,11.345){(b)}
\end{pspicture}
}


    \caption{直观理解卷积}
    \label{fig:convolution}
\end{figure}

\paragraph{(二)$Z=Y/X,Z=XY$的分布}

设 $(X, Y)$ 是二维连续型随机变量, 它具有概率密度 $f_{X,Y}(x, y)$.连续型随机变量$Z=Y/X$的概率密度是多少?

我们还是还是首先设出$F_{Y/X}(z)=P(Y/X\leq z)$. 但是这里由于函数的不连续性, 需要分两种情况: $x<0, x>0$分别考虑:

$$
\begin{aligned}
F_{Y / X}(z) & =P(Y / X \leq z)=\iint_{\stackrel{y / x \leq z}{x<0}} f(x, y) \mathrm{d} y \mathrm{~d} x+\iint_{\stackrel{y / x \leq z}{x>0}} f(x, y) \mathrm{d} y \mathrm{~d} x \\
& =\int_{-\infty}^0\left[\int_{z x}^{\infty} f(x, y) \mathrm{d} y\right] \mathrm{d} x+\int_0^{\infty}\left[\int_{-\infty}^{z x} f(x, y) \mathrm{d} y\right] \mathrm{d} x \\
& \varsub{y:=xu}{1cm} \int_{-\infty}^0\left[\int_z^{-\infty} x f(x, x u) \mathrm{d} u\right] \mathrm{d} x+\int_0^{\infty}\left[\int_{-\infty}^z x f(x, x u) \mathrm{d} u\right] \mathrm{d} x \\
& =\int_{-\infty}^0\left[\int_{-\infty}^z(-x) f(x, x u) \mathrm{d} u\right] \mathrm{d} x+\int_0^{\infty}\left[\int_{-\infty}^z x f(x, x u) \mathrm{d} u\right] \mathrm{d} x \\
& =\int_{-\infty}^z\left[\int_{-\infty}^{+\infty}|x| f(x, x u) \mathrm{d} u\right] \mathrm{d} x
\end{aligned}
$$

遵循同样的模式, 同样可以求出: $Z=XY$的概率分布. 
$$
\begin{aligned} & F_{XY}(z)=P(XY\leq z)\\
 & =\iint_{\substack{xy\leq z\\
 x<0}}f(x,y)\dd y\dd x+\iint_{\substack{xy\leq z\\
 x>0}}f(x,y)\dd y\dd x\\
 & =\int_{-\infty}^{0}\left(\int_{z/x}^{+\infty}f(x,y)\dd y\right)\dd x+\int_{0}^{+\infty}\left(\int_{-\infty}^{z/x}f(x,y)\dd y\right)\dd x\\
 & \varsub{y:=u/x}{1.5cm}\int_{-\infty}^{0}\left(\int_{z/x}^{+\infty}f\left(x,\frac{u}{x}\right)d\left(\frac{u}{x}\right)\right)\dd x+\int_{0}^{+\infty}\left(\int_{-\infty}^{z/x}f\left(x,\left(\frac{u}{x}\right)\dd x\right)\right)\\
 & =\int_{-\infty}^{0}\left(\left(\frac{1}{x}\right)\int_{z}^{-\infty}f\left(x,\frac{u}{x}\right)\dd u\right)\dd x+\int_{0}^{+\infty}\left(\frac{1}{x}\int_{-\infty}^{z}f\left(x,\frac{u}{x}\right)\dd u\right)\dd x\\
 & =\int_{0}^{z}\left(\int_{-\infty}^{\infty}\frac{1}{|x|}f\left(x,\frac{u}{x}\right)\dd u\right)\dd x
\end{aligned}
$$

\paragraph{(三) $M=\min \{X, Y\}$的分布}

$X, Y$ 是两个\emph{相互独立}的随机变量, 它们的分布函数分别为 $F_X(x)$ 和$F_Y(y)$.求 $M=\min \{X, Y\}$的分布函数.


   $$
\begin{aligned}
F_{\min }(z) & =P(N \leq z\}=1-P\{N>z) \\
& =1-P(X>z, Y>z\}=1-P(X>z) \cdot P\{Y>z)
\end{aligned}
$$

也就是
$$
F_{\min }(z)=1-\left[1-F_X(z)\right]\left[1-F_Y(z)\right] .
$$

上述的三个情况都可以推广到$n$维的情形.

\part{随机变量的数字特征}

\section{数学期望}

有这样的一个游戏: 花2元并投掷一颗均匀的骰子. 如果事件A = \{1, 2, 3\} 发生, 收到1元. 如果事件B = \{4, 5\} 发生, 收到2元. 如果事件C = \{6\} 发生, 收到6元. 你会参加这个游戏吗?

\begin{webaside}
    实际上, 真理元素的频道主实际上真的在路边做了这个实验. 可以参看他们的视频: \href{https://www.bilibili.com/video/BV1Xx411b7rM}{Bilibili: BV1Xx411b7rM}
\end{webaside}

可能我们的第一考虑是看看``平均''能得多少. 这样的随机变量$X$, \[
    X =
    \begin{cases}
    1 & \text{如果事件 A 发生} \\
    2 & \text{如果事件 B 发生} \\
    6 & \text{如果事件 C 发生}
    \end{cases}
    \]
    
    事件A、B、C的概率分别是:
    \[P(A) = \frac{3}{6}, \quad P(B) = \frac{2}{6}, \quad P(C) = \frac{1}{6}\]
    
    想法: 求平均值$1 \cdot P(A) + 2 \cdot P(B) + 6 \cdot P(C)= \frac{13}{6}$




\end{document}


