% !TEX root = main.tex
\documentclass{article}
\usepackage{amsmath, amsthm, amssymb, amsfonts}
\usepackage{thmtools}
\usepackage{graphicx}
\usepackage{setspace}
\usepackage{geometry}
\usepackage{float}
\usepackage{hyperref}
\usepackage[utf8]{inputenc}
\usepackage[english]{babel}
\usepackage{framed}
\usepackage[dvipsnames]{xcolor}
\usepackage{tcolorbox}

\colorlet{LightGray}{White!90!Periwinkle}
\colorlet{LightOrange}{Orange!15}
\colorlet{LightGreen}{Green!15}

\newcommand{\HRule}[1]{\rule{\linewidth}{#1}}

\declaretheoremstyle[name=Theorem,]{thmsty}
\declaretheorem[style=thmsty,numberwithin=section]{theorem}
\tcolorboxenvironment{theorem}{colback=LightGray}

\declaretheoremstyle[name=Proposition,]{prosty}
\declaretheorem[style=prosty,numberlike=theorem]{proposition}
\tcolorboxenvironment{proposition}{colback=LightOrange}

\declaretheoremstyle[name=Principle,]{prcpsty}
\declaretheorem[style=prcpsty,numberlike=theorem]{principle}
\tcolorboxenvironment{principle}{colback=LightGreen}

\declaretheoremstyle[name=Definition,]{prcpsty}
\declaretheorem[style=prcpsty,numberlike=theorem]{definition}
\tcolorboxenvironment{definition}{colback=white}

\usepackage{enumitem}

\setlist{nosep}

\setstretch{1.2}
\geometry{
    textheight=9in,
    textwidth=5.5in,
    top=1in,
    headheight=12pt,
    headsep=25pt,
    footskip=30pt
}

\usepackage{ctex}
% \usepackage[utf8]{inputenc}

\begin{document}

% \title{ \normalsize \textsc{}
		\\ [2.0cm]
		\HRule{1.5pt} \\
		\LARGE \textbf{{Lecture Notes}
		\HRule{2.0pt} \\ [0.6cm] \LARGE{2023秋 ~~~概率论与数理统计A} \vspace*{10\baselineskip}}
		}
\date{}
\author{\textbf{Author} \\ 
		AUGPath \\
		China University of Geosciences~(Wuhan) \\
		Department of Computer Science \\
		\today}

\maketitle
\newpage
\setcounter{tocdepth}{1}
\tableofcontents
\newpage


\section{概率中的基本概念}

\subsection{随机实验}

\begin{definition}
    随机试验是指: 
    \begin{itemize}
        \item 试验结果不止一个,但能明确所有的结果;
        \item 试验前不能预知出现哪种结果。
    \end{itemize}
\end{definition}

\begin{definition}
    随机事件: 对随机试验的结果的陈述称为\textbf{随机事件},简称\textbf{事件}.
\end{definition}

\begin{definition}
    样本空间: 随机试验的每个可能结果$\omega$称为一个\textbf{样本点}.全体样本点组成的集合$\Omega$称为\textbf{样本空间}.
    $$
        \Omega:=\{\omega\}.
    $$
    每个\textbf{随机事件}都可用样本空间的某个子集表示,通常记为$A, B, C$等.
    在随机试验中,当事件中的一个样本点出现时,称该事件\textbf{发生}.
\end{definition}

\subsection{事件的关系与运算}

\begin{definition}[事件的关系] 
    若事件$A$发生时,事件$B$一定发生.则称事件$A$包含于事件$B$(或事件$B$ \textbf{包含} $A$),记作
    $$A\subset B \ (\text{或}B\supset A)$$

    对任意事件$A$,有$\emptyset \subset A\subset \Omega$.

    若$A\subset B$,且$B\subset A$,则称事件$A$与$B$ \textbf{相等},记作$A=B$.
    
\end{definition}

\begin{definition}[事件的运算]
    
        “事件$A$、$B$至少有一个发生”
    称为事件$A$与$B$的\textbf{和}或\textbf{并}(union),记作
    $$A\cup B \ (\text{或}A+B)$$
    也就是
    $$A\cup B=\{\omega | \omega\in A \ \text{or}\ \omega\in B\}$$
\end{definition}

\end{document}
