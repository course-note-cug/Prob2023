\section{基本事件空间}
\paragraph{1. 基本事件空间}  考虑某项试验, 其结果在某一组条件下由有限种不同的结局 (现象) $\omega_1, \cdots, \omega_N$ 描绘. 关于这些结局的实际本性并不重要, 重要的是不同结局的个数 $N$ 是有限的. 我们把这些结局 $\omega_1, \cdots, \omega_N$ 称做基本事件, 而把一切结局的全体
$$
\Omega=\left\{\omega_1, \cdots, \omega_N\right\}
$$

称做 (有限) 基本事件空间, 或样本空间.

\begin{example}
    对于 “掷一枚硬币”, 基本事件空间由两个点组成:
$$
\Omega=\{\mathrm{Z}, \mathrm{F}\},
$$

其中 Z 表示出现 “正面”, 而 $\mathrm{F}$ 表示出现 “反面”. (这时假设, 诸如 “硬币在棱上立着”, “硬币丢失”.$\cdots \cdots$ 的情况不会出现.) 也就是假设不出现 “正面” 就出现 “反面”.

    将一枚硬币重复掷 $n$ 次, 基本事件空间为
$$
\Omega=\left\{\omega: \omega=\left(a_1, \cdots, a_n\right)\right\}, a_i=\text { Z或 } \mathrm{F},
$$

且基本事件的总数 $N(\Omega)=2^n$.
\end{example}

\paragraph{2. 事件及其关系和运算} 除基本事件空间的概念外, 现在引进重要概念事件. 事件的概念, 是建立所考察试验的各种概率模型 (“理论”) 的基础. 在试验的结果中, 试验者一般并不关心究竟出现了哪种具体的结局, 而关心出现的结局属于一切结局集合的哪个子集. 满足试验条件的一切子集 $A \subseteq \Omega$, 分为两种类型: “结局 $\omega \in A$ ” 或 “结局 $\omega \notin A$ ”. 我们称这样的子集 $A$ 为事件.

\begin{example}
    将一枚硬币重复掷三次, 一切可能结局的空间 $\Omega$, 由 8 个点构成:
$$
\Omega=\{000,001,010,011,100,101,110,111\}
$$

其中 0 和 1 分别表示郑出 “正面” 和 “反面”. 如果由 “一组条件” 可以记录 (确定、测量等) 所有 3 次郑硬币的结果, 则例如
$$
A=\{000,001,010,100\}
$$

就是事件: “将一枚硬币重复掷三次” 正面至少出现两次. 假如由 “一组条件” 只能确定第一次郑出的结果, 则 $A$ 已经不能称为事件, 因为关于 “试验的具体结局 $\omega$ 是否属于 $A "$, 既不能肯定也不能否定.
\end{example}

\subparagraph{事件的运算}
\begin{itemize}
    \item \emph {并.} 对于两个集合 $A$ 与 $B$, 称
    $$
    A \cup B=\{\omega \in \Omega: \omega \in A \text { 或 } \omega \in B\}
    $$
    为集合 $A$ 与 $B$ 的并, 表示由属于集合 $A$ 或 $B$ 的点形成的集合. 用概率论的语言, $A \cup B$ 表示事件 $\{$ 事件 $A$ 与 $B$ 至少出现一个\}.
    \item \emph {交.} 称 $A B$ 或
    $$
    A \cap B=\{\omega \in \Omega: \omega \in A \text { 且 } \omega \in B\}
    $$
    为两个集合 $A$ 与 $B$ 的交, 表示由既属于集合 $A$ 同时又属于集合 $B$ 的点形成的集合. 用概率论的语言, $A \cap B$ 表示事件 \{事件 $A$ 与 $B$ 同时出现 $\}$.
    \item \emph{补.} 如果 $A$ 是 $\Omega$ 的子集, 则称 $\bar{A}=\{\omega \in \Omega: \omega \notin A\}$ 为集合 $A$ 补, 表示 $\Omega$ 中不属于 $A$ 的点的集合.
    \item \emph{差.} 属于 $B$ 但不属于 $A$ 的点的集合称做 $B$ 与 $A$ 的差, 记作 $B \backslash A$. 那么, $\bar{A}=$ $\Omega \backslash A$. 用概率的语言, $\bar{A}$ 表示事件 “ $A$ 不出现”. 例如, 事件 $A=\{00,01,10\}$, 则事件 $\bar{A}=\{11\}$ 表示接连两次掷出反面.
\end{itemize}

\subparagraph{不可能事件和必然事件} 一般用 $\varnothing$ 表示空集. 概率论中空集 $\varnothing$ 称做不可能事件, 集合 $\Omega$ 自然称做必然事件. 
对于事件 $A$ 和 $B$, 若 $A \cap B=\varnothing$, 则称 $A$ 和 $B$ 不相容, 否则称 $A$ 和 $B$ 相容. 
\begin{itemize}
    \item \emph{和. } 当 $A$ 与 $B$ 不相交时 $(A B=\varnothing)$, 集合 $A$ 与 $B$ 的并称做集合 $A$ 与 $B$ 的和,记作 $A+B$.
\end{itemize}

\subparagraph{事件代数} 考虑集合 $A \subseteq \Omega$ 的某个集系 $\mathscr{B}_0$, 则利用集合运算 $\cup, \cap$ 与 $\backslash$ 可以由 $\mathscr{C}_0$ 构造新集系, 其中元素也是事件. 给这些事件补充上必然事件 $\Omega$ 和不可能事件 $\varnothing$, 得集系 $\mathscr{A}$, 则 $\mathcal{A}$ 是代数. 所谓 “代数” 即 $\Omega$ 的这样的集系, 满足
\begin{itemize}
    \item [1)] $\Omega \in \mathscr{A}$,
    \item [2)] 若 $A \in \mathscr{A}, B \in \mathscr{A}$, 则集合 $A \cup B, A \cap B, A \backslash B$ 也都属于 $\mathscr{C}$.
\end{itemize}

\begin{example}
    a) $\mathscr{b}=\{\Omega, \varnothing\}$集系由 $\Omega$ 和空集 $\varnothing$ 构成, 称做平凡代数;

b) $\mathscr{b}=\{A, \bar{A}, \Omega, \varnothing\}$事件 $A$ 产生的集系;

c) $\mathscr{A}=\{A: A \subseteq \Omega\}$ $\Omega$ 全部子集的集系 (包括空集 $\varnothing$ ).易见, 所有这些事件代数是按下面的原则得到的.
\end{example}

\subparagraph{分割 }我们称集系
$$
\mathscr{D}=\left\{D_1, \cdots, D_n\right\}
$$

构成集合 $\Omega$ 的一个分割, 而 $D_1, \cdots, D_n$ 是该分割的原子, 如果 $D_1, \cdots, D_n$ 非空且两两不相容, 而它们的和等于 $\Omega$ :
$$
D_1+\cdots+D_n=\Omega .
$$

\begin{example}
    例如, 假定集合 $\Omega$ 由 3 个点构成: $\Omega=\{1,2,3\}$, 则存在 5 个不同的分割:
$$
\begin{array}{ll}
\mathscr{D}_1=\left\{D_1\right\} & D_1=\{1,2,3\} ; \\
\mathscr{D}_2=\left\{D_1, D_2\right\} & D_1=\{1,2\}, D_2=\{3\} \\
\mathscr{D}_3=\left\{D_1, D_2\right\} & D_1=\{1,3\}, D_2=\{2\} ; \\
\mathscr{D}_4=\left\{D_1, D_2\right\} & D_1=\{2,3\}, D_2=\{1\} ; \\
\mathscr{D}_5=\left\{D_1, D_2, D_3\right\} & D_1=\{1\}, D_2=\{2\} ; D_3=\{3\} .
\end{array}
$$
\end{example}

如果考虑 $\mathscr{D}$ 中一切集合的并连同空集 $\varnothing$, 则得到的集系是代数, 称做 $\mathscr{D}$ 产生的代数, 记作 $\sigma(\mathscr{D})$. 于是, 代数 $\sigma(\mathscr{D})$ 的元素由空集 $\varnothing$ 与分割 $\mathscr{D}$ 之原子中集合的和组成.
这样, 如果 $\mathscr{D}$ 是 $\Omega$ 的某一分割, 则它与代数 $\mathscr{B}=\sigma(\mathscr{D})$ 一一对应.

逆命题也正确. 设 $\mathscr{B}$ 是有限空间 $\Omega$ 的子集的代数, 则存在唯一分割 $\mathscr{D}$, 其原子是代数 $\mathscr{B}$ 的元素, 并且 $\mathscr{B}=\sigma(\mathscr{D})$. 事实上, 假设集合 $\mathscr{D} \in \mathscr{B}$ 并且具有性质:对于任意 $B \in \mathscr{B}$, 集合 $D \cap B$ 要么与 $D$ 重合, 要么是空集. 那么, 这样集合 $D$ 的全体组成分割 $\mathscr{D}$ 并且具有所要求的性质 $B=\sigma(\mathscr{D})$. 对于例 a), 取只含一个集合的 $D_1=\Omega$ 平凡分割; 对于例 b), $\mathscr{D}=\{A, \bar{A}\}$. 对于例 c), $\mathscr{D}$ 是只含一个点的集合 $\left\{\omega_i\right\}, \omega_i \in \Omega$ 的最细小分割, 即 $\mathscr{D}$ 产生的代数是 $\Omega$ 的一切子集的代数.

对两个分割 $\mathscr{D}_1$ 和 $\mathscr{D}_2$, 如果 $\sigma\left(\mathscr{D}_1\right) \subseteq \sigma\left(\mathscr{D}_2\right)$, 则称分割 $\mathscr{D}_2$ 比分割 $\mathscr{D}_1$ “细小", 记作 $\mathscr{D}_1 \preccurlyeq \mathscr{D}_2$.

像前面一样, 假设空间 $\Omega$ 有有限个点 $\omega_1, \cdots, \omega_N$ 构成, 记 $N(\mathscr{A})$ 为例 $\left.\mathrm{c}\right)$ 中组成体系 $\mathscr{C}$ 的集合的总数. 我们证明 $N(\mathscr{A})=2^N$. 事实上, 每一个非空集合 $A \in \mathscr{A}$可以表示为 $A=\left\{\omega_{i_1}, \cdots, \omega_{i_k}\right\}(1 \leqslant k \leqslant N)$, 其中 $\omega_{i_j} \in \Omega$. 将该集合与由 0 或 1 形成的序列
$$
(0, \cdots, 0,1,0, \cdots, 0,1, \cdots),
$$

其中在编号为 $i_1, \cdots, i_k$ 的位置上为 1 , 而在其余位置上为 0 . 那么, 对于固定的 $k$,形如 $A=\left\{\omega_{i_1}, \cdots, \omega_{i_k}\right\}$ 的不同集合 $A$ 的总数等于 $k$ 个 1 ( $k$ 个不可辨质点) 分配 $N$ 个位置 ( $N$ 个箱子) 不同分法的总数. 根据表 1-4 的情形(4), 这样分法的总数等于 $\mathrm{C}_N^k$. 由此可见
$$
N(\mathscr{b})=1+\mathrm{C}_N^1+\cdots+\mathrm{C}_N^N=2^N,
$$

其中包括空集 $\varnothing$.

\paragraph{3. 概率空间}
为建立只有有限种可能结局的随机试验的概率模型 (理论), 我们暂时完成了最初的两步: 引进了基本事件空间 $\Omega$, 并建立了 $\Omega$ 子集的某种体系 $\mathscr{b}$ —代数, 其中的子集称做事件. 有时把 $\mathscr{E}=(\Omega, \mathscr{A})$ 等同于试验. 现在进行下一步: 赋予每一个基本事件 (每一种结局或现象) $\omega_i \in \Omega(i=1, \cdots, N)$ 某种 “权”, 记作 $p\left(\omega_i\right)$ 或 $p_i$, 称做基本事件 (结局) $\omega_i$ 的概率. 假设 $p\left(\omega_i\right)$ 满足条件:
a) $0 \leqslant p\left(\omega_i\right) \leqslant 1$ (非负性),
b) $p\left(\omega_1\right)+\cdots+p\left(\omega_N\right)=1$ (规范性).
从给定的基本事件 $\omega_i$ 的概率 $p\left(\omega_i\right)$ 出发, 按公式
$$
\mathbf{P}(A)=\sum_{\left\{i: \omega_i \in A\right\}} p\left(\omega_i\right)
$$

定义任意事件 $A \in \mathscr{A}$ 的概率.

\begin{definition}
    通常称
$$
(\Omega, \mathscr{A}, \mathbf{P})
$$

为 “概率空间”, 其中 $\Omega=\left\{\omega_1, \cdots, \omega_N\right\}, \mathscr{A}$ 是 $\Omega$ 的子集的代数, 而 $\mathbf{P}=\{\mathbf{P}(A): A \in$ $\mathscr{B}\}$. 概率空间决定 (定义) 只有有限种可能结局 (基本事件) 的, 随机试验的概率模型 (理论). 
\end{definition}

显然, $\mathbf{P}\left(\left\{\omega_i\right\}\right)=p\left(\omega_i\right)(i=1, \cdots, N)$. 

\paragraph{4. 古典概率}在一些具体的情形下建立概率模型时, 给出基本事件空间 $\Omega$ 和代数 $\mathscr{A}$, , 一般并不复杂. 这时, 在初等概率论里, 一般用 $\Omega$ 的全部子集的代数当做代数 $\mathscr{A}$. 较困难的问题, 是定义基本事件的概率. 实际上, 对这个问题的回答已经超出了概率论的范围, 我们不在此过多地讨论这个问题. 我们的基本任务, 并不是如何赋予某一个试验基本事件的概率, 而是根据基本事件的概率, 计算复合事件 ( $\mathscr{C}$ 中的事件) 的概率.

从数学观点来看十分清楚, 对于有限基本事件空间, 只要赋予基本事件 $\omega_1, \cdots$, $\omega_N$ 以满足 $p_1+\cdots+p_N=1$ 的非负实数 $p_1, \cdots, p_N$, 就可以得到一切可以想象的 (有限的) 概率空间.

对于具体的情形, 所确定的数值的正确性, 可以一定程度地利用以后将要介绍的大数定律来验证. 在上述情形下, 根据大数定律, 对于给定的在相同条件下进行的较长 “独立” 试验系列, 基本事件出现的频率 “十分接近” 它们相应的概率.

鉴于赋予试验基本事件以概率值的困难, 我们指出, 存在许多实际情形, 在这些情形下由于对称性或均衡性的直观, 把一切可能出现的基本事件视为等可能的是合理的. 因此, 假如基本事件空间 $\Omega$ 由点 $\omega_1, \cdots, \omega_N$ 构成, 其中 $N<\infty$, 则
$$
p\left(\omega_1\right)=\cdots=p\left(\omega_N\right)=\frac{1}{N},
$$

从而对于任何事件 $A \in \mathscr{A}$,
$$
\mathbf{P}(A)=\frac{N(A)}{N},
$$

其中 $N(A)$ 是事件 $A$ 所含基本事件的个数.
这样求概率的方法称做古典型方法. 显然, 这时求概率 $\mathbf{P}(A)$ 归结为计算导致事件 $A$ 的基本事件的个数. 这一般用排列组合的方法来实现. 