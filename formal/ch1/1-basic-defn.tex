\section{基本事件空间}
\paragraph{1. 基本事件空间}  考虑某项试验, 其结果在某一组条件下由有限种不同的结局 (现象) $\omega_1, \cdots, \omega_N$ 描绘. 关于这些结局的实际本性并不重要, 重要的是不同结局的个数 $N$ 是有限的. 我们把这些结局 $\omega_1, \cdots, \omega_N$ 称做基本事件, 而把一切结局的全体
$$
\Omega=\left\{\omega_1, \cdots, \omega_N\right\}
$$

称做 (有限) 基本事件空间, 或样本空间.

\begin{example}
    对于 “掷一枚硬币”, 基本事件空间由两个点组成:
$$
\Omega=\{\mathrm{Z}, \mathrm{F}\},
$$

其中 Z 表示出现 “正面”, 而 $\mathrm{F}$ 表示出现 “反面”. (这时假设, 诸如 “硬币在棱上立着”, “硬币丢失”.$\cdots \cdots$ 的情况不会出现.) 也就是假设不出现 “正面” 就出现 “反面”.

    将一枚硬币重复掷 $n$ 次, 基本事件空间为
$$
\Omega=\left\{\omega: \omega=\left(a_1, \cdots, a_n\right)\right\}, a_i=\text { Z或 } \mathrm{F},
$$

且基本事件的总数 $N(\Omega)=2^n$.
\end{example}

\paragraph{2. 事件及其关系和运算} 除基本事件空间的概念外,现在引进重要概念事件. 事件的概念, 是建立所考察试验的各种概率模型 (“理论”) 的基础. 在试验的结果中, 试验者一般并不关心究竟出现了哪种具体的结局, 而关心出现的结局属于一切结局集合的哪个子集. 满足试验条件的一切子集 $A \subseteq \Omega$, 分为两种类型: “结局 $\omega \in A$ ” 或 “结局 $\omega \notin A$ ”. 我们称这样的子集 $A$ 为事件.
