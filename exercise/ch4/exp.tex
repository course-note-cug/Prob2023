\section*{练习题: 数学期望}

\begin{exercise}
    某个大楼有10层, 某次有25人在一楼搭乘电梯上楼. 假设每人都等可能地在2到10层中的任何一层出电梯, 并且出电梯与否互相独立. 同时在2到10层中没有人上电梯. 并且电梯只有在有人要出电梯的时候才停止. 求电梯停下的总次数的数学期望. 
\end{exercise}

\begin{solution*}
每个人在第$i$层下的概率为$P_i=1/9, i=1,2,\cdots, 10$. 记第$k$个人在第$i$层下电梯记作$A_{k,i}$. 那么对于任意的$i$, $P(A_k)=1/9, P(\overline{A_k})=8/9. (k=1,2,\cdots, 25)$. 又因为$A_{1, i}, \cdots, A_{25, i}$相互独立, 那么第$i$层无人下电梯的概率为
$$
P\left(\prod_{i=1}^{25} \overline{A_k}\right) = \prod_{i=1}^{25}P\left(\overline{A_k}\right) = \left(\frac89\right), (k=1,2,\cdots, 25).
$$

设$X_i$是指示第$i$层有没有人下的示性函数, 那么, $P(X=0)=(8/9)^{25}$, $P(X=1)=1-(8/9)^{25}$.

因此电梯停的总次数为$X=\sum_{i=2}^{10}X_i=9\times\left(1-(8/9)^{25}\right)$
\end{solution*}

\begin{exercise}
    设随机变量$X,Y$相互独立, 且都服从$N(\mu, \sigma^2)$. 设$Z=\max\{X,Y\}$. 求$\Ep{Z}$.
\end{exercise}

\begin{solution*}
    正则化变量. 令$U:=\frac{X-\mu}{\sigma}, V:=\frac{Y-\mu}{\sigma}$. 那么$X=\sigma U+\mu, Y=\sigma V+\mu$. 由于$X,Y$独立, 因此$U,V$独立. 且 $U \sim N(0,1), V \sim N(0,1)$. 发现$Z=\max \{X, Y\}=\max \{\sigma U+\mu, \sigma V+\mu\}=\sigma \max (U, V)+\mu$.

    要求$\max (U, V)=\frac{1}{2}(U+V+|U-V|)$, 我们要知道$|U-V|$的分布. 令$T:=U-V \sim N(0,2)$. 考虑$$E(|T|)=\int_{-\infty}^{+\infty}|t| \frac{1}{\sqrt{2 \pi} \cdot \sqrt{2}} e^{-\frac{t^2}{2 \times 2}} d t=\frac{2}{\sqrt{\pi}}.$$ 因此得到$E[\max (U, V)]=\frac{1}{2}(E U+E V+E|U-V|)=\frac{1}{\sqrt{\pi}}$. 从而$E[\max \{X, Y\}]=\sigma E[\max (U, V)]+\mu=\frac{\sigma}{\sqrt{\pi}}+\mu$. 
\end{solution*}

\begin{exercise}
    对于任意的整数$k>1$, 证明$\Ep{X^k}\geq (\Ep{X})^k$.
\end{exercise}

\begin{solution*}
    使用Jensen不等式. 令$f(x):=x^n$. 
\end{solution*}

\begin{exercise}
    (负二项分布) 现在考虑投掷一枚硬币, 直到第$k$次出现正面的投掷次数的$X$的分布. 其中每次投掷硬币出现是独立的, 概率为$p$. 证明: 对于$n>k$, 这个分布是
    $$
    P(X=n) = \binom{n-1}{k-1}p^k(1-p)^k.
    $$
\end{exercise}

\begin{solution*}
    可以考虑在出现最后一次正面之前, 我们需要在$n-1$个空格里面插入$k-1$个正面. 其余的是反面. 由此可以得到这个等式的意义. 
\end{solution*}

\begin{exercise}
    对于一枚每次投出正面的概率是$p$的硬币, 每次投掷之间互相独立, 直到第$k$次正面出现的时候, 期望的投掷次数是多少? 
\end{exercise}

\begin{solution*}

    方法1. 令$\Ep{N_k}:=$有$n$个正面投掷的期望数. 令$X_1$是第一次投掷的结果. 如果第一次的结果是反面, 那么还会回到$N_k$, 但是期望会加1. 反之, 就会到达$N_{k-1}$, 同样期望会加一. 也就是
    $$
\Ep{N_k \mid X_1=T}=1+\Ep{N_k}
$$
$$
\Ep{N_k \mid X_1=H}=1+E\left[N_{k-1}\right]
$$

因此有表达式$\Ep{N_k}=\Ep{N_k \mid X_1=H} \cdot P\left(X_1=H\right)+\Ep{N_k \mid X_1=T} \cdot P\left(X_1=T\right)$. 解之, 得到$\Ep{N_k}=\Ep{N_{k-1}}+\frac{1}{p}$. 因此是 $\Ep{N_k}=n/p$.

    高中生: 考虑$f_i$代表现在已经扔出了$i$个正面, 还需要扔的期望数. 根据这个定义$f_k=0$, $f_{k-1}=(1-p)(f_{k-1}+1)+p(f_k+1)$. 

\end{solution*}


\begin{exercise}
    我们在一个$n$张不同的卡牌中进行带放回的抽卡, 抽出每一张卡的概率相等. 请问我们期望抽取多少次, 才能把所有的卡牌见到一遍? 
\end{exercise}

\begin{solution*}
    可以考虑当前已经见到了$k$个卡牌, 那么还有$k/n$的概率回到当前的状态, $(n-k)/n$的概率到达见到的少一个的状态. 
\end{solution*}

\begin{exercise}
    我们一遍一遍地投掷一个均匀的筛子. 在连续出现两个6之前, 期望的投掷次数是多少? (答案不是36).
\end{exercise}
\begin{solution*}
    想办法搞明白各个你的划分之间的转移关系. 或者看Markov链的相关内容. 
\end{solution*}


\begin{exercise}
     数字$1,2,\cdots, n$可以用如下的排列函数$\pi:[1..n] \to [1..n]$表示. 其中, $\pi(i)$是$i$在这个排列中的序号. 排列$\pi$的不动点是满足$\pi(x)=x$的$x$的集合. 求从所有排列中选取一个排列的不动点的期望. 
\end{exercise}

\begin{solution*}
    一个简单的方法: 定义$f(\sigma)$ 为随机一个排列$\sigma$的不动点的数量. 
    那么$$
    \mathbb{E}[{f}(\sigma)]=\mathbb{E}\left[\sum_{{i}=1}^{{n}} \mathbf{1}_{\sigma({i})={i}}\right]=\sum_{{i}=1}^{{n}} \mathbb{E}\left[1_{\sigma({i})={i}}\right]
    $$
    对于每个$i$,$i$ 成为 $\sigma$ 的一个不动点的概率等于 $\sigma(i)$ 等于 $i$ 的概率,因此等于 $\frac{1}{n}$。利用这个结果,我们得到 $\mathbb{E}[f(\sigma)]=1$。
\end{solution*}

\begin{exercise}
    假设 $a_1, a_2, \dots, a_n$ 是 {1, 2, $\dots$, $n$} 的一个随机排列,等可能地是 $n!$ 种可能的排列之一。当对列表 $a_1, a_2, \dots, a_n$ 进行排序时,元素 $a_i$ 必须移动 $|a_i - i|$ 个位置才能到达其在排序后的位置。求出$$
    \mathbf{E}\left[\sum_{i=1}^n\left|a_i-i\right|\right]
    $$
    也就是元素移动的期望总距离. 
\end{exercise}

\begin{solution*}
    和上面一样, 使用期望的线性性. 
\end{solution*}

\begin{exercise}
    排列 $\pi: [1, n] \rightarrow [1, n]$ 可以表示为一张图. 我们可以这样做: 为每个数字 $i, i = 1, \dots, n$ 设立一个顶点。如果排列将数字 $i$ 映射到数字 $\pi(i)$,则从顶点 $i$ 到顶点 $\pi(i)$ 绘制一个有向边。这导致一个由不相交环构成的图。注意,其中一些环可能是自环。在 $n$ 个数字的随机排列中,期望的环的数量是多少?
\end{exercise}

\begin{solution*}
    设 $h(n)$ 为 [n] 的随机排列中环的平均数;我声称 $h$ 满足以下递推关系:$h(n)=\frac{n-1}{n}h(n-1)+\frac{1}{n}(h(n-1)+1)$。因为:
    设 $\pi$ 是 {2,3,…,$n$} 的任意排列,用环形式表示。假设忽略括号,从左到右排列的 $\pi$ 的条目是 $\pi_1,\dots,\pi_{n-1}$。现在,在 $\pi$ 中以以下两种方式之一插入 $1$。
    \begin{itemize}
        \item 在 $\pi_1$ 的左边作为 $(1)$,形成一个独立的环。
        \item 立即在 $\pi_k$ 的某个 $k=1,\dots,n-1$ 的位置之后插入,与 $\pi_k$ 同一个环中。
    \end{itemize}
    每个 $[n]$ 的排列都可以唯一地通过这种方式从 {2,…,$n$} 的一个唯一排列 $\pi$ 获得。

一个 {2,…,$n$} 的随机排列的平均环的数量当然是 $h(n-1)$。递推关系 (1) 立即由以下事实推出:上述操作 (1) 增加了 $1$ 个环,并占据了所有情况中的 $\frac{1}{n}$,而操作 (2) 不改变环的数量,并占据了剩余的 $\frac{n-1}{n}$ 情况。

计算可知, 
$$
h(n)=H_n=\sum_{k=1}^n \frac{1}{k} \text {. }
$$
\end{solution*}