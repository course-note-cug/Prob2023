\section*{练习题: 方差. 矩. 协方差. 中位数.}
\begin{exercise}
    假设我们扔一个均匀的骰子100次. 令$X$为这100次中出现的次数和. 使用Chebyshev不等式得到$P(|X-350| \geq 50)$.
\end{exercise}
\begin{solution*}
    Chebyshev不等式: $P(|X-\mathbb{E}[X]| \geq a) \leq \frac{\operatorname{Var}[X]}{a^2}$
\end{solution*}

\begin{exercise}
    每次抛硬币,正面朝上的概率为 \(p\),并且相互独立。求直到第 \(k\) 次出现正面的抛硬币次数的方差。 
\end{exercise}

\begin{solution*}
    计算得到$\Ep{X}=\frac{i}{1-p}=\sum_{k=0}^{\infty} i p (1 - p)^{k - 1}$. 并且$\Ep{X^2}=\sum_{k=0}^{\infty} i^2 p (1 - p)^{k - 1}=\frac{i^2}{1-p}$. 那么方差就是$\Ep{X^2}-\Ep{X}^2=\frac{i^2}{1-p}-(\frac{i}{1-p})^2$.
\end{solution*}

\begin{exercise}
一个简单的股票市场模型表明,每天,价格为 \( q \) 的股票以概率 \( p \) 增长到 \( qr \),以概率 \( 1 - p \) 下跌到 \( \frac{q}{r} \)。假设我们从价格为1的股票开始,找到 \( d \) 天后股票价格的期望值和方差的公式。
\end{exercise}

\begin{solution*}
    类似于二项分布. 
\end{solution*}

\begin{exercise}
    当一个排列 \(\pi: [1, n] \rightarrow [1, n]\) 满足 \(\pi (x) = x\) 时,\(x\) 是该排列的一个不动点。求从所有排列中随机选择的排列中不动点数量的方差。
\end{exercise}

\begin{solution*}
    可以使用$D(X)=\Ep{X^2}-\Ep{X}^2$.
\end{solution*}

