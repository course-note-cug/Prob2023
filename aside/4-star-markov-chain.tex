    \subsection{练习: 仅仅依赖上一个状态的Markov链}

        \begin{wrapfigure}{l}{0.4\textwidth}
        \begin{center}
            \psscalebox{0.6 0.6} % Change this value to rescale the drawing.
{
\begin{pspicture}(0,-5.46)(8.3,0.42)
\definecolor{colour0}{rgb}{0.0,0.5019608,0.0}
\definecolor{colour1}{rgb}{0.9019608,0.3019608,0.3019608}
\psframe[linecolor=black, linewidth=0.04, dimen=outer](8.3,0.14)(0.0,-5.46)
\rput[bl](3.7,-1.46){A}
\rput[bl](1.9,-4.26){B}
\rput[bl](5.2,-4.36){C}
\rput[bl](0.1,0.14){Dorm}
\psellipse[linecolor=black, linewidth=0.04, dimen=outer](3.8,-1.31)(0.7,0.65)
\psellipse[linecolor=black, linewidth=0.04, dimen=outer](2.0,-4.16)(0.7,0.7)
\psellipse[linecolor=black, linewidth=0.04, dimen=outer](5.35,-4.21)(0.65,0.65)
\rput[bl](2.3,-0.66){0.2}
\psline[linecolor=black, linewidth=0.04, arrowsize=0.05291667cm 2.0,arrowlength=1.4,arrowinset=0.0]{->}(4.3,-1.76)(5.2,-3.56)
\psline[linecolor=black, linewidth=0.04, arrowsize=0.05291667cm 2.0,arrowlength=1.4,arrowinset=0.0]{->}(3.3,-1.86)(2.2,-3.56)
\rput[bl](4.7,-2.36){0.2}
\rput[bl](2.6,-2.56){0.6}
\psline[linecolor=colour0, linewidth=0.04, arrowsize=0.05291667cm 2.0,arrowlength=1.4,arrowinset=0.0]{->}(2.5,-3.66)(3.6,-1.96)
\psline[linecolor=colour0, linewidth=0.04, arrowsize=0.05291667cm 2.0,arrowlength=1.4,arrowinset=0.0]{->}(2.7,-4.36)(4.7,-4.36)
\rput[bl](2.9,-3.06){\textcolor{colour0}{0.3}}
\rput[bl](3.4,-4.46){\textcolor{colour0}{0.7}}
\psline[linecolor=colour1, linewidth=0.04, arrowsize=0.05291667cm 2.0,arrowlength=1.4,arrowinset=0.0]{->}(5.0,-3.66)(4.1,-1.86)
\rput[bl](4.3,-2.96){\textcolor{red}{0.5}}
\rput[bl](6.7,-4.46){\textcolor{red}{0.5}}
\psbezier[linecolor=colour1, linewidth=0.04, arrowsize=0.05291667cm 2.0,arrowlength=1.4,arrowinset=0.0]{->}(5.9,-3.96)(6.7516584,-3.4359026)(7.241421,-2.8700504)(7.1,-3.86)(6.9585786,-4.8499494)(6.796546,-4.5769544)(5.8,-4.66)
\psbezier[linecolor=black, linewidth=0.04, arrowsize=0.05291667cm 2.0,arrowlength=1.4,arrowinset=0.0]{->}(3.4,-0.76)(2.9143572,0.114157274)(3.0472136,0.4344272)(2.6,-0.46)(2.1527865,-1.3544272)(2.1636708,-1.0088766)(3.1,-1.36)
\end{pspicture}
}
        \end{center}
        \caption{班主任探访的规律}
        \label{fig:meeting-supervisor-pattern}
    \end{wrapfigure}
    \paragraph{第一幕 Markov链带来的稳定分布} 有三个喜欢自学的好奇学生A,B,C. 他们整天喜欢在宿舍学习. 这自然会使他们的班主任不开心. 这就会导致班主任每天就去他们的寝室去观察, 以确保他们确实在学习. 
    
    由于他们足够观察的足够仔细, 经过一段时间的观察, 他们认为班主任会按照如\cref{fig:meeting-supervisor-pattern}这样的的规律来``探访''他们.  如果我们只看$A$同学和黑色的箭头, 其意义如下: 

    若今天班主任找$A$同学, 那么明天: 
    \begin{itemize}
        \item 0.6的概率找$B$;
        \item 0.2的概率找$C$;
        \item 0.2的概率找$A$.
    \end{itemize}

    这里一个关键的假设是: 未来的状态仅仅取决于\emph{当前状态}. 也就是$P(X_{n+1}=x|X_n=x_n)$.
    
    在这个假设下, 如果我们已知班主任老师第一天找$B$, 第二天找$A$, 那么请问, 第三天找$C$的概率为多大? 按照我们刚刚的假设, 第三天找谁只与第二天有关. 因此第二天找了$A$, 自然第三天找$C$的概率为0.2. 

    自然, 日子一天天的过去, 他们想知道从长期来看, 班主任来``探访''他们的概率是多少. 比如, 他们记录了10天以内班主任的行踪轨迹: 
    $$
    A \to B \to A\to C \to A \to C\to C\to C\to A\to B.
    $$

    在这10天里面, $P(A\text{被找到}), P(B\text{被找到}),P(C\text{被找到})$的概率分别为4/10, 2/10, 4/10. 因此, 他们面临的第一个问题就是在前$n$天($n\to \infty$)的时候期望的结果是多少. 

    事实上, 上面的状态转换关系是一个图. 可以用$3\times 3$的矩阵表示如下(第一行和第一列为标示,$M[C][B]$表示第一天老师在$C$, 第二天去$B$的概率.):   
    $$
M=\left[\begin{array}{cccc}
&\text { A } & \text {B} & \text {C} \\
\text{A}&0.2 & 0.6 & 0.2 \\
\text{B}&0.3 & 0 & 0.7 \\
\text{C}& 0.5 & 0 & 0.5
\end{array}\right]
$$以及最初的状态(比如找了B, 初始状态用$1\times 3$的矩阵表示为$
\pi_0=\left[\begin{array}{ccc}
\text{A} & \text{B} & \text {C} \\
0 & 1 & 0
\end{array}\right]
$)若达到了稳定状态, 意味着找到这样的一个$\pi$, 使得$\pi A=\pi$. 这就是矩阵的特征向量.在这个情况下, 这个稳定的$\pi$为$\pi=\left[\frac{25}{71}, \frac{15}{71}, \frac{31}{71}\right]$.

为什么这样是对的呢? 我们考虑到达了状态$j$,从$i$状态开始,在 $n$ 步后, 定义为$P_{ij}(n)$. 同样如果我们有一个矩阵$$
M=\left[\begin{array}{ccc}
A_{11} & A_{12} & A_{13} \\ 
A_{21} & A_{22} & A_{23} \\ 
A_{31} & A_{32} & A_{33} \\ 
\end{array}\right]
$$
那么用一步从0到达2只有一种可能的情况:$P_{02}(1)=A_{02}$.  两步从0到2到达的一种情况就要分情况了. 
$$
\begin{aligned}
 P_{02}(2)&=A_{01} \times A_{12}+A_{00} \times A_{02}+A_{02} \times A_{22} \\
=& \left[\begin{array}{lll}
A_{00} & A_{01} & A_{02}
\end{array}\right] \times\left[\begin{array}{l}
A_{02} \\
A_{12} \\
A_{22}
\end{array}\right] \\
&
\end{aligned}
$$
类似的, 有$$
P_{10}(2)=\left[\begin{array}{lll}
A_{10} & A_{11} & A_{12}
\end{array}\right] \times\left[\begin{array}{c}
A_{00} \\
A_{10} \\
A_{20}
\end{array}\right]
$$
把每个状态都列举一遍, 我们就可以使用矩阵乘法. 也就是$$
P(2)=A^2=\left[\begin{array}{ccc}
0.5 & 0.2 & 0.3 \\
0.6 & 0.2 & 0.2 \\
0.1 & 0.8 & 0.1
\end{array}\right] \times\left[\begin{array}{ccc}
0.5 & 0.2 & 0.3 \\
0.6 & 0.2 & 0.2 \\
0.1 & 0.8 & 0.1
\end{array}\right]
$$也就是$P_{i_j}(n)=A_{i j}^n$. 

\paragraph{第二幕 Markov链的期望时间} 现在, 他们摸清楚了每一次他们老师来的模式. 
现在他们的想法是...