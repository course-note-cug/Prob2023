% !TEX root = main.tex

\subsection*{多知道一点: 概率测度和概率空间}

\subsubsection*{Bell函数}

前面在\cref{def:prob}说过, 公理化的概率定义是一个对于样本空间里面关注的事件映射到一个实数的情况. 

我们希望选取一类可以定义概率的集合以及一类可以作为随机变量的函数. 我们先从最简单的函数
开始: 集合 $A$ 的指示函数.

\begin{definition*}
    集合 $A$ 的指示函数是在集合$A$上面的所有点取值为1, 而在 $A$ 的补集上面取值为
    0的函数. 用 $\one_A$表示.
\end{definition*}

我们来考虑交集 $C=A\cup B$. 它的指示函数在$\one_A$或者$\one_B$为0的时候等于0. 也就是
说, 它的指示函数$\one_C=\inf(\one_A, \one_B)$. 也就是两个之间的较小者. 我们这里使用$\inf$是为了再连续的情况下, 我们同样能够轻松地处理问题. 

\begin{asidebox}
    回顾高等数学中有界的定义: 
    \begin{definition*}
        设 $X \subset \mathbb{R}$ 是一集合; 如果存在一数 $c \in \mathbb{R}$, 使一切 $x \in X$ 都满足 $x \leq c(c \leq x)$, 就说集合 $X$ 是\emph{上 (下) 有界集}.
    这时, 数 $c$ 就叫做 $X$ 的一个\emph{上 (下) 界}.
    既有上界又有下界的集合叫做\emph{有界集}.
    \end{definition*}


    以及我们熟知的最大值和最小值的定义: 
    \begin{definition*}
        $$\begin{aligned} & (a=\max X):=(a \in X \wedge \forall x \in X(x \leq a)), \\ & (a=\min X):=(a \in X \wedge \forall x \in X(a \leq x)) .\end{aligned}$$
    \end{definition*}

    上确界就是最小的上界. 这样的想法用符号这样描述: 
    \begin{definition*}
         集合 $X \subset \mathbb{R}$ 的上界中的最小者, 叫做 $X$ 的上确界, 写作 $\sup X$ 或 $\sup _{x \in X} x$. 即: $$(s=\sup X):=\forall x \in X\left((x \leq s) \wedge\left(\forall s^{\prime}<s. \exists x^{\prime} \in X\left(s^{\prime}<x^{\prime}\right)\right)\right).$$
    \end{definition*}

    同样可以知道下确界是最大的下界. 同样可以写出类似的定义. 此处不再赘述.
\end{asidebox}

为了用好这个重要的想法, 我们把在每一点 $x$ 上等于 $f(x)$ 和 $g(x)$ 的值中的较小者的函数记为 $f \cap g$, 上述两个函数的较大者记作$f \cup g=\sup (f, g)$. 它们适用于任意个数的函数. 有时候我们可以像求和那样简记符号: 
$$
f_1 \cap \cdots \cap f_n:=\bigcap_{k=1}^n f_k, \quad f_1 \cup \cdots \cup f_n:=\bigcup_{k=1}^n f_k 
$$

接下来我们来考虑: 对于每一个$x$, 它是不是有唯一的取值? 如果是的话, 我们就可以使用这个来得到一个新的函数了. 

现在来考虑新东西, 也就是这个无穷序列 $\left\{f_n\right\}$. \cref{def:capcupdef}中定义的函数依赖于 $n$, 而且函数值关于$n$单调, 因而极限 $\bigcup_{k=1}^{\infty} f_k$ 和 $\bigcap_{k=1}^{\infty} f_k$ 是完全确定的(可能因为没有上界从而是无穷). 所以我们说, 对于固定的 $j$,
$$
w_j=\bigcap_{k=j}^{\infty} f_k
$$
是单调函数列 $f_j \cap \cdots \cap f_{j+n}$ 的极限. 

\begin{definition}[函数的``交''和``并''可以构成新函数]
    在每一点 $x$ 上等于 $f(x)$ 和 $g(x)$ 的值中的较小者的函数记为 $f \cap g$, 上述两个函数的较大者记作$f \cup g=\sup (f, g)$.
    \label{def:capcupdef}
    定义
    $$
f_1 \cap \cdots \cap f_n:=\bigcap_{k=1}^n f_k, \quad f_1 \cup \cdots \cup f_n:=\bigcup_{k=1}^n f_k 
$$
在每一点 $x$, 上述的两个函数分别等于对应的 $n$ 个值 $f_1(x), \cdots, f_n(x)$ 中的最小值和最大值.  
\end{definition}

回到刚才的讨论. 对于固定的 $j$, 定义$w_j:=\bigcap_{k=j}^{\infty} f_k$是单调函数列 $\{f_j, f_j\cap f_{j+1},\cdots,f_j \cap \cdots \cap f_{j+n}, \cdots\}$ 的极限. 由于我们在取最小值, 序列 $\left\{w_j\right\}$ 本身也是单调的. 也就是$w_n=w_1 \bigcup \cdots \bigcup w_n$. 根据定义, $w_n$ 是数列 $f_n(x), f_{n+1}(x), \cdots$ 的最大下界 (下确界). 因而 $w_n$ 的极限和 $\liminf f_n$ 相同, 于是
$$
\liminf f_n=\bigcup_{j=1}^{\infty} \bigcap_{k=j}^{\infty} f_k .
$$
这样, $\liminf$ 是通过两次取单调序列的极限得到的. 对于 $\lim \sup f_n$, 我们只需把上式中的 $\cap$ 与 $\cup$ 交换一下.

\begin{example}
    用$\one_A$做一个例子: 当且仅当$\one_A=\lim\one_{A_n}$的时候, 我们会写$A=\lim A_n$. 这表示当且仅当 $A$ 的每一点除有限个集合外属于所有的 $A_n$, 余集$A^{\prime}$ 的每一点最多属于有限个 $A_n$ 时, 集列 $\left\{A_n\right\}$ 收敛到 $A$.
\end{example}

我们刚刚已经见到了最基本的函数. 我们希望由一些基本函数构造出更加抽象的函数. 具体的, Euler
关于函数概念的现代说法是比较合适的. 

设已经给定了一个连续函数, 构造新函数的唯一有效方法是通过取极限. 我们对具有下列性质的函数类 $\mathfrak{B}$ 感兴趣: (1) 每个连续函数属于 $\mathfrak{B}$; (2) 如果 $f_1, f_2, \cdots$ 属于 $\mathfrak{B}$, 且极限 $f(x)=\lim f_n(x)$ 对一切 $x$ 都存在, 则 $f$ 属于 $\mathfrak{B}$. 称这类函数\emph{在点态极限下封闭}. 所以我们说: 

\begin{definition*}
    包含一切连续函数的最小封闭函数类称为贝尔类, 并用 $\mathfrak{B}$ 表示它. $\mathfrak{B}$ 中的函数称为贝尔函数. 
\end{definition*}

\subsubsection*{区间函数与在 $\mathbf{R}^r$ 上的积分}

首先回顾我们在中学的时候学过的表示区间的方法: $(a,b); [a, b]; (a,b]; [a,b)$. 把 $a$ 或 $b$ 的一个或几个坐标换为 $\pm \infty$ 的极限情形是允许的; 特别地, 整个空间是区间 $(-\infty, \infty)$.

点函数 $f$ 对单个点赋予值 $f(x)$. 集函数 $F$ 对集合或空间的区域赋予值. 如$\mathbf{R}^3$ 中的体积、 $\mathbf{R}^2$ 中的面积或 $\mathbf{R}^1$ 中的长度都是集函数的典型的例子. 此外, 概率是我们最关心的一种特殊情形. 我们只对具有下列性质的集函数感兴趣: 如果集合 $A$被分成两个集合 $A_1$ 和 $A_2$, 那么 $F\{A\}=F\left\{A_1\right\}+F\left\{A_2\right\}$. 这样的函数称为\emph{可加集函数}.

\begin{definition*}((有限)可加的)
    设 $F$ 是一个对每个区间 $I$ 赋予有限值 $F\{I\}$ 的函数. 这样的函数称为 (有限) 可加的, 如果对于把区间 $I$ 分成有限多个不相交的区间 $I_1, \cdots, I_n$ 的任一划分, 
$$
F\{I\}=F\left\{I_1\right\}+\cdots+F\left\{I_n\right\} .
$$
\end{definition*}

\begin{example}
    $\mathbf{R}^1$ 中的分布. 在前面的叙述中, 我们考虑了对于点 $a_1, a_2, \cdots$ 赋予概率 $p_1, p_2, \cdots$的离散概率分布. 这里 $F\{I\}$ 是 $I$ 中的所有点 $a_n$ 的概率 $p_n$ 之和.

    如果 $G$ 是任意一个从 $-\infty$ 为 0 增加到 $\infty$ 为 1 的单调增连续函数, 则可以定义 $F\{(a, b)\}=G(b)-G(a)$.
\end{example}


